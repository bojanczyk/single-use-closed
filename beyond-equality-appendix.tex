\section{Beyond equality}
\label{sec:beyond-equality-appendix}

As mentioned in the main body of the paper, the proof of Theorem~\ref{thm:single-use-closed} extends without any difficulty to the case of relational structures. We only prove the decidability result from Theorem~\ref{thm:first-order-decidable}.


\begin{proof}[Proof of Theorem~\ref{thm:first-order-decidable}]
    When defining the category of single-use functions, we viewed it as a restriction of a larger category, in the morphisms were equivariant functions.  We begin by describing the generalization of this larger category to the case where the parameter $\atoms$ is some relational structure. The idea is to use first-order definable functions as the general analogue of equivariant functions. 

\begin{definition}[Category of first-order definable functions]
    Let $\atoms$ be a relational structure. The \emph{category of first-order definable functions over $\atoms$} is defined as follows.
    \begin{itemize}
        \item The objects are polynomial sets over $\atoms$, i.e.~finite disjoint unions of powers of $\atoms$.
        \item The morphisms are the  first-order definable functions. Here, a function 
        \begin{align*}
            f : \atoms^{n_1} + \cdots + \atoms^{n_k} \to \atoms^{m_1} + \cdots + \atoms^{m_\ell}
            \end{align*}
            is called  \emph{first-order definable} if for every $i \in \set{1,\ldots,k}$ and  $j \in \set{1,\ldots,\ell}$ the set 
            \begin{align*}
            \setbuild{ (x,y) \in \atoms^{n_i} \times \atoms^{m_j}}{f(\text{$x$ in the $i$-th component}) = \text{$y$ in the $j$-th component}}
            \end{align*}
            can be defined by a first-order formula over the vocabulary of $\atoms$.
    \end{itemize}
\end{definition}
The above is easily seen to be a category, since first-order definable functions  can be composed. 
If the underlying structure $\atoms$ has a decidable first-order theory, then the category described above is reasonably tame, in particular one can decide if two morphisms are equal.

The single-use category admits a faithful functor to the above category. In particular, this implies that the problem of deciding if two morphisms are equal is decidable in the single-use category. 
\end{proof}
