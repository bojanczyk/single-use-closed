\newcommand{\invar}[1]{#1_{\mathrm{in}}}
\newcommand{\outvar}[1]{#1_{\mathrm{out}}}

\section{Game semantics}
\label{sec:game-semantics}

This section is devoted to the proof of Theorem~\ref{thm:single-use-closed}. To construct the function space $X \Rightarrow Y$, we use game semantics to identify a certain normal form of programs that compute single-use functions. The presentation in this section is self-contained, and does not assume any knowledge of game semantics. We base our notation on~\cite{abramsky2013semantics}.

Let us begin with a brief motivation for why game semantics will be useful.

While it is intuitively clear which functions should be allowed as single-use for simple types such as $\atoms \to 1+1$ or $\atoms \otimes \atoms \to \atoms + \atoms$, these intuitions start to falter when considering more complex types. How can one show that a function is \emph{not} single-use? If one were to use the definition of single-use functions alone, then one would need to rule out any possibility of constructing the function from the primes using the combinators, including constructions that use composition many times, and with unknown intermediate types. 

This is the reason why we consider game semantics. It will allow us to give  a more principled description of the intuition that pairs of type $X \otimes Y$ can be used on both coordinates, while pairs of type $X + Y$ can be used on a chosen coordinate only. The idea behind game semantics is to give the description in terms of an interaction between two players.  The two players are:
\begin{enumerate}
    \item System, who represents the function (we will identify with this player); and
    \item Environment, who supplies inputs and requests outputs of the function.
\end{enumerate}
One of the intuitions behind the setup is that if a type $X \& Y$ appears in the input of the function, then it is the System who can choose to use $X$ or $Y$, while if the type appears in the output, then it is the Environment who makes the choice. (In this paper, we consider functions of first-order types of the form $X \to Y$, where $X$ and $Y$ are linear types that do not use $\otimes$, and therefore there will be a clear distinction between input and output values.)
Before giving a formal definition of game semantics, we give simple example of the interaction.

\begin{example}\label{ex:amp-otimes-distr}
    Consider the two types 
    \begin{align*}
        X \otimes (Y \& Z) 
        \quad \text{and} \quad
        (X \otimes Y) \& (X \otimes Z).
    \end{align*}
    Among the prime functions in Table~\ref{fig:prime-morphisms-with-with}, we find distributivity in the direction $\rightarrow$, but not in the direction $\leftarrow$. We explain this asymmetry using the interaction between two players System and Enviroment.

    Let us first consider the interaction in the  direction $\rightarrow$. Player Environment begins be requesting an output. Since this output is of type $(X \otimes Y) \& (X \otimes Z)$, this means that Environment can choose to request either of the two types  $X \otimes Y$ and $X \otimes Z$. Suppose that Environment requests $X \otimes Y$. Now player System needs to react, and produce two elements: one of type $X$ and one of type $Y$. Both can be obtained from the input; for the second one player System can choose how to resolve the input $Y \& Z$ to get the appropriate value. 

    Consider now the interaction in the opposite direction $\leftarrow$. As we will see, player System will be unable to react to the behavior of player Environment, which will demonstrate that there is no distributivity in this direction. The problem is that player Environment can begin by requesting an element of type $X$, since the output type is $X \otimes (Y \& Z)$, while still reserving the possibility to request $Y \& Z$ in the future (because the tensor product $\otimes$ means that both output values need to be produced). To produce this element, player System will need  choose one of the two coordinates in the input type $(X \otimes Y) \& (X \otimes Z)$, and any of these two  choices will be premature, since player Environment can then request the opposite choice in the output type.  \exampleend
\end{example}

As illustrated in the above example, we will use a game to describe the possible behaviours of a function, as modelled by behaviours of player System. The game will be played in an arena, which will arise from the type of the function, and will tell us what are the possibilities for the moves of both players.  Here is the  outline the plan for the rest of this section. 
\begin{enumerate}
    \item For every two linear types $X$ and $Y$, we  define an arena, which is a data structure that describes all possible interactions between players Environment and System that can arise when running a  single-use  functions of type $X \to Y$;
    \item In each arena, we will be interested in the strategies of player System, i.e.~the ways in which System reacts to moves of Environment. We will show how such strategies can be interpreted as single-use functions: to a strategy we will assign a single-use function of type $X \to Y$ that is represented by this strategy. This mapping will be partial, i.e.~some ill-behaved strategies will not represent any functions. 
    \item We will show that the  set of strategies in an arena is  large enough to represent all single-use functions, but small enough to be orbit-finite. 
    \item We will then strengthen this: not only is the set of strategies orbit-finite, but it can be equipped with the structure of a linear type, such that both evaluation and currying will be single-use functions.
\end{enumerate}

The arenas from the first step of the plan will  be defined in two sub-steps. We begin by defining arenas and strategies for functions that do not use the structure of the atoms, i.e. constants and equality tests. This will be a fairly generic definition, almost identical to the classical game semantics for linear logic. Then we will extend the definition to cover the extra structure. 

\subsection{Arenas and strategies without constants and equality tests}
\label{sec:arenas-without-constants-and-equality-tests}

We begin with a simpler version of the game semantics, in which the arenas and strategies will describe functions that are not allowed to perform equality tests, and are not allowed to use constants. These strategies will model functions such as the identity function $\atoms \to \atoms$, which directly passes its input to its output, but they  will not model the equality test $\atoms \otimes \atoms \to 1 + 1$, or the constant functions of type $1 \to \atoms$. The general idea is to use standard game semantics for linear logic, with an extra feature that we call \emph{register operations}. The register operations will be used to model the way in which atoms are passed from the input to output. For example, in the identity function,  Environment will  write the input atom into the register, and then player System will read the output atom from that register. The following definition of an arena is based on the definition from \cite[p.4]{abramsky2013semantics}, slightly
adapted for the context of this paper:
\begin{definition}[Arena] \label{def:arena}
    An \emph{arena} consists of:
    \begin{enumerate}
        \item A set of \emph{moves}, with each move having an assigned owner, who is either ``System'' or ``Environment'', and one of three\footnote{\label{footnote:read-write} In all arenas that we consider, the ``read'' moves will be owned by System and the ``write'' moves will be owned by Environment. Therefore, we could simplify the register operations and have just one, called ``read/write'', whose status is determined by its owner. 
        } register operations, which are ``none'', ``read'', or ``write''.
                \item A set of plays, which a set of finite sequences of moves that is closed under prefixes, and such that in every play, the owner of the first move is Environment, and then the owners alternate between the two players.
    \end{enumerate}
\end{definition}




An arena will correspond to a type. The inhabitants of that type, which will be  functions if the type is a functional type $X \to Y$, will be described by strategies in the arena. Such a strategy tells us how player System should react to the moves of player Environment. Intuitively speaking, in the case of a functional type, a strategy will say how the function reacts to requests in the output type and values in the input type. We will only be talking about strategies for player System, so from now on, all strategies will be for player System. The following definition corresponds to the definition from \cite[p.5]{abramsky2013semantics}:
%unless otherwise stated.
% In the above definition, the operations are a non-standard part of the arenas.  The idea is that the operations modify a memory store which has exactly one register, which contain a single atom or be empty.  A move that has an associated operation that is not ``none'' will be called an \emph{atom move}.


\begin{definition}[Strategy]
    A strategy  in an arena  is a subset of plays in the arena, which: 
    \begin{enumerate}
        \item is closed under prefixes;
        \item\label{item:sys-ext} if the strategy contains a play $p$ that ends with a move owned by player System, then it also contains all possible plays  in the arena that extend $p$ with one move of player Environment;
        \item\label{item:env-ext} if the strategy contains a play $p$ that ends with a move owned by player Environment, then it contains exactly one play  in the arena that extends $p$ with one move of player System;
         \item there is some $k$ such that all plays in the strategy have length at most $k$;
        \item every ``read'' move is directly preceded by a ``write'' move (in particular a play cannot begin with ``read''), and every ``write'' move is either the last move, or directly succeeded by a ``read'' move.
    \end{enumerate}
\end{definition}

Conditions \ref{item:sys-ext} and \ref{item:env-ext}, which are standard in game semantics,  guarantee that the strategy only ``ends'' when 
Environment has no moves to play.  Let us now comment on the last two conditions, which are not standard.

The fourth condition is motivated by the idea that we study ``finite'' types,  and there will be no need for unbounded computations. 

The last condition will be called the  \emph{immediate read condition}. It ensures that there is matching between ``read'' and ``write'' moves in plays that do not end with write. Since ``write'' will always be owned by Environment, the immediate read condition will ensure a matching between ``write'' and ``read'' moves.
%in the plays that do not end with a move by player Environment, and such plays will be the most important ones. 

We now show how to associate to each linear type a corresponding arena, and also how to associate an arena to a function type $X \to Y$. This definition will be compositional, i.e.~it will arise through operations on arenas that correspond to the type constructors such as $1$ or $\otimes$. The arenas that we will construct so far will not be our final proposal, since the corresponding strategies will not be able to use constants or perform equality tests. This will be fixed in Section~\ref{sec:arenas-with-constants-and-equality-tests}, where a more complex arena will be defined for function types. Before giving formal definition, we begin with a simple example of the arena for the type $\atoms \to \atoms$.
\begin{example}\label{ex:identity-function-without-equality-tests-and-constants}
    We define an arena for the type $\atoms \to \atoms$. This arena will be rather impoverished, since the  only allowed strategy in it will correspond to the identity function. However, this is consistent with the stage that we are at, where we only consider functions that do not use constants or perform equality tests; for such functions the only possibility is the identity function.

     The arena will describe the following interaction between the two players: Environment  requests an output,  then System requests an input, then Environment grants the input, and finally player System grants the output by forwarding the input grant. 
    The arena is shown in the following picture: 
    \mypic{6}
    The methodology of drawing this picture will become clearer later on, as we define operations on arenas such as $\otimes$. For the moment we describe the arena without caring that it arises as a special case of some general construction. The arena has four moves: 
    \begin{center}
        \begin{tabular}{lll}
         move & owner & register operation \\
            \hline
            request output & Environment & none \\
            request input & System & none \\
            grant input & Environment & write \\
            grant output & System & read 
        \end{tabular}    
    \end{center}
    The set of plays is defined as follows. These are all sequences that begin with a move of player Environment, alternate between players, use each move only once, and have the following condition:  ``grant output'' can only be played after ``request output'', and likewise for ``grant input'' and ``request input''.  

    A quick inspection of the above definition reveals that the arena has a unique maximal play, where the moves are played in the order from the table, and all other plays are prefixes of this maximal play. Because of the uniqueness of responses, the set of plays is also a strategy. As mentioned at the beginning of this example, the strategy describes the identity function.  \exampleend
\end{example}

We hope that the above example explains some intuitions about how arenas describe types and strategies describe functions. We now give a formal definition. As mentioned before, this definition is compositional: we define arenas for the basic types $1$ and $\atoms$, and then we define operations on arenas that correspond to the type constructors  $+$, $\&$, and $\otimes$. We begin with the basic types.


\begin{definition}[Arenas for $1$ and $\atoms$]\label{def:arenas-without-atoms-or-functions} \ 
    \begin{enumerate}
        \item     The arena for the type $1$ is empty: there are no moves and the only play is the empty sequence. 
        \item The arena for type $\atoms$ has two moves, which must be played one after the other: first player Environment makes a move called ``request'' that has no register operation, and then player System responds with  a move called ``grant'' that has register operation ``read''.
    \end{enumerate}
\end{definition}

In the above definition, we only described the behaviour of $\atoms$ when viewed as an output type. To get the input type, where the players are swapped and read is swapped with write, we will use duality, which is another operation on arenas. This operation, together with other operations that correspond to the type constructors, are defined below.   

\begin{definition}[Operations on arenas]\label{def:composition-of-arenas}
    Let  $A$ and $B$ be arenas. We define the following arenas (see also Figure~\ref{fig:arena-constructors}):
        \begin{description}
            \item[$A+B$] The moves in this arena are the disjoint union of the moves of $A$ and $B$, with inherited owners and register operations, plus three extra moves: ``ask'' owned by Environment, and ``left'', ``right'' owned by System. 
            The plays are defined as follows. Player Environment begins with  an ask move, then System responds with a left or right move, and the remaining sequence is a play in the arena $A$ or $B$, depending on whether System chose left or right. 
            \item[$A \& B$] The moves in this arena are the disjoint union of the moves of $A$ and $B$, with inherited owners and register operations, plus three extra moves: ``acknowledge'' owned by System, and ``left'', ``right'' owned by Environment. 
            The plays are defined as follows. Player Environment begins by choosing left or right, then player System responds with an acknowledge move, and the remaining sequence is a play  in the arena $A$ or $B$, depending on whether Environment chose left or right.
            (This construction differs slightly from the one from \cite[Excercise~1.10]{abramsky2013semantics} -- this is because we want to keep it analogous to the construction for $A + B$.)
            \item[$A \otimes B$] The moves in this arena are the disjoint union of the moves of $A$ and $B$, with inherited owners and register operations. A play in this arena is any shuffle of plays in the two arenas $A$ and $B$. (A shuffle of two words is any word obtained by interleaving them, e.g.~shuffles of ``abc'' and ``123'' include ``a1b23c'' and ``12ab3c'').
            By Definition~\ref{def:arenas-without-atoms-or-functions}, we require that the owners of the move alternate in the 
            interleaved sequences. (This construction is based on \cite[p.7]{abramsky2013semantics}.) 
            

            \item[$\bar A$] This is called the dual arena of $A$. The moves and plays are the same as in $A$, except the owners are swapped, and the ``read'' and ``write'' register operations are swapped.
        \end{description} 
\end{definition}

\begin{figure}
The arena $A+ B$:
            \mypic{7}
The arena $A \& B$: 
            \mypic{8}
The arena $A \otimes B$:
            \mypic{9}
    \caption{\label{fig:arena-constructors} Pictures of the operations on arenas. The picture for $\otimes$  is less useful than previous two, since the root node of the tree is not a player, but a node labelled by $\otimes$. The intuition is that the game is played in parallel on both arenas, and therefore a position in it can be visualized as a pair of positions in the two arenas.}
\end{figure}


        
Equipped with the above definitions, we  present our first attempt at assigning arenas to types. In the second item of the following definition, we use the name \emph{library-less}, because our final definition of the arena for a function type, as presented in the next section, will be equipped with an extra feature that will be called a library. 

\begin{definition} Let $X$ and $Y$ be linear types.
    \begin{itemize}
        \item The \emph{arena for  $X$} is defined by inductively applying the constructions from Definition~\ref{def:arenas-without-atoms-or-functions} and Definition~\ref{def:composition-of-arenas} according to the structure of the type.
        \item The \emph{library-less arena for $X \to Y$} is defined to be (dual of arena of $X$) $\otimes$ (arena of $Y$).
    \end{itemize}
\end{definition}

As discussed previously, our notion of arenas does not yet take into account the structure of the atoms, i.e.~the constants and equality tests. This will be fixed in the next section, by modifying the second item in the above definition. On the other hand, the arenas from that first item in the above definition, for linear types without function types, are already in their final form. 

In principle the construction from the second item in the above definition can be nested, and thus used to assign arenas to higher order types that can nest $\to$ with the other type constructors. This is how it is usually done in linear logic. However,  the construction that we will describe in the next section will  be less amenable to nesting, and will use it only  to describe functions between types that do not use $\to$.


 

\begin{example}
    The arena for the identity type $\atoms \to \atoms$ is the same as the arena from Example~\ref{ex:identity-function-without-equality-tests-and-constants}.
\end{example}





\subsection{Arenas and strategies with constants and equality tests}
\label{sec:arenas-with-constants-and-equality-tests}
In the Section~\ref{sec:arenas-without-constants-and-equality-tests}, we described arenas for functions that did not use the structure of the atoms, i.e.~constants and equality tests. We now show how these arenas can be extended to cover this structure. The general idea is to equip the arenas with an extra part, which we call the \emph{library},  that describes the allowed operations on the atoms. (The library as we present it here only contains functions for equality and constants, but in the proof of Theorem~\ref{thm:single-use-automata-relational-structures}, we can use a library that has other relations beyond equality. )



\begin{definition}[The library arena] \label{def:library-arena} The library arena and its parts are defined as follows, see Figure~\ref{fig:library-arenas} for pictures.
    \begin{enumerate}
        \item The \emph{constant choice arena} is the following arena $\atoms+1$ moves:
        first player System chooses an atom, then player Environment plays move with register operation ``write''. 
        \item The \emph{equality test arena} is an arena which the plays are as follows:
    \begin{enumerate}
        \item first player System plays a move with register operation ``read'';
        \item then player Environment plays an move with no register operation;
        \item then player System plays a move with register operation ``read'';
        \item then player Environment plays one of two moves, called $=$ and $\neq$, with no register operation.
    \end{enumerate}
    \item The \emph{library arena} is defined to be an arena that is obtained by applying $\otimes$ to infinitely many copies of the constant choice arena and infinitely many copies of the  equality test arena.
    \end{enumerate}
\end{definition}

\begin{figure}
The constant choice arena:
        \mypic{10}
The equality test arena:
    \mypic{11}
\caption{\label{fig:library-arenas} Pictures of the library arenas.  We use the convention that the register operations are in red, and the names of the moves, which have no other role than to distinguish them, are in black. 
Note that the first move in this arena is owned by System, and we assume in Definition~\ref{def:arena} that the first move is owned by Environment. This is because this arena, like all arenas in Definition~\ref{def:library-arena}, is not intended to be a stand-alone arena, but only as part of the bigger arena from Definition~\ref{def:arena-for-function-type} where the first move is indeed owned by Environment. } 
\end{figure}
       
The library arena is infinite. Taking the tensor product of infinitely many copies of the two arenas ensures that the library arena satisfies the following property, which corresponds to the $!$ operation from linear logic: 
\begin{align}\label{eq:bang-library-arena}
\text{library arena} 
\quad \equiv \quad 
\text{(constant choice arena)} \otimes 
\text{(equality test arena)} \otimes
 \text{(library arena)}.
\end{align}
In the above, $\equiv$ refers to isomorphism of arenas, which is defined in the natural way: this is a bijection between the moves, which is consistent with the owners, register operations and plays in the expected way.  Another property is that the library arena is isomorphic to a tensor product of itself: 
\begin{align}\label{eq:library-arena-isomorphism}
\text{library arena}
\quad \equiv \quad
\text{library arena} \otimes \text{library arena}.
\end{align}


We are now ready to give the final definition of arenas for functions between linear types, which takes into account the structure of the atoms.

\begin{definition}[Arena for a function type]\label{def:arena-for-function-type} For linear types $X$ and $Y$, the arena of $X \to Y$ is 
    \begin{align*}
    \text{(library arena)} \otimes \text{(dual arena of $X$)} \otimes \text{(arena of $Y$)}.
    \end{align*}
\end{definition}

This completes the game semantics of linear types and functions between them. We do not intend to give game semantics for higher order types, such as functions on functions etc. As a result, we will only be using the dual once, namely for the arena of the input type. Also, note that the read/write operations will be used in a restricted way, as announced in Footnote~\ref{footnote:read-write}, namely that the ``read'' moves will be owned by System and the ``write'' moves will be owned by Environment.  This is because the library arena has this property, the arena for $\atoms$ has this property, and all operations on arenas that we have defined preserve this property.

\subsubsection{Composition of strategies}
\label{sec:composition-of-strategies}
One of the main points about strategies in game semantics is that they can be composed.
The usual construction for the function type $X \to Y$ is to take the tensor product of the dual arena for $X$ and the arena for $Y$. 
Our construction is a bit more involved, since the arena that we use has a copy of the library arena. We now explain how to compose strategies in a way that accounts for the library arena. 

We begin by describing the usual construction for composing strategies, which we call \emph{shuffling and hiding}, see \cite[p.12]{abramsky2013semantics}, with a minor adaptation to our setting that has register operations. 

\begin{definition}[Shuffling and hiding]\label{def:shuffling-and-hiding}
    Let $A, B, C$ be arenas, and consider two strategies
    \begin{align*}
    \sigma_1 \text{ in the arena $A \otimes \bar B$}
    \qquad
    \sigma_2 \text{ in the arena $B \otimes C$}.
    \end{align*}
    The shuffling and hiding strategy for  $\sigma_1$ and $\sigma_2$, which is a strategy  in the  arena $A \otimes C$, is defined to be the set of plays    $p$  such that there exist plays $p_1 \in \sigma_1$ and  $p_2 \in \sigma_2$  with the following properties: 
\begin{itemize}
    \item the play $p$ satisfies the immediate read condition;
    \item the following two sequences of moves are equal: 
    \begin{enumerate}
        \item the subsequence of moves in $p_1$ that are from the arena $\bar B$;
        \item the subsequence of moves in $p_2$ that are from the arena $B$.
    \end{enumerate}
\end{itemize}
\end{definition}

Note that the set of moves in the arenas $\bar B$ is the same as the set of moves in the arena $B$ (the owners and register operations are changed),  and therefore the subsequences in items 1 and 2 above can be meaningfully compared.  The following lemma, whose straightforward proof is left to the reader,  shows that the above definition is well-formed.


\begin{lemma}
    The strategy described in Definition~\ref{def:shuffling-and-hiding} is a valid strategy.
\end{lemma}

% To show that this construction 
% preserves the immediate read condition, let us consider two consecutive moves 
% $m_e, m_s$ in a sequence from $\sigma;\tau$,
% such that $m_e$ is a write move by the environment, 
% and $m_s$ is the system's response, and let us show 
% that $m_s$ is a read move. Since moves $m_e$ and $m_s$ are consecuive in 
% $\sigma ; \tau$, it follows that in some sequence in $\sigma || \tau$, 
% they are separated by some moves from $B$:
% \[ \ldots m_e b_1 b_2 \ldots b_n m_s \ldots \quad \in \quad \sigma || \tau \]
% Assume (w.l.o.g.) that $m_e$ belongs to $\sigma$.
% Then, from definition of $\sigma || \tau$,
% we know that $m_e$, $b_1$ is a consecutive pair of moves in (some sequence from) $\sigma$. 
% It follows that $b_1$ is a read move in $\bar{A} \otimes B$,
% which means that $b_1$ is a write move in $\bar{B} \otimes C$. 
% By definition of $\sigma || \tau$, we know that $b_1, b_2$
% are consecutive moves in $\tau$, which means that $b_2$ is 
% a read-move in $\bar{B} \otimes C$, which in turn means that 
% it is a write-move in $\bar{A} \otimes B$. By repeating this reasoning, 
% we obtain that $b_n$ is a wite move in either $\bar{A} \otimes B$ or
% $\bar{A} \otimes B$, which means that $m_s$ is a write move from that arena.
% We can use the same reasoning to show that if $m_s$ is a read move, 
% then $m_e$ has to be a write move.  


We can now use shuffling and hiding to compose strategies in the arena for a function type that were defined in Definition~\ref{def:arena-for-function-type}.
Consider three linear types $X, Y, Z$, and two strategies, in the arenas for $X \to Y$ and $Y \to Z$, respectively.
 By unfolding the definition of arenas from Definition~\ref{def:arena-for-function-type}, these are strategies in the arenas 
\begin{align*}
\text{(library arena)} \otimes \overline{\text{arena for $X$}} \otimes \text{arena for $Y$} & \qquad \text{the arena for $X \to Y$, and}
\\
\text{(library arena)} \otimes \overline{\text{arena for $Y$}} \otimes \text{arena for $Z$} & \qquad \text{the arena for $Y \to Z$}.
\end{align*}
Using the construction from Definition~\ref{def:shuffling-and-hiding}, we can combine them into a single  strategy in 
\begin{align*}
\text{(library arena)} \otimes  \text{(library arena)}\otimes \overline{\text{arena for $X$}} \otimes \text{arena for $Z$}.
\end{align*}
Since the library arena is isomorphic to a tensor product of two copies of itself, the above strategy gives us a strategy in the arena 
\begin{align*}
    \text{(library arena)} \otimes   \overline{\text{arena for $X$}} \otimes \text{arena for $Z$} & \qquad \text{the arena for $X \to Z$.}
    \end{align*}

This strategy is defined to be the \emph{composition} of the original two  strategies. We write $\sigma; \tau$ for this composition. Thanks to \cite[Proposition~1.2]{abramsky2013semantics}, we know that the the usual
composition of library-free strategies is associative. It follows that our composition of library strategies is associative 
up to isomoprhism. 


\subsection{Strategies as single-use functions}
In this section, we explain how a strategy in a function type $X \to Y$ can be interpreted as a single-use function from $\sem X$ to $\sem Y$. This interpretation is partial, since some strategies will be considered inconsistent, and will not represent any function. We begin with an example which presents the intuition of why a strategy might be inconsistent. 

\begin{example}[Inconsistent strategy] \label{ex:inconsistent-strategy} Consider a strategy in the arena for  $1 \to \atoms \otimes \atoms$. We would like to think of this strategy of producing a pair of atoms, possibly using the constants from the library. Indeed, if player Environment requests an output, then player System will ultimately need to produce a corresponding read move, which will necessarily need to be fed by a write move that is in the constant choice arena (the appropriate formalizations of these notions will be given in Section~\ref{sec:some-notation-for-describing-plays}). However, an inconsistent strategy for player System could use different constants depending on the order in which Environment requests the two output atoms in $\atoms \otimes \atoms$.  For example, the output could be (John, Mark) for one order, and (Eve, Mary) for some other order.  \exampleend
\end{example}

Below, we will formalize the intuition from the above example. We begin by developing some notation for describing plays.





\subsubsection{Some notation for describing plays}
\label{sec:some-notation-for-describing-plays}
Consider a move in the arena associated to a function type $X \to Y$. There are three kinds of moves here: moves from the library, moves from $X$, and moves from $Y$. Depending on where the moves come from, we will associate to this move an \emph{origin} as follows. 
\begin{itemize}
    \item In the case of moves which come from the library, define the  {origin} to be the copy of the constant choice arena or the equality test arena from which the move is taken.
    \item   In the case of moves which come from either $X$ or $Y$, define the origin to be the corresponding node in the syntax tree of $X$ or $Y$. Here, the syntax tree of a type is defined in the obvious way, see Figure~\ref{fig:syntax-tree}.
\end{itemize}


\begin{figure}
    \mypic{12}
    \caption{\label{fig:syntax-tree} The syntax tree of the type $((\atoms +1) + 1) \otimes ((\atoms + \atoms) \& \atoms)$.}
\end{figure}


Consider a play in the arena for $X \to Y$. Define a \emph{library call} to be a subsequence of the play that is obtained by choosing some copy of the constant choice arena or the equality test arena, and then returning all moves in the play that originate from this copy. A library call will have at most two moves if it originates from some copy of the constant choice arena (we use the name \emph{constant call} for such library calls), and at most four moves if it originates from some copy of the equality test arena (we use the name \emph{equality call} for such library calls). For each constant call, there is an associated at atom, which is the first move in the call. 

Consider a read move in the play. By the immediate read condition, this move is directly preceded by a write move. The write move will either originate from a leaf labelled by $\atoms$ in  the syntax tree of the input type $X$, or it will originate from some constant call. In the latter case, it will have a corresponding atom. Therefore, to each read move in a play we can associate either some leaf labelled by $\atoms$ in the syntax tree of the input type, or some individual atom constant. The associated object will be called the \emph{read value} of the read move.



\subsubsection{Strategies in $1 \to X$ as values in $\sem X$}
\label{sec:strategies-in-1-to-x-as-elements-of-sem-x}
Equipped with the above notation, we are ready to define which strategies will be considered consistent, and how they can  be interpreted as values or functions.  We begin by describing strategies which represents values in the underlying set $\sem X$ of a linear type $X$, and then we lift this interpretation to functions.  To represent such values, we will use strategies not in the arena for $X$, but in the arena for $1 \to X$. This is because the second arena contains the library, which is needed to  produce  atom constants.



We begin by using the notation from Section~\ref{sec:some-notation-for-describing-plays} to define a notion of consistency that disallows the behaviour from Example~\ref{ex:inconsistent-strategy}.     Let $X$ be a linear type, and consider a strategy $\sigma$ in the arena for $1 \to X$. We say that a node of the syntax tree of $X$ is used by the strategy if at least one play in the strategy contains a move that originates from this node. It is not hard to see that if the strategy uses  a node of the syntax tree that is  labelled by $\otimes$ or $\&$, then it must also use both of its children. This is because player System must be able to respond to moves of player Environment that correspond to both children. On the other hand, if a node is labelled by $+$, then System should consistently choose only one of the child, namely the one that corresponds to the output value. This requirement, together with a similar one for the leaves of the syntax tree that are labelled by $\atoms$, is presented in the following definition.



\begin{definition}[Value consistent strategy]\label{def:consistency}
    Let $X$ be a linear type. A strategy in  the arena for $1 \to X$   is  called \emph{value consistent} if: 
    \begin{enumerate}
        \item Let $x$ be a node in  the syntax tree of $X$ that is labelled by co-product $+$. Then at most one child of $x$ is  used by the strategy. 
        \item Let $x$ be a node in the syntax tree of $X$ that  is labelled by $\atoms$. Then there is some atom $a$ such that for every play in the strategy, if the play contains a read move that originates from $x$, then the read value of this move is $a$. 
    \end{enumerate}
\end{definition}

To a consistent strategy in the arena for $1 \to X$ we can now assign a value in $\sem X$, which will be called \emph{the value represented by the strategy}. This is done as follows.  For every node $x$ in the syntax tree (whose subtree represents some linear type) that is used by the strategy,  we define a value in the corresponding type. This value is defined by induction on the size of type. Consider a node that is used by the strategy. If the node is a leaf labelled by $1$, then the value is the unique element of the underlying set of the type $1$. If the node is a leaf labelled by $\atoms$, then the value is the atom from the consistency requirement. If the node is labelled by co-product, then the value is inductively defined by using the child from the consistency requirement. Finally, if the node is labelled by $\otimes$ or $\&$, then both of its children must also be used by the strategy, and therefore, we can define the value by pairing the two values from the induction assumption. 


\subsubsection{Strategies in $X \to Y$ as single-use functions}
We now move to the main point of this section, which is to show how to interpret a strategy in the arena for $X \to Y$ as a single-use function of type $X \to Y$. This is done by using the definition from for values in the previous section, and then lifting it to functions by using composition of strategies from Section~\ref{sec:composition-of-strategies}.


\begin{definition}\label{def:strategy-as-single-use-function}
    Let $X$ and $Y$ be linear types, and let $\sigma$ be a strategy in the arena $X \to Y$. We say that $\sigma$ \emph{represents} a function  
    \begin{align*}
    f : \sem{X} \to \sem{Y}
    \end{align*}
    such that for every value $x \in \sem X$, if  $x$ is represented by some value consistent strategy $\tau$ in the arena for $1 \to X$, then the composition strategy $\tau;\sigma$ in the arena for $1 \to Y$ is value consistent and represents the value $\sem \sigma (x)$. This is illustrated in the following diagram:
    \[
\begin{tikzcd}
    [column sep=4cm]
\text{value consistent strategies in the arena for $1 \to X$}
 \arrow[r, "\text{represented value}"] \arrow[d, "\tau \mapsto \sigma;\tau"' ]  &
 \sem X 
 \ar[d,"f"]\\
\text{value consistent strategies in the arena for $1 \to Y$"} 
\ar[r,"\text{represented value}"'] & 
\sem Y
\end{tikzcd}
\]
\end{definition}

The function represented by a strategy, if it exists, is unique, and will be called \emph{the function represented by the strategy} and denoted by $\sem \sigma$. The following lemma shows that the composition of strategies is consistent with the composition of functions.

\begin{lemma}    Let $X, Y, Z$ be linear types.  Consider two strategies as follows: 
    \[
    \begin{tikzcd}
    X \ar[r, "\sigma_1"] & Y \ar[r, "\sigma_2"] & Z.
    \end{tikzcd}
    \]
If $\sem {\sigma_1}$ and $\sem {\sigma_2}$ are  both defined, then $\sem{\sigma_1;\sigma_2}$ is also defined and equal to $\sem {\sigma_1} ; \sem {\sigma_2}$.
\end{lemma}
\begin{proof}
    Let $\tau$ be some value consistent strategy in the arena for $1 \to X$. By associativity of strategy composition, we know that the strategies 
    \begin{align*}
    \tau;(\sigma_1;\sigma_2) \quad \text{and} \quad (\tau;\sigma_1);\sigma_2
    \end{align*}
    are equal, up to an isomorphism of the library arena. Since such an isomorphism does not affect value consistency or the represented value, either both or none are value consistent, and if they are value consistent, then they represent the same value. The right strategy is value consistent by the assumption of the lemma, and therefore the same is true for the left strategy, and the represented values are equal.
\end{proof}




\begin{lemma}
    Let $X$ and $Y$ be linear types, then every single-use function of type $X \to Y$ is represented by at least one strategy. 
    % \begin{itemize}
    %     \item \emph{Soundness.} Every function represented by a strategy in the arena for $X \to Y$ is single-use.
    %     \item \emph{Completeness.} Every single-use function of type $X \to Y$ is represented by at least one strategy.
    % \end{itemize}
\end{lemma}
\begin{proof}
    It suffices to show that all single-use prime functions can be represented as strategies,
    and show how to raise combinators $\circ$, $\otimes$, $\times$, $\&$ from functions to 
    strategies in a way that preserves the semantics. Let us start with the combinators:
    \begin{description}
        \item[$f \circ g$] Here we use the construction for strategy composition from Section~\ref{sec:composition-of-strategies}.
        This means that we need to show that for all strategies $\sigma : X \to Y$ and $\tau: X \to Z$, 
        and for all functions $f : \sem{X} \to \sem{Y}$ and $g: \sem{Y} \to \sem{Z}$,
        if $\sem{\sigma} = f$ and $\sem{\tau} =g$, then $\sem{\sigma; \tau} = g \circ f$. By \ref{def:strategy-as-single-use-function}, 
        this means that we need to show that for all strategies $\nu : 1 \to X$, it holds that: 
        \[ \mathtt{val}(\nu; (\sigma; \tau)) = g(f(\mathtt{val}(\nu))) \]
        First, let us notice that (as explained in Section~\ref{sec:composition-of-strategies}) 
        the strategy $\nu; (\sigma; \tau)$ is equal to $(\nu; \sigma); \tau$ up to arena isomorphism. 
        Since area isomoprhisms preserve $\mathtt{val}$ (as they only rename library funcions), 
        we obtain that:
        \[ \mathtt{val}(\nu; (\sigma; \tau)) = \mathtt{val}((\nu; \sigma;) \tau) = g(\mathtt{val}(\nu; \sigma)) = g(f(\mathtt{val}(\nu))) \]

        \item[$f + g$] First, let us show how to construct a strategy $\sigma_1 + \sigma_2 : (X_1 + X_2) \to (Y_1 + Y_2)$ 
        from strategies $\sigma_1 : X_1 \to Y_1$ and $\sigma_2 : X_2 \to Y_2$. The strategies work as follows (remember that 
        now we play as system): The first move of environment has to be ``ask' (in $Y_1 + Y_2$), 
        we reply with ``ask'' in $X_1 + X_2$, then environment can only reply with either ``left'' or ``right''
        in $X_1 + X_2$ -- if it replies ``left'', we reply with ``left'' in $Y_1 + Y_2$ and contiune playing according to $\sigma_1$, 
        else we reply ``right'' and continue playing according to $\sigma_2$.
        For $\sigma_1 + \sigma_2$ defined in this way, it is not hard to see that if $\sem{\sigma_1} = f_1$ and $\sem{\sigma_2} = f_2$, 
        then $\sem{\sigma_1 + \sigma_2} = f + g$. This because by definition of strategy compsition, $\mathtt{val}$ and $+$, 
        we know that $\nu; (\sigma_1 + \sigma_2)$ is either equal to (a) $\nu'; \sigma_1$ for some $\nu'$ 
        such that $\mathtt{val}(\nu) = \mathtt{left}(\mathtt{val(\nu_1)})$, or
        $\nu'; \sigma_2$ for some $\nu'$  such that $\mathtt{val}(\nu) = \mathtt{right}(\mathtt{val(\nu')})$. 

        \item[$f \& g$] Here the construction is similar to the one for $+$: We strat by showing 
        how to construct a strategy $\sigma_1 + \sigma_2 : (X_1 \& X_2) \to (Y_1 \& Y_2)$ 
        $\sigma_1 : X_1 \to Y_1$ and $\sigma_2 : X_2 \to Y_2$: The first move of environment has to be either 
        ``left'' or ``right'' in $Y_1 \& Y_2$. We respond with the same in $X_1 \& X_2$. The only move of 
        the environment is now to ``acknowledge'' in $Y_1 \& Y_2$, after which we ``acknowledge'' in $X_1 \& X_2$. 
        Then we play either according to $\sigma_1$ or $\sigma_2$ (depending on whether the first move of the environment
        was ``left'' or ``right''). To see that if $\sem{\sigma_1} = f_1$ and $\sem{\sigma_2} = f_2$, then 
        $\sem{\sigma_1 \& \sigma_2} = f_1 \& f_2$, we notice that by definition of composition and $\&$, 
        we know that after environment plays ``left'' in $\nu; (\sigma_1 \& \sigma_2)$ the game proceeds 
        to a state equivalent to $\nu_1 ; \sigma_1$, where $\pi_1(\mathtt{val}(\nu)) = \mathtt{val}(\nu_1)$. 
        (and analogously for ``right'').

        \item[$f \otimes g$] Again, let us start by showing how to constructed
        $\sigma_1 \times \sigma_2 : (X_1 + X_2) \to (Y_1 + Y_2)$ from $\sigma_1 : X_1 \to Y_1$ and $\sigma_2 : X_2 \to Y_2$.
        The first move is by environment. It can either play in $Y_1$ or in $Y_2$. Let us assume that
        it plays in $Y_1$ (as the other case is summetrical). Then, we play according to 
        $\sigma_1$. As $\sigma_1$ is bounded, this means that after some interaction with $X_1$ and the library\footnote{
            It might be worth clarifying that during this time, if we play in $X_1$ the environment needs to respond
            immediately in $X_1$ and if we play in the library, the environment needs to respond immediately in the library, 
            as it has no other available moves. 
        }
        we will respond in $Y_1$. Then the environment can choose again if it wants to play in $Y_1$ or $Y_2$
        if it plays in $Y_1$ we play according to $\sigma_1$, and if it plays in $Y_2$ we play according to $\sigma_2$. 
        We continue in this fashion until the environment has no available moves in $Y_1$ or $Y_2$.
        To see that if $\sem{\sigma_1} = f_1$ and $\sem{\sigma_2} = f_2$, then
        $\sem{\sigma_1 \otimes \sigma_2} = f_1 \otimes f_2$. Let us observe that during the 
        $Y_1$-part of computing $\mathtt{val}(\nu; (f_1 \otimes f_2))$ the game looks exactly the 
        same as the game for computing $\mathtt{val}(\nu_{Y_1}; f_1)$, 
        where $\nu_{Y_1}$ is define as the plays subset of plays from $\nu$, where 
        environment never plays in $Y_2$. This is a strategy in $1 \to Y_1$, because if 
        the environment never plays in $Y_1$ then system cannot respond in $Y_1$. 
        It is not hard to see that $\mathtt{val}(\nu_{Y_1}) = \pi_1(\mathtt{val}(\nu))$. 
        It follows that during the $Y_1$-part of the play for computing $\mathtt{val}(\nu; (\sigma_1 \otimes \sigma_2))$, 
        we obtain $f_1(\pi_1(\mathtt{val}(\nu)))$. Similarly, one can show that during the $Y_2$ part of computing
        $\mathtt{val}(\nu; (\sigma_1 \otimes \sigma_2))$, we obtain $f_2(\pi_2(\mathtt{val}(\nu)))$. It follows that: 
        \[ \mathtt{val}(\nu; (\sigma_1 \otimes \sigma_2)) \quad = \quad \bigl( f_1(\pi_1(\mathtt{val}(\nu))),\ f_2(\pi_2(\mathtt{val}(\nu))) \bigr) \quad = \quad (f_1 \otimes f_2) (\mathtt{val}(\nu))\]
    \end{description}
    Let us now deal with prime functions. For the sake of brevity we only show how to implement them 
    for the following representative set of examples (all of the prime functions not listed below, 
    can be implemneted without ever playing the library, and are well known tautoligies of the affine logic):
    \begin{description}
        \item[$\textrm{eq} : \atoms \otimes \atoms \to 1 + 1$]:
        For this funcion the system wants to call the library function for comparing atoms 
        and return the result. Here is the exact strategy:    
        First, the environment has to play ``ask'' in $1+1$. Then we play ``request'' in the left $\atoms$.
        Environment replies with ``grant'' which is a write move. Then we play the read 
        move in atoms equality library function. Environment acknowledges. We play ``request'' in the 
        right $\atoms$. Environment plays ``grant''. We play the read move in the atoms equality function. 
        The environment replies with either $=$ or $\neq$, to which we respond respectively with ``left''
        or ``right'' in $1+1$.

        \item[$\textrm{const}_a : 1 \to \atoms$] For this function we want to call the constant
        functionality of the library. Here is the exact strategy: First, the environment has to play 
        ``request'' in $\atoms$. We respond in the $a$-move in the constant library component. 
        The environment has to respond with the write move, we respond with ``grant''. 

        \item[$\textrm{proj}_1 : X \otimes Y \to X$] Here we play a restricted version of the copycat strategy:
        whenever the environment plays in the right copy of $X$, we play the same move in he left copy of $X$, 
        and whenever the environment responds in the left copy of $X$, we respond with the same move in the right 
        copy of $X$. 

        \item[$\textrm{distr}_{\&, \otimes} : X \otimes (Y \& Z) \to (X \otimes Y) \& (X \otimes Z)$]
        Here we play according to the strategy from \ref{ex:amp-otimes-distr}: The environment 
        starts by either playin ``left'' or ``right'' in $(X \otimes Y) \& (X \otimes Z)$, 
        we respond with the same move in $(Y \& Z)$. The environment has to play ``acknowledge'' 
        in $(Y \& Z)$, after which we play acknowledge. This leaves in a state that is either 
        equivalent to $X \otimes Y \to X \times Y$ or  $X \otimes Z \to X \times Z$. 
        In both cases we play according to the copycat strategy.  \rafal{Define copycat?}
    \end{description}
\end{proof}
One can also show that the system is sound (i.e. that every function repersented by a strategy is single-use). \rafal{reference 
to a claim, which follows from SMCC-ness. We don't defer the proof of completeness because the proof of SMCC-ness depends on it.}

\subsection{The set of strategies as a linear type}
The purpose of this section is to show that the set of strategies in an arena for a function type $X \to Y$ can be described using some linear type, and furthermore the relevant operations on strategies, such as application and currying, can be performed in a single-use way.

Before we do this we introudce the idea of a \emph{partially applied arena}.
Let $A$ be an arena, and $m$ a move. We define the partially applied arena ${m^{-1}}A$ 
to be set of sequences that follow $m$ in $A$, i.e:
\[m^{-1}A = \{s \ | \ ms \in A \}\]
The only difference between an arena and a partially applied arena, is that the first moves in $m^{-1}A$ will belong to the system (and not to environment).
For this reason, we keep track of the \emph{owner} of the arena, if environment owns $A$, then the system owns $m^{-1}A$ (and vice versa).
Observe, that in our case if moves $a, b$ belong to $\atoms$ (i.e. they are constant requests), then $a^{-1}A = b^{-1}A$. 
For this reason we define $\atoms^{-1}A$ to be the arena $A$ partially applied with any one of the atomic moves. 

We are now ready the type for storing $k$-bounded strategies. 
\begin{definition}
    For a (partially applied) arena $A$, we define the type $\Strat(A, k)$ of $k$-bounded strategies on $A$ using the following induction:
\begin{itemize}
    \item $k=0$. This type is $1$ if the Environment owns $A$ and has no moves to play. Otherwise, this type is $\emptyset$. 
    \item $k + 1$. This type is defined differently, depending on whether Environment or System owns $A$.
    \begin{itemize}
        \item Environment owns $A$. Let $m_1,\ldots,m_n$ be the set of possible first moves. (This set is finite.) The type is defined to be 
        \begin{align*}
            \Strat(m_1^{-1}A, k) \& \ldots \& \Strat(m_n^{-1}A, k).
        \end{align*}
        \item System owns $A$. Let $m_1,\ldots,m_n$ be the set of possible first moves that are not first moves in the library.  (This set is finite.) The type is defined to be 
        \begin{align*}
            \Strat(m_1^{-1}A, k) + \ldots + \Strat(m_n^{-1}A, k) + \atoms \otimes \Strat(\atoms^{-1}A, k) + \Strat({eq}^{-1}A, k).
        \end{align*}
    \end{itemize}
\end{itemize}
    \[ \Strat(A, 0 ) = \begin{cases}
            \varnothing & \textrm{if System owns $A$}\\
            \varnothing & \textrm{if Environment owns $A$ and still has moves to play}\\
            1 & \textrm{if Enviroment own $A$ and has no moves to play}
    \end{cases}\] 
    \[
      \Strat(A, k+1) = \begin{cases}
        \underset{\substack{
            \textrm{for every}\\
            \textrm{first move $m$}\\
        }}{\mathlarger{\mathlarger{\mathlarger{\mathlarger{\&}}}}} \Strat(m^{-1}A, k) & \textrm{if environment owns $A$} \\ 
        \left(\underset{\substack{
            \textrm{for every first}\\
            \textrm{move $m \not\in \atoms$}\\
        }}{\mathlarger{\mathlarger{\mathlarger{\mathlarger{+}}}}} \Strat(m^{-1}A, k)\right) + 
        \underbrace{\atoms \otimes \Strat(\atoms^{-1}A, k)}_{\textrm{if $a \in \atoms$ is a valid first move}} & 
        \textrm{if system owns A}
      \end{cases} 
    \]
    The intuition behind this is that if that the first move belongs to environment, then the strategy has to be prepared 
    for all of them, because it does not know what the system will do. However, the environment will play only one of its moves,
    so the strategy will only have to pick one of the future behaviours, which is the perfect use caase for $\&$. 
    If the arena belongs to System, then the strategy already knows which move it is going to play. This is represented 
    with $+$. The exact definitions of the big versions of $+$ and $\&$ are as follows:
    \[ 
    \begin{tabular}{cc}
        $\underset{ m \in M
        }{\mathlarger{\mathlarger{\mathlarger{\mathlarger{\&}}}}} S_m = \begin{cases}
            \varnothing & \textrm{if some } S_m = \varnothing\\
            1 & \textrm{if } M = \varnothing\\
            S_{m_1} \& \ldots \& S_{m_k}  & \textrm{otherwise}
        \end{cases}$ &
        $\underset{ m \in M
        }{\mathlarger{\mathlarger{\mathlarger{\mathlarger{+}}}}} S_m = \begin{cases}
            \varnothing & M = \varnothing\\
            1 & \textrm{if all } S_m = \varnothing\\
            S_{m_1} + \ldots + S_{m_k}  & \substack{\textrm{for the non-empty } m_i\textrm{'s}\\
                                                    \textrm{otherwise}}
        \end{cases}$


    \end{tabular}
    \]
\end{definition}







In the definition of the linear type, we will be only interested in certain strategies, which avoid behavior that is not relevant to the computation.



\begin{definition}[Irrelevant question] We say that a play in an arena for $X \to Y$ contains an \emph{irrelevant question} if it contains an equality test on two constants. We say that this strategy \emph{does not ask irrelevant questions} if none of its plays have irrelevant questions.
\end{definition}
  
The following lemma shows that if irrelevant questions are disallowed, then plays have length bounded by some constant that depends only on the types involved. This will be one of the useful properties of such strategies, since it will allow us to represent them using a finite linear type.

\begin{lemma}\label{lem:no-irrelevant-questions-are-bounded}
    For every linear types $X$ and $Y$ there is some $k$ such that if a strategy in the arena for $X \to Y$ does not ask irrelevant questions, then all plays in the strategy have length at most $k$.
\end{lemma}

There is a subtle point, which is illustrated in the following example:  strategies that do not ask irrelevant questions are not closed under composition. 

\begin{example}[irrelevant questions from composition]
    Consider the  following two functions: 
    \begin{enumerate}
        \item the function of type  $1 \to \atoms \otimes \atoms$ that outputs the pair (Mark, John) for its unique input;
        \item the equality test of type $\atoms \otimes \atoms \to 1 + 1$.
    \end{enumerate}
    Like all single-use functions, the above two functions can be represented using strategies that do not ask irrelevant questions. However, if we apply the composition construction from Section~\ref{sec:composition-of-strategies} to these two strategies, we will get a strategy that asks the irrelevant question of whether Mark and John are equal. \exampleend
\end{example}

In light of the above example, we will want to eliminate irrelevant questions. 

\begin{lemma}\label{lem:eliminate-irrelevant-questions}
    For every linear types $X$ and $Y$, every single-use function of type  $X \to Y$ is represented by  some strategy that does not ask irrelevant questions.
\end{lemma}


The following lemma shows that the set of $k$-bounded strategies can be described using a linear type.

\begin{lemma}\label{lem:linear-type-of-k-bounded-strategies}
    Let $X$ and $Y$ be linear types, and let $k \in \set{0,1,\ldots}$. One can find: 
    \begin{enumerate}
        \item \label{it:k-bounded-type} a linear type, call it $X \Rightarrow_k Y$;
        \item \label{it:k-bounded-bijection} a bijection of its underlying set with the set of $k$-bounded strategies in the arena for $X \to Y$;
        \item \label{it:k-bounded-eval} a single-use function of type $X \otimes (X \Rightarrow_k Y) \to Y$;\end{enumerate}
    such that for every $k$-bounded strategy $\sigma$, the following diagram commutes:
    \[
    \begin{tikzcd}[column sep=3.5cm]
    X  
    \ar[rr,"{x \mapsto x \otimes \text{(element corresponding to $\sigma$ via bijection in item~\ref{it:k-bounded-bijection}})}"]
    \ar[dr,"{\text{function represented by $\sigma$}}"']
    & & 
    X \otimes (X \Rightarrow_k Y)
    \ar[dl,"\text{function from item~\ref{it:k-bounded-eval}}"]
    \\
    & Y
    \end{tikzcd}
    \]    
\end{lemma}




\begin{proof}[Proof of Theorem~\ref{thm:single-use-closed}]
    Consider two types $X$ and $Y$.
 Take $k$ to be the number from Lemma~\ref{lem:no-irrelevant-questions-are-bounded}. This ensures that the type $X \Rightarrow_k Y$ is rich enough to represent all strategies that do not ask irrelevant questions. By Lemma~\ref{lem:eliminate-irrelevant-questions} the type is  rich enough to represent all single-use functions. Define the  function space $X \Rightarrow Y$ to be this type.   Define the evaluation function 
    \begin{align*}
    \text{eval} : X \otimes (X \Rightarrow Y) \to Y 
    \end{align*}
    to be the evaluation function in item~\ref{it:k-bounded-eval} in Lemma~\ref{lem:linear-type-of-k-bounded-strategies}. We will now prove that this type admits currying. Consider then some single use function 
    \begin{align*}
    f : Z \otimes X \to Y.
    \end{align*}
    By Lemma~\ref{lem:bounded-k}, there is some $\ell$ such that $f$ is represented by some  $\ell$-bounded strategy $\sigma$ in the game of type $Z \otimes X \to Y$. For strategies, there is a straightforward notion of currying, which is single-use, as given in the following claim.

    \begin{claim}
        Let $X$, $Y$, and $Z$ be linear types, and let $\ell \in \set{0,1,\ldots}$. For every $\ell$-bounded strategy $\sigma$ in the game  of type $Z \otimes X \to Y$  there is a single-use function $\Lambda : Z \to (X \Rightarrow_\ell Y)$, such that the following diagram commutes:
        \[
        \begin{tikzcd}
            [column sep=3.5cm]
        X
        \ar[r, "{x \mapsto z \otimes x}"]
        \ar[dr, "\text{strategy represented by $\Lambda(z)$}"']
        &
        Z \otimes X 
        \ar[d, "{\text{strategy represented by $\sigma$}}"] \\
        &
        Y
        \end{tikzcd}
        \]
        \end{claim}
        The currying operation from the above claim gives us almost the currying that is required in Theorem~\ref{thm:single-use-closed}. The only issue is that the output is in $X \Rightarrow_\ell Y$, where the number $\ell$ depends also on the extra set $Z$, instead of $X \Rightarrow_k Y$. To fix this, we post-compose it with the single-use function from $X \Rightarrow_\ell Y$ to $X \Rightarrow_k Y$  that was described in the remarks after Lemma~\ref{lem:bounded-k}. 
    

\end{proof}