\section{Single-use sets and functions}
\label{sec:single-use-sets}
We now turn to our first solution for the problem of orbit-finite function spaces.  Our solution builds on the idea from~\cite[Section 2.2]{stefanski-phd}, which is to consider only functions that are single-use. We describe this idea in Section~\ref{sec:single-use-functions-over-polynomial-orbit-finite-sets}, and we show how it almost, but not quite, achieves function spaces. Then, in the rest of this section, we show how function spaces can be recovered by using a more refined type system. 

\subsection{Single-use functions over polynomial orbit-finite sets}
\label{sec:single-use-functions-over-polynomial-orbit-finite-sets}

In the introduction, we have already given an intuitive description behind the single-use functions; these are functions that destroy the argument after any use of it, such as comparing it to a constant. We now give a more formal definition.

We do not define the single-use functions on all orbit-finite sets, but only a syntactically defined fragment, namely the \emph{polynomial orbit-finite sets}, which are sets that can be generated from $1$ and $\atoms$ using  
 products $\times$ and disjoint unions $+$. Therefore, we will allow orbit-finite sets like $1 + \atoms^2$, but we will not allow orbit-finite sets like 
the set of non-repeating pairs
$\setbuild{(a,b)}{$a \neq b \in \atoms$}$ or the set of unordered pairs $\setbuild{\set{a,b}}{$a \neq b \in \atoms$}$. It is an open problem to find a satisfactory definition of single-use functions on all orbit-finite sets. A simple hack is to use a quotienting construction, similar to Section~\ref{sec:quotient-category}, but what we would really like to do is to identify some extra structure in a set, possibly an action of some yet unknown group or semigroup, which enables us to speak about single-use functions.

Consider two polynomial orbit-finite sets $X$ and $Y$. To define which functions $X \to Y$ are single-use, we use an inductive definition. We begin with certain functions that are considered single-use, such as the equality test of type $\atoms \times \atoms \to 1 + 1$. These functions are called the \emph{prime functions}, and their full list is given in Figure~\ref{fig:prime-morphisms-without-with}. Next, we combine the prime functions into new ones using three combinators. The first, and most important, combinator is  function composition. Then, we have two combinators for the two type constructors: if two functions $f_1 : X_1 \to Y_1$ and $f_2 : X_2 \to Y_2$ are single-use, then the same is true for:
\begin{align*}
    f_1 \times f_2 : X_1 \times X_2 \to Y_1 \times Y_2
    \qquad &
    f_1 + f_2 : X_1 + X_2 \to Y_1 + Y_2 \\
    \scriptstyle  (x_1,x_2) \mapsto (f_1(x_1),f_2(x_2)) 
    \qquad &
    {\scriptstyle \text{left}(x_1) \mapsto \text{left}(f_1(x_1)) 
        \quad 
        \text{right}(x_2) \mapsto \text{right}(f_2(x_2)) }.
\end{align*}

Crucially, the list of prime single-use function does not contain the copying function $a \in \atoms \mapsto (a,a) \in \atoms^2$. Therefore, an alternative name for the single-use functions is \emph{copyless}. If we added copying, then we would get all finitely supported functions~\cite[Lemma 23]{stefanski-phd}.

\begin{table}[h!]
    \centering
    \begin{tabular}{lll}
        \textbf{Function} & \textbf{Type} & \textbf{Definition} \\ \\
        \emph{Functions about $\atoms$} \\
        equality test & $\atoms \times \atoms \to 1 + 1$ & $a, b \mapsto \text{if } a = b \text{ then true else false}$ \\
        constant $a$ & $1 \to \atoms$ & $x \mapsto a$ \\
        identity & $\atoms \to \atoms$ & $x \mapsto x$ \\
        \\
        \emph{Functions about \(\times\)} \\
        commutativity of $\times$ & $X \times Y \to Y \times X$ & $x \times y \mapsto y \times x$ \\
        first projection & $X \times Y \to X$ & $x \times y \mapsto x$ \\
        second projection & $X \times Y \to Y$ & $x \times y \mapsto y$ \\
        append 1 & $X \to X \times 1$ & $x \mapsto x \times ()$ \\
        associativity of $\times$ & $(X \times Y) \times Z \to X \times (Y \times Z)$ & $(x \times y) \times z \mapsto x \times (y \times z)$ \\ \\
        \emph{Functions about \(+\)} \\
        first co-projection & $X \to X + Y$ & $x \mapsto \text{left}(x)$ \\
        second co-projection & $Y \to X + Y$ & $y \mapsto \text{right}(y)$ \\
        co-diagonal & $X + X \to X$ & $\left\{\begin{tabular}{l}
            $\text{left}(x) \mapsto x$\\
            $\text{right}(x) \mapsto x$
            \end{tabular}\right.$ \\
        commutativity of $+$ & $X + Y \to Y + X$ & $\left\{\begin{tabular}{l}
        $\textrm{left}(x) \mapsto \textrm{right}(x)$\\
        $\textrm{right}(y) \mapsto \textrm{left}(y)$
        \end{tabular}\right.$ \\
        associativity of $+$ & $(X + Y) + Z \to X + (Y + Z)$ & $\left\{
        \begin{tabular}{l}
        $\text{left}(\text{left}(x)) \mapsto \text{left}(x)$\\
        $\text{left}(\text{right}(y)) \mapsto \text{right}(\text{left}(y))$\\
        $\text{right}(z)\mapsto \text{right}(\text{right}(z))$
        \end{tabular}\right.$ \\
        \\
        \emph{Distributivity}
        \\
        $+$ distributes over $\times$ & $X \times (Y + Z) \to (X \times Y) + (X \times Z)$ & $\left\{\begin{tabular}{l}
            $x \times (\text{left}(y)) \mapsto \text{left}(x \times y)$\\
            $x \times (\text{right}(z)) \mapsto \text{right}(x \times z)$
        \end{tabular}\right.$ \\
        \\
    \end{tabular}
    \caption{The prime single-use functions for polynomial orbit-finite sets $X, Y$ and $Z$.}
    \label{fig:prime-morphisms-without-with}
\end{table}





\begin{example}\label{ex:six-compositions}
    Consider function of type $\atoms^3 \to \atoms$ which inputs a triple $(a,b,c)$ of atoms and returns $a$ if $c$ is equal to Mark, and $b$ otherwise. This function is a single-use function. It is obtained by composing the six functions listed below:
\begin{center}
    \begin{tabular}{ll}
        Function & Type after function \\
        \hline
        Append 1. & $ \atoms \times \atoms \times \atoms \times 1$ \\
        Replace added $1$ with Mark using the constant function. & $\atoms \times \atoms \times \atoms \times \atoms$ \\
        Apply the equality test to the last two components. & $ \atoms \times \atoms \times (1+1)$ \\
        Distribute. & $ \atoms \times \atoms \times 1 +   \atoms \times \atoms \times 1$ \\
        Project to first and second components, respectively. & $\atoms + \atoms$ \\
        Co-diagonal & $\atoms$ 
    \end{tabular}
\end{center}
To justify this description, one should also show that the six functions are single-use. Three of the functions, namely append 1, distributivity and co-diagonal are prime functions. The other three are obtained by combining prime functions using the combinators. For example, the equality test is paired, using the combinator for $\times$, with the identity on the remaining two atoms. 
\end{example}

The design goal of the single-use restriction is to have orbit-finite function spaces. The rough idea is that a single-use function can only use a bounded number of atoms in its source code, which guarantees orbit-finiteness of the function space. 

\begin{example}\label{ex:first-single-use-function-space}
    Consider function of type $\atoms \to 1 + 1$, which can be seen as subsets of the atoms, with $1+1$ representing the Booleans. We will consider two function spaces: the larger space of all finitely supported functions, and the smaller space of the single-use function.
    
    A function in the larger space is any finitely supported subset of the atoms; such subsets are the same as the finite and co-finite subsets. Therefore, the larger function space admits an equivariant bijection with a disjoint union of two copies of the finite powerset $\powerset_{\text{fin}} \atoms$, in particular it is not orbit-finite.
    
    Consider now the smaller single-use function space. There are four possible functions of this kind: (a) always return true; (b) always return false; (c) check for equality with some fixed atom $a$; (d) check for disequality with with some fixed atom $a$.  Therefore,  the set of single-use functions  admits an equivariant bijection with the orbit-finite set $1 + 1 + \atoms + \atoms$.
    \exampleend
\end{example}


The above example shows that the space of single-use functions of a some type $X \to Y$ is orbit-finite, and in fact it can be described using a polynomial orbit-finite set. This is true for every choice of polynomial orbit-finite sets $X$ and $Y$, as proved in~\cite[Theorem 5]{stefanski-phd}, and illustrated in the following example. 

\begin{example}\label{ex:decision-tree-types} Assume that the input type $X$ is some power of the atoms $\atoms^k$, and the output type $Y$ does not use atoms, e.g.~it is~$Y = 1 +1$. The assumption on the input type can be made without loss of generality using distributivity, while the assumption on the output type is a proper restriction, but it will allow us to skip some technical details of the general construction while retaining the important intuitions. 
We describe below a type $\atoms^k \Rightarrow Y$ that represents all single-use functions from $\atoms^k$ to $Y$. 

This type is defined by  induction on $k$. In the base case of $k=0$ we simply need to give a value from the output type, and therefore $\atoms^0 \Rightarrow Y$ is the same as $Y$. Consider now the induction step of $k > 0$.  
    We observe that a single-use function that inputs $\atoms^k$ must begin with some equality test, and then continue with one of two single-use function that have fewer arguments (one for the case when the equality test returns true, and one for the other case). This observation leads to the following definition of the type $\atoms^k \Rightarrow Y$:
\begin{align*}
\myunderbrace{ \coprod_{i \in \set{1,\ldots,k}} 
    \atoms \times (\atoms^{k-1} \Rightarrow Y) \times (\atoms^{k-1} \Rightarrow Y)
 }{starts by comparing $i$-th  \\
 coordinate to some constant}  \quad + \quad 
\myunderbrace{\coprod_{i, j \in \set{1,\ldots,k}} (\atoms^{k-2} \to Y) \times (\atoms^{k-2} \to Y)}{
    starts by comparing the \\ 
    $i$-th and $j$-th coordinates}.
\end{align*}
Note that the above representation of the function space is not necessarily unique, i.e.~the same function can be represented in several different ways. For example, the order in which equality tests are performed will matter for the representation, but might not matter for the function. This is not something that we worry about, and we will use function spaces with non-unique representations in the paper, see also Section~\ref{sec:quotient-category} for how we deal with non-uniqueness. \exampleend
\end{example}


\textbf{Problem with Currying.}
Unfortunately, the proposal illustrated in Example~\ref{ex:decision-tree-types} and described in more detail in~\cite{stefanski-phd}  does not give a satisfactory solution to the problem of function spaces. The problem is that the set of representations $X \Rightarrow Y$ should  also support operations on functions. More specifically, we should be able to indicate single-use operations which do the following:
\begin{enumerate}
    \item evaluation: a single-use function from $(X \Rightarrow Y) \times X$ to $Y$ which inputs a representation of a function and applies it to an argument;
    \item composition: a function from $(X \Rightarrow Y) \times (Y \Rightarrow Z)$ to $(X \Rightarrow Z)$ which inputs the representations of two functions and returns a representation of their composition.
    \item Currying: for each single-use function from  $ X \times Y $ to $Z$, there should be a single-use function from $X$ to $Y \Rightarrow Z$ which inputs a first argument and returns a representation of the  partially applied function;
\end{enumerate}
Only in the presence of all of these operations can we speak of a function space, and the corresponding category can be called closed. (Composition can be obtained through evaluation and Currying, so the essential operations are evaluation and Currying.) The following example shows that the Currying operation is not single-use, and therefore the space of single-use functions as defined in this section is not closed. 


\begin{example}\label{ex:Currying-not-single-use}
    Consider  the single-use function
    \begin{align*}
    f : \atoms \times \atoms \to 1 + 1 \qquad (a,b)  
    \mapsto \begin{cases}
        \text{result of test $a = $ Mark} & \text{if } b = \text{Eve} \\
        \text{result of test $a = $ John} & \text{otherwise}.
        \end{cases}
    \end{align*}
The Currying of this function, is a new function which  maps a first argument $a \in \atoms$ to the partially applied function $f(a,\_)$. This Currying is
\begin{align*}
    a \mapsto \begin{cases}
        b \mapsto b = \text{Eve} & \text{if } a = \text{Mark} \\
        b \mapsto b \neq \text{Eve} & \text{if } a = \text{John} \\
        b \mapsto \text{false} & \text{otherwise}
        \end{cases}
\end{align*}
Recall that in Example~\ref{ex:first-single-use-function-space} we showed that the space of single-use functions of type $\atoms \to 1 + 1$ can be represented as $1 + 1 + \atoms + \atoms$. If we use this representation,  then the Currying of the function $f$  is not single-use, because we need to compare the input atom $a$ to two constants, Mark and John. If we use the representation from Example~\ref{ex:decision-tree-types}, then the corresponding type will be $\atoms \otimes (1 + 1)^2$, but the problems with Currying will persist. \exampleend
\end{example}

For similar reasons, 
the function space proposed above will also not support function composition, which means that it cannot be used to convert automata into single-use monoids, as we would like to do, since the resulting monoid would need to use function composition as its monoid operation\footnote{This problem is solved in~\cite{bojanczykstefanski2020} and~\cite{stefanski-phd} in a different way, namely by showing that every orbit-finite monoid is necessarily divides a single-use monoid, using a  Krohn-Rhodes construction. However, this construction is difficult and delicate, in particular it does not work for atoms that have more structure than equality alone. In contrast, the proposal that we give in this paper works for other kinds of atoms, as discussed in Section~\ref{sec:beyond-equality}. }

To solve the problems above, we will introduce a more refined type system, which is based on linear types.  The main idea is to pay more attention to type  in Example~\ref{ex:decision-tree-types}. In that definition, we describe a single-use use function by specifying the first equality test that it makes, and then giving two descriptions of the functions that will be used in each of the two possible outcomes of the equality test. The main observation is that these two outcomes are mutually exclusive, and therefore we intend to use only one of the two descriptions. For this reason, we will use a  type constructor $\&$ that comes from linear logic. The intended meaning is  that an object of type $X \& Y$ consists of two objects, but with the ability to use only one of them. Since linear logic uses $\otimes$  for the  product that we have so far denoted by $\times$, we will also follow that convention. Using these two kinds of product, the appropriate type for Example~\ref{ex:decision-tree-types} will now become: 
\begin{align*}
\coprod_{i \in \set{1,\ldots,k}} 
        \atoms \otimes ((\atoms^{k-1} \Rightarrow Y) \& (\atoms^{k-1} \Rightarrow Y))
          \quad + \quad 
    \coprod_{i, j \in \set{1,\ldots,k}} (\atoms^{k-2} \to Y) \& (\atoms^{k-2} \to Y).
    \end{align*}
Under this definition, the problems from Example~\ref{ex:Currying-not-single-use} will be solved, at least for the particular type considered in that example. However, by introducing a new type constructor, we will have to redefine the single-use functions, and then we will have to give a representation of functions that allow this new type constructor, without incurring the need to add any other new type constructors. This is the subject of the next section.

\subsection{Linear types and single-use functions on them}
\label{sec:linear-types-and-single-use-functions}
As mentioned above, to solve the problems with single-use function spaces, we will consider a type system with two kinds of products, as in the following definitoin.
\begin{definition}[Linear types]\label{def:datatypes}
    A \emph{linear type} is any expression constructed from the atomic types $1$ and $\atoms$ using three\footnote{We managed to set up our type system without using the type constructor  $\rotatebox[origin=c]{180}{\&}$.} binary type constructors $+, \&$ and $\otimes$. 
\end{definition}
In our linear types, it is only the products that are differentiated, while  $+$ comes in only one version. 
    Here is the intuitive explanation of the difference between the two kinds of products, following Girard~\cite[p.2]{girard1995advances}. Having a pair $x \otimes y$ is like having the ability of using both components $x$ and $y$. On the other hand, having a pair $x \& y$ is like having the ability to use one of the two components, at our choice, but not both at once. For example, the input type of the equality test will be $\atoms \otimes \atoms$ not $\atoms \& \atoms$, since the test will need to consume both arguments. This intuition can only go so far; for example it is not entirely clear what ``our choice'' means. We revisit this intuition in  the appendix, where game semantics will be used to indicate who makes which choices. 



We think of each linear type $X$ as representing a set $\sem X$, as defined below:
\begin{align*}
\hspace{-0.6cm}
    \sem{1} = 1
\quad 
\sem{\atoms} = \atoms 
\quad 
\sem{X+Y} = \sem X + \sem Y 
\quad 
\sem{X \otimes Y} = \sem X \times \sem Y 
\quad
\sem{X \& Y} = \sem X \times \sem Y.
\end{align*}
All sets that arise in this way will be polynomial orbit-finite sets.
Note that the two kinds of product represent the same set, namely the set of pairs in the usual set-theoretic sense. 
However, the two type constructors will be  different, because different functions will be allowed to operate on them. As the expression goes, ``the proof of the pudding is in the eating''; in this case the pudding is the types and the eating is the functions.  

As we did in Section~\ref{sec:single-use-functions-over-polynomial-orbit-finite-sets}, the single-use functions will be defined in terms of prime functions and combinators. The combinators are the same, except that instead of $f_1 \times f_2$ we now have two ways of pairing functions, using $\otimes$ and $\&$. The prime functions are inherited from the previous system, with $\times$ understood as $\otimes$, together with a few new functions for $\&$, as described in Table~\ref{fig:prime-morphisms-with-with}. This is summarized in the following definition.



\begin{definition}[Single-use functions] The class of single-use functions is the least class of functions with the following properties:
    \begin{enumerate}
        \item It contains the functions from Tables~\ref{fig:prime-morphisms-without-with} and~\ref{fig:prime-morphisms-with-with}, with $\times$ in Table~\ref{fig:prime-morphisms-without-with} understood as $\otimes$;
        \item It is closed under composition, as well as under combining functions using  $+$, $\otimes$ and $\&$. 
    \end{enumerate}
\end{definition}




\begin{table}[h!]
    \centering
    \begin{tabular}{lll}
        \textbf{Function} & \textbf{Type} & \textbf{Definition} \\ \\
        diagonal  & $X \to X \& X$ & $x \mapsto x \& x$ \\
        first projection & $X \& Y \to X$ & $x \& y \mapsto x$ \\
        second projection & $X \& Y \to Y$ & $x \& y \mapsto y$ \\
        $\&$ distributes over $\otimes$ & $X \otimes (Y \& Z) \to (X \otimes Y) \& (X \otimes Z)$ & $x \otimes (y \& z) \mapsto (x \otimes y) \& (x \otimes z)$ \\
        $\&$ distributes over $+$ & $X + (Y \& Z) \to (X \& Y) + (X \& Z)$ & $\left\{\begin{tabular}{l}
        $x \& \text{left}(y) \mapsto \text{left}(x \& y)$\\
        $x \& \text{right}(z) \mapsto \text{right}(x \& z)$
        \end{tabular}\right.$ \\ \\ 
    \end{tabular}
    \caption{Prime single use functions that involve $\&$.}
    \label{fig:prime-morphisms-with-with}
\end{table}


   


Formally speaking, a single-use function consists of an input linear type $X$, an output linear type $Y$, and a function between the sets $\sem X$ and $\sem Y$ that is generated using the prime functions and combinators from the above definition.  As was the case in Section~\ref{sec:single-use-functions-over-polynomial-orbit-finite-sets}, all single-use functions are  finitely supported.  Therefore, one can think of the single-use functions of type $X \to Y$ as being a subset of the set of all finitely supported functions from $\sem X$ to $\sem Y$. This subset is strict: as we will see, the space of single-use functions will be orbit-finite, unlike the space of all finitely supported functions. We will be thinking of the single-use functions as a category.




\begin{definition}[Category of single-use sets]\label{def:single-use-category}
    The category of single-use sets is:
    \begin{enumerate}
        \item The objects are linear types, as per Definition~\ref{def:datatypes}.
        \item Morphisms between types $X$ and $Y$ are partial single-use functions from $\sem X$ to $\sem Y$.
    \end{enumerate}
\end{definition}

In the very definition of the above category, there is a faithful functor to the category of  orbit-finite sets with finitely supported functions. This functor maps objects $X$ to their underlying sets $\sem X$, which are orbit-finite sets, and it maps morphisms to the corresponding functions. The functions seen to be finitely supported, and the functor is faithful  by definition, since the morphisms in Definition~\ref{def:single-use-category} are defined to be single-use functions. 



% We use the name \emph{single-use morphisms} for morphisms of the above category. A single-use morphism is the same as a single-use function, together with the information about the types between which it is a morphism. For example, the idenitity function on $\sem{\atoms \& \atoms}$ is the same as the identity function on $\sem{\atoms \otimes \atoms}$, because the two underlying sets are the same, but the corresponding morphisms are different. 
% From now on, instead of talking about single-use functions of type $\sem X \to \sem Y$, we will speak about single-use morphisms of type $X \to Y$.



 
The main technical result of this paper is that the category of single-use sets has function spaces, as stated in the following theorem.  The appropriate product will be $\otimes$, and not $\&$. Since the Cartesian product in our category is $\&$ and not $\otimes$, this means that the result we are targetting is symmetric monoidal closed with respect to $\otimes$, and not Cartesian closed. (The same situation will happen  for vector spaces later in this paper.)  Our theorem stops a bit short of saying that the category is monoidal closed, since several different elements of the function space might represent the same function. This non-uniqueness of representation can be overcome  by quotienting the function space  by an equivalence relation that identifies two elements if they represent the same single-use function; this is explained in Appendix~\ref{sec:quotient-category}. For now, we stick with non-uniqueness of representation, and we state the theorem as follows.

\begin{theorem}\label{thm:single-use-closed}
    Let $V$ and $W$ be objects (i.e.~linear types). There exists an object, denoted by  $\funspace V W$, and a morphism (i.e.~a partial single-use function)
    \begin{align*}
    eval : (\funspace V W) \otimes V \to  W
    \end{align*}
    with the following property. For every morphism
    \begin{align*}
    f : {X \otimes V} \to  W
    \end{align*}
    there is a (not necessarily unique) morphism
    \begin{align*}
    h :  X \to (\funspace V W)
    \end{align*}
    such that the following diagram commutes:
    \[
    \begin{tikzcd}
    X \otimes V 
    \arrow[r,"h \otimes id"]
    \arrow[dr,"f"']
    &
    (\funspace V W) \otimes V
    \arrow[d,"eval"] \\
    &
    W
    \end{tikzcd}
    \]
\end{theorem}

The above theorem is the main technical contribution of this paper, and its proof is presented in Appendix~\ref{sec:game-semantics}. The difficulty in the proof is finding a representation of the single-use  functions that is rich enough to capture all functions, but simple enough to be described by a linear type (in particular, the corresponding set will be orbit-finite). In  Section~\ref{sec:single-use-functions-over-polynomial-orbit-finite-sets}, when the types did not have $\&$, we could pull off a relatively simple construction, which was possible mainly due to the strong distributivity rules that allowed to convert each type into a normal like $\atoms^{n_1} + \cdots + \atoms^{n_\ell}$. In the presence of $\&$, the distributivity rules are not as strong, and the way in which a single-use program can interact with its input is rather subtle. 
Our solution, and the technical core of this paper, is to use game semantics, appropriately extended to describe the type $\atoms$ and the operations on it that are allowed. This solution is presented entirely in the appendix.









