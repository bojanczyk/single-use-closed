
\section{A quotient construction}
\label{sec:quotient-category}
A drawback of Theorem~\ref{thm:single-use-closed} is that the function space $\funspace V W$ can contain different representations of the same function; this will mean that Currying is not unique. To overcome this issue, we use a simple quotient construction. 
Define a \emph{partial equivalence relation} to be a relation that is symmetric and transitive, but not necessarily reflexive.  This is the same as (complete) equivalence relation on some subset. We will use a partial equivalence on the function space $X \Rightarrow Y$ to: (1)
 remove objects that do not represent any morphism; (2) identify two objects if they represent the same morphism. After such a quotient, the function space will have unique representations for functions. 




\begin{definition}
    The \emph{quotiented single-use category} is: 
    \begin{itemize}
    \item Objects are pairs (linear type $X$, equivariant partial equivalence relation on $\sem X$);
    \item Morphisms between objects $(X,\sim_X)$ and $(Y,\sim_Y)$ are single-use functions from $\sem X$ to $\sem Y$ such that the domain of the function is contained in the domain of $\sim_X$, and equivalent inputs are mapped to equivalent outputs.
    \end{itemize}
\end{definition}

The quotiented single-use category is also equipped with a tensor product $\otimes$ on its objects.
\begin{theorem}
    The quotiented single-use category, equipped with the tensor product $\otimes$, is a monoidal closed category, i.e.~it satisfies the conclusions of Theorem~\ref{thm:single-use-closed}, but, furthermore, the morphism $h$ is unique.
\end{theorem}

