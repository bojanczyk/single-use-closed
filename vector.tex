\section{Orbit-finite vector spaces}
As a complement to the single-use category described before, we present a second example of a symmetric monoidal closed category that  uses atoms. In this example, instead of restricting functions, we generalize them. This generalization is based on the  orbit-finitely spanned vector spaces that were introduced in~\cite{bojanczykKM21OrbitFiniteVector}. As mentioned in the introduction,  this section contains no new technical results beyond those from~\cite{bojanczykKM21OrbitFiniteVector}. 
% Also, unlike the single-use setup, the construction in this section is rather delicate: it is only known to work for atoms that have equality and order, and it is known to fail for certain choices of atoms, such as the Rado graph, for which the single-use results hold.

To explain the intuitive reason behind the usefulness of vector spaces, consider finite (and not orbit-finite) sets. The number of functions from set with $n$ elements to a set with $m$ elements is $m^n$. However, if we consider linear maps instead of functions without any structure, then we get an exponential improvement, namely instead of $m^n$ we will have $m \cdot n$. 
This is because a linear map from a vector space of dimension $n$ to a vector space of dimension $m$ can be specified by giving an $m \times n$ matrix. As we show in this section,  in the orbit-finite world, the improvement is more pronounced: from infinite to finite.

For the sake of concreteness, all  vector spaces will be over the field of rational numbers. However, the choice of the field will not affect the results. The general idea is that the field has no atoms in it, and therefore it has a trivial action of atom automorphisms. 
% In particular, when below we say that scalar multiplication is equivariant, we mean that for every field element $c$, the function $v \mapsto c \cdot v$ is equivariant.
\begin{definition}[Vector space with atoms]
    A \emph{vector space with atoms} is a set with atoms, equipped with a vector space structure, such that the vector space structure is equivariant, i.e.~vector addition is equivariant, and  scalar multiplication $v \mapsto cv$ is equivariant for every field element $c$.
\end{definition}


\begin{example} \label{ex:lina} For a set $X$, let us write $\Lin X$ for the vector space that consists of finite linear combinations of elements from $X$. When $X$ is the set of atoms, the resulting vector space $\Lin \atoms$ is a vector space with atoms, because it comes equipped with a natural action of atom automorphisms.  An example vector in this space is $
3 \cdot \text{John} + 2 \cdot \text{Eve} - 5 \cdot \text{Mark}.
$ \exampleend
\end{example}

When talking about vector spaces with atoms, one has to be careful about using bases. This is because finding a basis might require choice, and choice is not available in the presence of atoms. 
For example, consider space $\Lin \atoms$ from Example~\ref{ex:lina}, and its subspace $V$ which contains those vectors where all coefficients sum up to zero. This space is spanned (i.e.~generated using linear combinations) by the orbit-finite set which consists of all pairs $a - b$, where $a$ and $b$ are distinct atoms.
However, this spanning set is not a basis, since it contains linearly dependent vectors, for example John - Eve and Eve - John. Keeping only one of these vectors would require choice, which cannot be done using equivariant functions, and in fact this space has no basis that is equivariant~\cite[Example 6]{bojanczykKM21OrbitFiniteVector}. For this reason, the appropriate notion of finiteness is having an orbit-finite spanning set, which is not necessarily linearly independent. This leads to the following category.

\begin{definition}\label{def:orbit-finite-vector-space-category}
    The category of orbit-finitely spanned vector spaces is:
    \begin{enumerate}
        \item The objects are vector spaces with atoms that have an orbit-finite spanning set.
        \item The morphisms are equivariant linear maps.
    \end{enumerate}
\end{definition}

For the above category, we will use tensor product $\otimes$; this is the usual notion of tensor product for vector spaces. For example, if we take vector spaces $\Lin X$ and $\Lin Y$, i.e.~finite linear combinations of orbit-finite sets $X$ and $Y$ respectively, then the resulting tensor product is $\Lin (X \times Y)$. More generally, the tensor product of two orbit-finitely spanned vector spaces is also orbit-finitely spanned~\cite[Theorem VI.3]{bojanczykKM21OrbitFiniteVector}. 

% In the terminology of category theory, the following theorem says that the category of orbit-finite vector spaces with atoms is symmetric monoidal closed, with respect to the tensor product. However, since the readers might not be familiar with category theory, we unfold the definitions in the statement of the theorem. 


\begin{theorem}\label{thm:orbit-finite-vector-space-closed} The category of orbit-finitely spanned vector spaces is symmetric monoidal closed, with respect to the tensor product.
    % Let $V$ and $W$ be orbit-finitely spanned vector spaces. There exists an orbit-finitely spanned  vector space, denoted by  $\funspace V W$, and an equivariant linear map
    % \begin{align*}
    % eval : (\funspace V W) \otimes V \to W
    % \end{align*}
    % with the following property. For every equivariant linear map
    % \begin{align*}
    % f : X \otimes V \to W
    % \end{align*}
    % there is a unique equivariant linear map
    % \begin{align*}
    % h : X \to (\funspace V W)
    % \end{align*}
    % such that the following diagram commutes:
    % \[
    % \begin{tikzcd}
    % X \otimes V 
    % \arrow[r,"h \otimes id"]
    % \arrow[dr,"f"']
    % &
    % (\funspace V W) \otimes V
    % \arrow[d,"eval"] \\
    % &
    % W
    % \end{tikzcd}
    % \]
\end{theorem}


A corollary of the above theorem is that, in the category of orbit-finitely spanned vector spaces, deterministic automata recognize the same languages as monoids. This is because the usual translation can be done, with a deterministic automaton of states $Q$ being mapped to a monoid with elements $\funspace Q Q$. This was already observed in~\cite[Theorem VIII.3]{bojanczykKM21OrbitFiniteVector}.

However, there are two limitations of the orbit-finitely spanned vector spaces. 

The first limitation is that the existence of function spaces is dependent on the choice of atoms. Theorem~\ref{thm:orbit-finite-vector-space-closed}  works when the atoms have equality only, and it also works when the atoms are equipped with a total order. This is because the dual spaces are orbit-finitely spanned in these cases, as proved in~\cite[Corollary VI.5]{bojanczykKM21OrbitFiniteVector}. However, the dual spaces are no longer orbit-finitely spanned for other choices of  atoms, such as the Rado graph, see~\cite[Example 9]{bojanczykKM21OrbitFiniteVector}. This is in contrast to the single-use category, where the existence of function spaces is independent of the choice of atoms. 

A second limitation is that it does not support two-way automata, unlike the single-use category (see Section~\ref{sec:two-way-automata}). This is because this category generalizes the orbit-finite category (i.e.~it admits a faithful functor from it), and in the orbit-finite category emptiness is undecidable for deterministic two-way automata~\cite[Theorem 5.3]{nevenFiniteStateMachines2004}. This precludes the kind of traced construction that we did in the single-use category. This issue appears already without atoms: the category of finite-dimensional vector spaces is not traced with respect to the sum $+$ of vector spaces. 

