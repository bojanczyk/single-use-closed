\section{Two-way automata}  


\begin{proof}[Proof of Lemma~\ref{lem:traced-finite-iteration}]
    We only need the weaker assumption, which is that the function $f$ is finitely supported. Consider some set of atoms that supports $f$. This same set of atoms will also support all functions $f^{(k)}$, and it will also support their domains. In particular, the domains of these functions will form a decreasing sequence of subsets of $\sem{A + X}$, with all subsets having the same support. Since an orbit-finite set can have finitely many subsets with a given support, this sequence must terminate in finitely many steps.
\end{proof} 

\begin{proof}[Proof of Theorem~\ref{thm:two-way-automata}]
    The transition function in the automaton tells us how it behaves for a single input letter. We now extend this function to input strings.    For every input string $w$, we will define  a single-use function 
    \begin{align*}
       \delta_w :  Q+Q \to Q + Q + 1
    \end{align*}
    that describes the behaviour of the automaton on $w$. 
    
    This is done by induction on the length of the input string. Suppose that we have already defined the functions for  input strings $w_1$ and $w_2$.
    We want to compose them into a function for the string $w_1 w_2$. This function will need to account for multiple crossings of the head on the boundary between $w_1$ and $w_2$, as described in the following picture: 
    \mypic{13}
    Taking care of these crossings is exactly what is accomplished by the trace. Indeed, consider the function 
        \begin{align*}
            f : \myunderbrace{Q+Q}{$A$} + \myunderbrace{Q+Q}{$X$} \to \myunderbrace{Q+Q + 1}{$B$} + \myunderbrace{Q+Q}{$X$},
        \end{align*}
        where $A$ represents entering $w_1 w_2$ from either the left or right, $B$ represents exiting $w_1 w_2$ from either the left or right or accepting,  and $X$ represents the boundary between $w_1$ and $w_2$. The trace of $f$ is exactly the function that we need, in particular it will be undefined for inputs that lead to an infinite run.


        The definition of $\delta_w$ described above describes a function 
        \begin{align*}
        \text{input strings} \to \text{single-use functions of type }Q+Q \to Q + Q + 1.
        \end{align*}
        This function is a monoid homomorphism, for a suitably defined  monoid structure on the image. We use the name \emph{transition monoid} for the output monoid. We can view the transition monoid  as a linear type, namely the function space 
        \begin{align*}
        \funspace {Q+Q} {Q + Q + 1}.
        \end{align*}

        Like any linear type, the transition monoid represents an orbit-finite set. This set is equipped with a monoid operation. Although this monoid operation is not single-use (because tracing is not single-use, as an operation on function spaces), it is still an equivariant operation. Therefore, we have an orbit-finite monoid. We can now use a standard fix-point algorithm on orbit-finite sets~\cite{bojanczyk_slightly2018} to compute a representation of the sub-monoid that is generated by the images of the single letters. Finally, we can check if this sub-monoid contains an accepting element, i.e.~a transition function that maps the initial configuration (initial state at the left of the word) to an accepting state. 
\end{proof}

