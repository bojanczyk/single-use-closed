\section{Proof of Theorem~\ref{thm:orbit-finite-vector-space-closed}}
 We apply the standard proof that the category of vector spaces (without atoms and orbit-finiteness) is symmetric monoidal closed. We then observe that: (1) both evaluation and Currying are equivariant; (2) the resulting space is orbit-finitely spanned. 
    
    For the first observation (1), we simply note that the construction of the function space, as well as  the evaluation and Currying morphisms, are constructed using the language of set theory, and therefore they will necessarily be equivariant~\cite[Equivariance Principle]{bojanczyk_slightly2018}.

    Let us now explain the second observation (2), about the function space being orbit-finitely spanned. This uses a non-trivial result from~\cite{bojanczykKM21OrbitFiniteVector} which says that orbit-finitely spanned spaces are closed under duals. 
    The space $\funspace V W$ is defined to be the space of finitely supported linear maps from $V$ to $W$, with the natural vector space structure. This can be seen as a subspace of a larger space, call it $\funspacenonfs V W$, which contains all linear maps, not necessarily finitely supported. A standard result in linear algebra is that $\funspacenonfs V W$ is linearly isomorphic to $ \funspacenonfs {V \otimes W} 1$, where $1$ is the field. This isomorphism is equivariant, and it preserves the property of being finitely supported. Therefore, this gives us an equivariant linear isomorphism between $\funspace V W$ is isomorphic and $ \funspace {V \otimes W} 1$. The latter space is the dual of the orbit-finitely spanned space $V \otimes W$, and therefore it is orbit-finitely spanned by~\cite[Corollary VI.5]{bojanczykKM21OrbitFiniteVector}.


    We would like to remark that a similar result as Theorem~\ref{thm:orbit-finite-vector-space-closed} was observed in~\cite[Theorem 3.8]{przybylek2024note}, but using the smaller category of vector spaces that admit an orbit-finite basis. 
