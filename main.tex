\documentclass[a4paper,UKenglish,cleveref, autoref, numberwithinsect, thm-restate,anonymous]{lipics-v2021}

\usepackage{amsmath}
\usepackage{amsthm}
\usepackage{proof}
\usepackage{graphicx}
\usepackage{xspace}
\usepackage{color} 
\usepackage[all]{xy}
\usepackage{tikz-cd}
\usepackage{hyperref}
\usepackage{mathtools}
\usepackage{macros}
\usepackage{MnSymbol}
\usepackage[shortlabels]{enumitem}
%\usepackage{libertine}
\usepackage{booktabs}
\usepackage{relsize} 


\bibliographystyle{plainurl}% the mandatory bibstyle

\usepackage{todonotes}
\newcommand{\tito}[1]{\todo[inline,color=green!40]{Tito --- #1}}
\newcommand{\rafal}[1]{\todo[color=blue!40]{Rafal --- #1}}



\title{Function spaces for orbit-finite sets} %TODO Please add



\author{Miko{\l}aj Boja\'nczyk}{University of Warsaw}{bojan@mimuw.edu.pl}{https://orcid.org/0000-0002-1825-0097}{(Optional) author-specific funding acknowledgements}

%\author{L{\^{e}} Th{\`{a}}nh D\~ung Nguy{\~{\^e}}n}{ENS Lyon}{nltd@nguyentito.eu}{https://orcid.org/0000-0002-1825-0097}{(Optional) author-specific funding acknowledgements}


%\author{Rafa{\l} Stefa\'nski}{University of Warsaw}{rafal.stefanski@mimuw.edu.pl}{https://orcid.org/0000-0002-1825-0097}{(Optional) author-specific funding acknowledgements}



\authorrunning{M. Boja\'nczyk, L. T. D. Nguy{\^{e}}n and R. Stefa\'nski } %TODO mandatory. First: Use abbreviated first/middle names. Second (only in severe cases): Use first author plus 'et al.'

\Copyright{Miko{\l}aj Boja\'nczyk, L{\^{e}} Th{\`{a}}nh Dung Nguy{\^{e}}n and Rafa{\l} Stefa\'nski } %TODO mandatory, please use full first names. LIPIcs license is "CC-BY";  http://creativecommons.org/licenses/by/3.0/

\ccsdesc[100]{\textcolor{red}{Replace ccsdesc macro with valid one}} %TODO mandatory: Please choose ACM 2012 classifications from https://dl.acm.org/ccs/ccs_flat.cfm 

\keywords{Dummy keyword} %TODO mandatory; please add comma-separated list of keywords

\category{} %optional, e.g. invited paper

\relatedversion{} %optional, e.g. full version hosted on arXiv, HAL, or other respository/website
%\relatedversiondetails[linktext={opt. text shown instead of the URL}, cite=DBLP:books/mk/GrayR93]{Classification (e.g. Full Version, Extended Version, Previous Version}{URL to related version} %linktext and cite are optional

%\supplement{}%optional, e.g. related research data, source code, ... hosted on a repository like zenodo, figshare, GitHub, ...
%\supplementdetails[linktext={opt. text shown instead of the URL}, cite=DBLP:books/mk/GrayR93, subcategory={Description, Subcategory}, swhid={Software Heritage Identifier}]{General Classification (e.g. Software, Dataset, Model, ...)}{URL to related version} %linktext, cite, and subcategory are optional

%\funding{(Optional) general funding statement \dots}%optional, to capture a funding statement, which applies to all authors. Please enter author specific funding statements as fifth argument of the \author macro.

%\acknowledgements{I want to thank \dots}%optional

%\nolinenumbers %uncomment to disable line numbering
\hideLIPIcs


%Editor-only macros:: begin (do not touch as author)%%%%%%%%%%%%%%%%%%%%%%%%%%%%%%%%%%
\EventEditors{John Q. Open and Joan R. Access}
\EventNoEds{2}
\EventLongTitle{42nd Conference on Very Important Topics (CVIT 2016)}
\EventShortTitle{CVIT 2016}
\EventAcronym{CVIT}
\EventYear{2016}
\EventDate{December 24--27, 2016}
\EventLocation{Little Whinging, United Kingdom}
\EventLogo{}
\SeriesVolume{42}
\ArticleNo{23}
%%%%%%%%%%%%%%%%%%%%%%%%%%%%%%%%%%%%%%%%%%%%%%%%%%%%%%



\begin{document}


\newcommand{\loli}{\multimap}


\maketitle 
\begin{abstract}
    Orbit-finite sets are a generalisation of finite sets, and as such support many operations allowed for finite sets, such as pairing, quotienting, or taking subsets. However, they do not support function spaces, i.e.~if $X$ and $Y$ are orbit-finite sets, then the space of finitely supported functions from $X$ to $Y$ is not orbit-finite. In this paper we propose two solutions to this problem: one is obtained by generalising the notion of orbit-finite set, and the other one is obtained by restricting it. In both cases, function spaces and the original closure properties are retained. Curiously, both solutions are ``linear'': the generalisation is based on linear algebra, while the restriction is based on linear logic.
\end{abstract}

\section{Introduction}
The class of orbit-finite sets is a class of sets that contains all finite sets and some infinite sets, but still shares some properties with the class of finite sets.  The idea, which dates back to Fraenkel Mostowski models of set theory,  is to begin with an infinite set $\atoms$ of \emph{atoms} or \emph{urelements}. We think of the atoms as being names, such as Eve or John, and atoms can only be compared with respect to equality. Intuitively speaking, an  orbit-finite set is a set  that can be constructed using the atoms, such as $\atoms^2$ or $\atoms^*$, subject to the constraint that there are finitely many elements up to  renaming atoms. For example, $\atoms^2$ is orbit-finite because it has two elements up to renaming atoms, namely (John, John) and (John, Eve), while $\atoms^*$ is not orbit-finite, because the length of a sequence is invariant under renaming atoms, and there are infinitely many possible lengths. For a survey on orbit-finite sets, see~\cite{bojanczyk_slightly2018}.

The notion of orbit-finiteness can be seen as an attempt to find an appropriate notion of finiteness for the  nominal sets of Gabbay and Pitts~\cite{PittsAM:nomsns}.  This attempt emerged from the study of computational models such as monoids~\cite{bojanczykNominalMonoids2013} and automata~\cite{bojanczykAutomataTheoryNominal2014} over infinite alphabets. Since automata are also the main use case for the present paper, we illustrate orbit-finiteness using an automaton example. 

\begin{example}[An orbit-finite automaton]\label{ex:first-letter-repeats}
    Let  $L \subseteq \atoms^*$ be the language of  words in which the letter from the first position does not appear again. This language contains John $\cdot$ Mark $\cdot$ Mark $\cdot$ Eve, because John does not reappear, but it does not contain John $\cdot$ Mark $\cdot$ John. To recognise this language, we can use a deterministic automaton, which uses its state to remember the first letter. In this automaton, the input alphabet is $\Sigma = \atoms$ and  the state space is $Q = 1 + 1 + \atoms$. In this state space, there are two special states, namely the initial state and a rejecting error state, and furthermore there is one state for each atom $a \in \atoms$, which represents a situation where the first letter was $a$ but it has not been seen again yet. This state space is orbit-finite, since each of the three components in $Q$ represents a single orbit, with the orbits of the $1$ components being singletons.\exampleend
\end{example}

Orbit-finite sets have many advantages, which ensure that they are good setting for automata theory, and discrete mathematics in general. For example, an orbit-finite set can be represented in a finite way~\cite{bojanczyk_slightly2018}, which ensures that it becomes meaningful to talk about algorithms that input orbit-finite sets, such as an emptiness check for an automaton. Also, orbit-finite sets are closed under taking disjoint unions and products, which ensures that natural automata constructions, such as the union of two nondeterministic automata or the product of two deterministic automata can be performed.
%  (Closure of orbit-finite sets under products is not immediately obvious, for example the set $\atoms$ has one orbit, but the set $\atoms^3$ has five orbits.)

However, orbit-finite sets do not have all the closure properties of finite sets. Notably missing is the powerset construction, and more generally taking function spaces. For example, if we look at the powerset of $\atoms$, then this powerset will not be orbit-finite, since already the finite subsets give infinitely many orbits (two finite subsets of different size will be in different orbits). The lack of powersets means that one cannot do the subset construction from automata theory, and in particular deterministic and nondeterministic automata are not equivalent. This non-equivalence was known from the early days of automata for infinite alphabets~\cite{kaminskiFiniteMemoryAutomata1994}, and in fact, some decision problems, such as equivalence, are decidable for deterministic automata but undecidable for nondeterministic automata~\cite{nevenFiniteStateMachines2004}. Another construction that fails is converting a deterministic automaton into a monoid~\cite[p.~221]{bojanczykNominalMonoids2013}; this is because function spaces on orbit-finite sets are no longer orbit-finite, as explained in the following example. 

\begin{example}[Failure of the monoid construction]\label{ex:first-letter-repeats-monoid}
    Let us show that the  automaton from Example~\ref{ex:first-letter-repeats} cannot be converted into a monoid. The standard construction would be to define the monoid as the subset $M \subseteq Q \to Q$ of all state transformations, namely the subset generated by individual input letters. 
    Unfortunately, this construction does not work. This is because in order  for  two input words to give the same state transformation, they need to have the set of letters that appear in them. In particular, the corresponding set of set transformations is not orbit-finite, for the same reason as why the finite powerset is not orbit-finite. Not only does the standard construction not work, but also this language is not recognised by any orbit-finite monoid.\exampleend
\end{example}


An attempt to address this problem was provided in~\cite{stefansk-msc,stefanski-phd,bojanczykstefanski2020}, by using \emph{single-use} functions. The idea, which originates in linear types and linear logic,  is to restrict the functions so that they use each argument at most once. For example, consider the following two functions that input atoms and output Booleans:
\begin{align*}
a \in \atoms \mapsto 
\begin{cases}
    \text{true} & \text{if $a=$ John}\\
    \text{false} & \text{otherwise}
\end{cases}
\qquad \qquad 
a \in \atoms \mapsto 
\begin{cases}
    \text{true} & \text{if $a=$ John or $a = $Eve}\\
    \text{false} & \text{otherwise}
\end{cases}
\end{align*}
The first function is single use, since it compares the input atom to John only, while the second function is not single use, since it requires two comparisons, with John and Eve. Here is another example, which shows that the problems with the monoid construction from Example~\ref{ex:first-letter-repeats-monoid} could be blamed on a violation of the single-use condition.
\begin{example}
    Consider  the transition function of the automaton in Example~\ref{ex:first-letter-repeats}, which inputs a state in $1 + 1 + \atoms$ together with an input letter from $\atoms$, and returns a new state.  This function is not single use. Indeed, if the state is in $\atoms$, then the transition function it for equality with the input letter, but then still keeps the state for future comparisons. \exampleend
\end{example}

If one restricts attention to functions that are single use, much of the usual robustness of automata theory is recovered, with deterministic automata being equivalent to monoids, and both being equivalent to two-way deterministic automata~\cite{bojanczykstefanski2020}.

Despite the success of the single-use restriction in solving automata problems, one would ideally prefer a more principled approach, in which instead of defining single-use automata, we would define a more general object, namely single-use sets and functions. Then the definitions of  automata and monoids, as well theorems speaking about them, should arise automatically as a result of suitable closure properties of the sets and functions.

This approach was pursued in~\cite{stefanski-phd}, in which a \emph{category} of orbit-finite sets with single-use functions was proposed. In this corresponding category,  one can represent the set of all of single-use functions between two orbit-finite sets $X$ and $Y$ as a new set, call it $X \Rightarrow Y$, which is  also orbit-finite.  However, as we will see later in this paper, this proposal is not entirely satisfactory, since it fails to account for standard operations that one would like to perform on function spaces, most importantly partial application (currying). Note that partial application is crucial for converting an automaton into a monoid, since the monoid consists of partially applied transition functions, in which the input word is known, but the input state is not. In the language of category theory, the proposal from~\cite{stefanski-phd} failed to be a monoidal closed category.

Let us mention there have been several works using category theory to generalize classical operations on automata, such as the coalgebraic ``generalized powerset construction''~\cite{DBLP:journals/corr/abs-1302-1046}. The closest to the philosophy we exposed above might be the setting introduced by Colcombet and Petrişan~\cite{colcombet2020automata}, where automata in different categories are compared (see~\cite{ColcombetPS21,Aristote24} for applications). Within this setting, Nguy{\~{ê}}n and Pradic have studied some properties of automata over monoidal closed categories~\cite[Sections~1.2.3~and~4.7--4.8]{titoPhD} as part of their research on ``implicit automata''~\cite{IATLC,IATLC2,titoPhD} (cf.~\S\ref{sec:conclusion}).

 \textbf{Contributions of this paper.}
The main contribution of this paper is to propose a notion of single-use sets and functions, which extends the proposal from~\cite{stefanski-phd}, but which is rich enough to have function spaces. More formally, we propose a category, which describes single-use functions on orbit-finite sets, and we prove that this category is symmetric monoidal closed. 
The main idea is to follow the tradition of linear types, and to  distinguish two kinds of products, namely $X \times Y$ and $X \& Y$. Thanks to this distinction, the function space can be typed so that the appropriate operations on functions, namely application and currying, can be implemented in a single-use way. 

Our proposed category is strongly inspired by linear types, and the proof that it is symmetric monoidal closed uses a form of \emph{game semantics}\footnote{In this paper, the game semantics will only appear in the appendix, as they are part of our proofs rather than our main claims. That said, let us point to the lecture notes~\cite{abramsky2013semantics,Hyland1997} as references for the category of ``simple games'' upon which we build. For a recent survey of modern game semantics, see~\cite{ClairambaultHDR}.} -- a tool that we take from programming language theory. However, to the best of our knowledge, it is an original idea to have infinite but orbit-finite base type $\atoms$, and to observe that all constructions in game semantics are consistent with orbit-finiteness. We believe that the resulting category deserves further study, and that it is an interesting and non-trivial example of a category representing ``finite'' objects.

Along the way, we provide examples of how the category can be useful in automata theory. Our main example is converting a deterministic automaton into a monoid, but another example is converting a two-way deterministic automaton into a one-way deterministic automaton. As observed by Hines~\cite[Section 4]{hines2003categorical}, this construction works in traced categories, and we prove that our category is indeed traced. 

An important property of our construction is that it is generic. In fact, instead of having a set that is equipped with equality only, one could use apply the construction to any relational structure, e.g.~real arithmetic $(\mathbb R, +, \times, <)$, as long as the structure is given using relations and not functions. Under further assumptions on the structure, such as having a decidable first-order theory (which holds for real arithmetic) or being $\omega$-categorical (which does not hold for real arithmetic but does hold for the rational numbers with their linear order), further benefits in the resulting single-use category can be derived, such as having an emptiness algorithm for automata. 

Finally, as a minor contribution, we present an alternative solution for the problem of function spaces, which is  to use vector spaces of orbit-finite dimension. This is a minor contribution as far as the present paper is concerned, because the technical tools were developed already in~\cite{bojanczykKM21OrbitFiniteVector}, and the only contribution -- if any -- of this paper is one of perspective, namely framing it as a symmetric monoidal closed category. An advantage of the vector space category is its simplicity, and the fact that it is ``bigger'' in the following sense. The two solutions for function spaces discussed in this paper, namely the single-use solution and the vector space solution, sit on both sides of the classical category of orbit-finite sets, as witnessed by two faithful functors, one from the single-use category to the orbit-finite category, and one from the orbit-finite category to the vector space category. 
The generality of vector spaces comes at a price, though. As mentioned before, the single-use construction can be applied to any structure, and the orbit-finite benefits can be derived for all $\omega$-categorical structures (this is the standard assumption in the study of orbit-finiteness). In contrast, the vector space construction is more brittle, and it fails for certain $\omega$-categorical  structures such as the Rado graph.   Another disadvantage of the vector space category is that it is not traced with respect to the coproduct, unlike the single-use category, which precludes the applications to two-way automata. We do not make any claims about the superiority of the single-use category over the vector space category. 

\section{Sets with atoms}
We begin  with  a brief introduction to orbit-finite sets. For a more detailed treatment, see~\cite{bojanczyk_slightly2018}.

Fix for the rest of this paper a countably infinite set $\atoms$, whose elements will be called atoms.  We assume that this set has no other structure except for equality, which will mean that we will only be interested in notions which \emph{equivariant}, i.e.~invariant under renaming atoms. For example, $\atoms$ has only two equivariant subsets, namely the empty and full subsets. On the other hand, the set $\atoms^2$ has four equivariant subsets; this is because any subset containing (Eve, Eve) must contain all pairs on the diagonal, and any subset containing (Eve, John) must contain all pairs outside the diagonal.  In order to meaningfully speak about equivariant subsets, we must be able to have an action of atom renamings on the set, as formalized in the following definition. The finite support condition is a technical condition that ensures that the action is well-behaved; this condition dates back to the work of Fraenkel and Mostowski, and can be explained in the survey texts~\cite{PittsAM:nomsns} and~\cite{bojanczyk_slightly2018}.


%  and therefore an \emph{atom renaming} is defined to be any bijection of $\atoms$ with itself, which may move infinitely many atoms\footnote{Pitts, see~\cite[Definition 1.13]{PittsAM:nomsns}, allows moving only finitely many atoms, but the resulting theory is the same.}. 
%  Later in the paper, we  consider atoms with more structure.

% This paper is about sets with atoms, which are sets whose elements are constructed using atoms. Typically, we use sets that are disjoint unions of powers of the atoms, e.g.~$\atoms^4 + \atoms^3$. A set with atoms comes equipped with a natural action of atom renamings. For example, if $\pi$ is an atom renaming that swaps John and Eve, the set $X$ is $\atoms^4$, and the element is  $x=$ (John, Eve, John, Mark) then the result of applying $\pi$ to the element is $\pi(x)=$ (Eve, John, Eve, Mark).  A crucial part of the theory is that we only consider elements which use finitely many atoms, which is formalized using the finite support condition  below.

\begin{definition}[Set with atoms]
    A \emph{set with atoms} is a set $X$, equipped with an action of the group of atom renamings, subject to the following \emph{finite support condition}: for every $x \in X$ there is a finite set of atoms, such that if an atom renaming $\pi$ fixes all atoms in the set, then it also fixes $x$.
\end{definition}

The idea is that a set with atoms is any kind of object that deals with atoms, such as the set $\atoms^*$ of all words over the alphabet $\atoms$, or the family of finite subsets of $\atoms$. Among such objects, we will be interested in those which are ``finite''. This will be formalized by  saying that there are  finitely many orbits, as described below.
Define the \emph{orbit} of an element $x$ in a set with atoms to be the elements that can be obtained from $x$ by applying some atom renaming. For example, in the set $\atoms^2$,  the orbit of (John, Eve)   contains  (Mark, John), but it does not contain (John, John). The orbits form a partition of a set with atoms. 


\begin{definition}[Orbit-finite set]   A set with atoms is called \emph{orbit-finite} if it has finitely many orbits. 
\end{definition}

A typical example of an orbit-finite set is $\atoms^4$, or more general any polynomial expression such as $\atoms^4 + \atoms^3 + \atoms^3 + 1$. Here, $1$ represents the set of zero-length sequences; this set has a unique element which is its own orbit.  For  example,  $\atoms^3$ has five orbits, because there are five possible ways of choosing a pattern of equalities in a sequence of three names. On the other hand,  $\atoms^*$ has infinitely many orbits, since sequences of different lengths are necessarily in different orbits.   The family of finite subsets of $\atoms$  is also  not orbit-finite, because subsets of different size are in different orbits. The full powerset $\powerset \atoms$ is not even a legitimate object in our setting, because some of its elements, i.e.~some subsets of $\atoms$, violate the finite support condition. 

% \begin{example}[Polynomial orbit-finite sets]\label{ex:polynomial-orbit-finite-sets} 
%   A \emph{polynomial orbit-finite set} is a finite disjoint unions of finite powers of the atoms, i.e.~a set of the form 
%     \begin{align*}
%     \atoms^{n_1} + \cdots + \atoms^{n_k} \qquad \text{for some }n_1,\ldots,n_k \in \set{0,1,\ldots}.
%     \end{align*}
%     When the power is zero, i.e.~in the set $\atoms^0$, there is only one element  that represents the empty tuple, and the action of atom renamings is trivial. We sometimes write $1$ for the set $\atoms^0$, and we write () for its unique element, since it represents the empty tuple. Some orbit-finite sets are not polynomial, here are two examples:
%     \begin{align*}
%     \myunderbrace{
%         \setbuild{(a,b)}{$a \neq b \in \atoms$}
%     }{ non-repeating pairs}
%     \qquad 
%     \myunderbrace{
%     \setbuild{\set{a,b}}{$a \neq b \in \atoms$}
%     }{ unordered pairs} 
%     \end{align*}  
        
% \end{example}

\subsection{Finiteness of function spaces}
\label{sec:orbit-finite-function-spaces}
As mentioned above, orbit-finite sets can be seen as  a certain generalisation of finite sets. They allow some, but not all, operations that can usually be done on finite sets. For example, orbit-finite sets are closed under disjoint unions $X + Y$ and products $X \times Y$.  Another good property is that an orbit-finite set has only finitely many equivariant subsets (an equivariant subset is one that is invariant under the action of atom permutations). This is because an equivariant subset is a union of some of the finitely many orbits. This accounts for some of the good computational properties of orbit-finite sets. For example, nonemptiness is decidable for orbit-finite automata (see below), because the state space can be searched orbit by orbit.
However, orbit-finite sets do not have orbit-finite function spaces, as explained in the following example. 

\begin{example}
    Consider the space of functions of type $\atoms \to \atoms$. What are the legitimate functions? One choice is that we only allow the equivariant functions. Under this choice, there is only one possible function,  namely the identity function. However, as we will explain below, the more appropriate choice is the class of finitely supported functions, i.e.~those that are invariant under all atom renamings that fix some finite set of atoms that depends only on the function. For example,  the  function
    \begin{align*}
    f(a) = \begin{cases}
        \text{Mark} & \text{if $a \in \set{\text{John, Eve, Bill}}$} \\
        a & \text{otherwise}
    \end{cases}
    \end{align*}
    is finitely supported, because invariant under all atom renamings that fix Mark, John, Eve and Bill. The space of finitely supported functions will be called the \emph{finitely supported function space}. This space is equipped with an action of atom renamings. For example if $\pi$ is the atom renaming that swaps Mark with Adam, then applying it to the function $f$ defined above gives the function $\pi(f)$ that has the same definition (or source code, if a programming intuition is to be followed), except that Mark is used instead of Adam.
    In the case of $\atoms \to \atoms$, the finitely supported function space is not orbit-finite. Indeed, the output Mark could be conditioned on the input belonging to some very long finite list of exceptional values, and extending the list of exceptional values will give us a function in a different orbit.
\end{example}

In the above example, we explained how the finitely supported function space might not be orbit-finite. On the other hand, the equivariant function space will always be literally finite, if the input and output types are orbit-finite, because it will be an equivariant subset of the product $X \times Y$. So why do we insist on the finitely supported function space? A more principled reason will appear later in the paper, where we discuss symmetric monoidal closed categories, and where the finitely supported function space will turn out to be the right one. However, we can already give a simple reason, which appeals to automata theory, namely converting an automaton into a monoid. As we have seen in Example~\ref{ex:first-letter-repeats-monoid}, when converting an automaton to a monoid, we will want to use partially applied transition functions, and such functions will be finitely supported but not equivariant. 


% \begin{example}[From automata to monoids]\label{ex:automata-to-monoids}
%     Consider the following two models for representing equivariant languages $L \subseteq \Sigma^*$ over some orbit-finite alphabet $\Sigma$.  The first model is deterministic orbit-finite automata~\cite[Section 3]{bojanczykAutomataTheoryNominal2014}. This model is defined as usual in language theory, except that the states and input alphabet are orbit-finite sets, and the remaining automaton structure (initial state, transition function, accepting set) is equivariant. The second model is orbit-finite monoids~\cite[Section 3]{bojanczykNominalMonoids2013}. This is defined as usual in language theory, except that the input alphabet and  the underlying set of the monoid are orbit-finite, and the remaining structure (the monoid homomorphism, the multiplication operation, and the accepting set) is equivariant.

 
% In the usual setting of finite sets (not orbit-finite sets), the two models have the same expressive power. In the proof, from an automaton with states $Q$ one constructs a monoid that describes functions $Q \to Q$. This, however, does not work in the orbit-finite setting. Note that the appropriate space is the finitely supported one, since we the monoid should consist of state transformations $q \mapsto \delta(q,w)$ that arise from input words $w$, and such functions are no longer equivariant, but they are finitely supported (by whatever supports the input word).  In fact, not only the standard proof fails, but the result is simply false.  For example, the language 
% \begin{align*}
% \setbuild{ w \in \atoms^+}{the first letter in $w$ appears at least twice}
% \end{align*}
% is recognized by a deterministic orbit-finite automaton, but not by an orbit-finite monoid. 
% \end{example}

The lack of function spaces is the problem that we address in this paper. The rest of this paper is devoted to two solutions of this problem. In each of the two solutions, we modify the notion of orbit-finite sets, by either restricting it or generalizing it, in a way that recovers function spaces. 





\section{Single-use sets and functions}
\label{sec:single-use-sets}
We now turn to our first solution for the problem of orbit-finite function spaces.  Our solution builds on the idea from~\cite[Section 2.2]{stefanski-phd}, which is to consider only functions that are single-use. We describe this idea in Section~\ref{sec:single-use-functions-over-polynomial-orbit-finite-sets}, and we show how it almost, but not quite, achieves function spaces. Then, in the rest of this section, we show how function spaces can be recovered by using a more refined type system. 

\subsection{Single-use functions over polynomial orbit-finite sets}
\label{sec:single-use-functions-over-polynomial-orbit-finite-sets}

In the introduction, we have already given an intuitive description behind the single-use functions; these are functions that destroy the argument after any use of it, such as comparing it to a constant. We now give a more formal definition.

We do not define the single-use functions on all orbit-finite sets, but only a syntactically defined fragment, namely the \emph{polynomial orbit-finite sets}, which are sets that can be generated from $1$ and $\atoms$ using  
 products $\times$ and disjoint unions $+$. Therefore, we will allow orbit-finite sets like $1 + \atoms^2$, but we will not allow orbit-finite sets like 
the set of non-repeating pairs
$\setbuild{(a,b)}{$a \neq b \in \atoms$}$ or the set of unordered pairs $\setbuild{\set{a,b}}{$a \neq b \in \atoms$}$. It is an open problem to find a satisfactory definition of single-use functions on all orbit-finite sets. (A simple hack is to use a quotienting construction, similar to \Cref{sec:quotient-category}, but what we would really like to do is to identify some extra structure in a set, possibly an action of some yet unknown group or semigroup, which enables us to speak about single-use functions.)

Consider two polynomial orbit-finite sets $X$ and $Y$. To define which functions $X \to Y$ are single-use, we use an inductive definition. We begin with certain functions that are considered single-use, such as the equality test of type $\atoms \times \atoms \to 1 + 1$. These functions are called the \emph{prime functions}, and their full list is given in Figure~\ref{fig:prime-morphisms-without-with}. Next, we combine the prime functions into new ones using three combinators. The first, and most important, combinator is  function composition. Then, we have two combinators for the two type constructors: if two functions $f_1 : X_1 \to Y_1$ and $f_2 : X_2 \to Y_2$ are single-use, then the same is true for:
\begin{align*}
    f_1 \times f_2 : X_1 \times X_2 \to Y_1 \times Y_2
    \qquad &
    f_1 + f_2 : X_1 + X_2 \to Y_1 + Y_2 \\
    \scriptstyle  (x_1,x_2) \mapsto (f_1(x_1),f_2(x_2)) 
    \qquad &
    {\scriptstyle \text{left}(x_1) \mapsto \text{left}(f_1(x_1)) 
        \quad 
        \text{right}(x_2) \mapsto \text{right}(f_2(x_2)) }.
\end{align*}

Crucially, the list of prime single-use functions does not contain the copying function $a \in \atoms \mapsto (a,a) \in \atoms^2$. Therefore, an alternative name for the single-use functions is \emph{copyless}. If we added copying, then we would get all finitely supported functions~\cite[Lemma 23]{stefanski-phd}.

\begin{table}[h!]
    \centering
    \begin{tabular}{lll}
        \textbf{Function} & \textbf{Type} & \textbf{Definition} \\ \\
        \emph{Functions about $\atoms$} \\
        equality test & $\atoms \times \atoms \to 1 + 1$ & $a, b \mapsto \text{if } a = b \text{ then true else false}$ \\
        constant $a$ & $1 \to \atoms$ & $x \mapsto a$ \\
        identity & $\atoms \to \atoms$ & $x \mapsto x$ \\
        \\
        \emph{Functions about \(\times\)} \\
        commutativity of $\times$ & $X \times Y \to Y \times X$ & $x \times y \mapsto y \times x$ \\
        first projection & $X \times Y \to X$ & $x \times y \mapsto x$ \\
        second projection & $X \times Y \to Y$ & $x \times y \mapsto y$ \\
        append 1 & $X \to X \times 1$ & $x \mapsto x \times ()$ \\
        associativity of $\times$ & $(X \times Y) \times Z \to X \times (Y \times Z)$ & $(x \times y) \times z \mapsto x \times (y \times z)$ \\ \\
        \emph{Functions about \(+\)} \\
        first co-projection & $X \to X + Y$ & $x \mapsto \text{left}(x)$ \\
        second co-projection & $Y \to X + Y$ & $y \mapsto \text{right}(y)$ \\
        co-diagonal & $X + X \to X$ & $\left\{\begin{tabular}{l}
            $\text{left}(x) \mapsto x$\\
            $\text{right}(x) \mapsto x$
            \end{tabular}\right.$ \\
        commutativity of $+$ & $X + Y \to Y + X$ & $\left\{\begin{tabular}{l}
        $\textrm{left}(x) \mapsto \textrm{right}(x)$\\
        $\textrm{right}(y) \mapsto \textrm{left}(y)$
        \end{tabular}\right.$ \\
        associativity of $+$ & $(X + Y) + Z \to X + (Y + Z)$ & $\left\{
        \begin{tabular}{l}
        $\text{left}(\text{left}(x)) \mapsto \text{left}(x)$\\
        $\text{left}(\text{right}(y)) \mapsto \text{right}(\text{left}(y))$\\
        $\text{right}(z)\mapsto \text{right}(\text{right}(z))$
        \end{tabular}\right.$ \\
        \\
        \emph{Distributivity}
        \\
        $+$ distributes over $\times$ & $X \times (Y + Z) \to (X \times Y) + (X \times Z)$ & $\left\{\begin{tabular}{l}
            $x \times (\text{left}(y)) \mapsto \text{left}(x \times y)$\\
            $x \times (\text{right}(z)) \mapsto \text{right}(x \times z)$
        \end{tabular}\right.$ \\
        \\
    \end{tabular}
    \caption{The prime single-use functions for polynomial orbit-finite sets $X, Y$ and $Z$.}
    \label{fig:prime-morphisms-without-with}
\end{table}





\begin{example}\label{ex:six-compositions}
    Consider function of type $\atoms^3 \to \atoms$ which inputs a triple $(a,b,c)$ of atoms and returns $a$ if $c$ is equal to Mark, and $b$ otherwise. This function is a single-use function. It is obtained by composing the six functions listed below:
\begin{center}
    \begin{tabular}{ll}
        Function & Type after function \\
        \hline
        Append 1. & $ \atoms \times \atoms \times \atoms \times 1$ \\
        Replace added $1$ with Mark using the constant function. & $\atoms \times \atoms \times \atoms \times \atoms$ \\
        Apply the equality test to the last two components. & $ \atoms \times \atoms \times (1+1)$ \\
        Distribute. & $ \atoms \times \atoms \times 1 +   \atoms \times \atoms \times 1$ \\
        Project to first and second components, respectively. & $\atoms + \atoms$ \\
        Co-diagonal & $\atoms$ 
    \end{tabular}
\end{center}
To justify this description, one should also show that the six functions are single-use. Three of the functions, namely append 1, distributivity and co-diagonal are prime functions. The other three are obtained by combining prime functions using the combinators. For example, the equality test is paired, using the combinator for $\times$, with the identity on the remaining two atoms. \exampleend
\end{example}

The design goal of the single-use restriction is to have orbit-finite function spaces. The rough idea is that a single-use function can only use a bounded number of atoms in its source code, which guarantees orbit-finiteness of the function space. 

\begin{example}\label{ex:first-single-use-function-space}
    Consider function of type $\atoms \to 1 + 1$, which can be seen as subsets of the atoms, with $1+1$ representing the Booleans. We will consider two function spaces: the larger space of all finitely supported functions, and the smaller space of single-use functions.
    
    A function in the larger space is any finitely supported subset of the atoms; such subsets are the same as the finite and co-finite subsets. Therefore, the larger function space admits an equivariant bijection with a disjoint union of two copies of the finite powerset $\powerset_{\text{fin}} \atoms$, in particular it is not orbit-finite.
    
    Consider now the smaller single-use function space. There are four possible functions of this kind: (a) always return true; (b) always return false; (c) check for equality with some fixed atom $a$; (d) check for disequality ($\neq$) with some fixed atom $a$.  Therefore,  the set of single-use functions  admits an equivariant bijection with the orbit-finite set $1 + 1 + \atoms + \atoms$.
    \exampleend
\end{example}


The above example shows that the space of single-use functions of some type $X \to Y$ is orbit-finite, and in fact it can be described using a polynomial orbit-finite set. This is true for every choice of polynomial orbit-finite sets $X$ and $Y$, as proved in~\cite[Theorem 5]{stefanski-phd}, and illustrated in the following example. 

\begin{example}\label{ex:decision-tree-types} Assume that the input type $X$ is some power of the atoms $\atoms^k$, and the output type $Y$ does not use atoms, e.g.~it is~$Y = 1 +1$. The assumption on the input type can be made without loss of generality using distributivity, while the assumption on the output type is a proper restriction, but it will allow us to skip some technical details of the general construction while retaining the important intuitions. 
We describe below a type that represents all single-use functions from $\atoms^k$ to $Y$; we shall denote it by $\atoms^k \Rightarrow Y$. Note that $\Rightarrow$ is \emph{not} a primitive type constructor in our grammar of types; it is a notation that stands for the inductive construction below.

This type is defined by  induction on $k$. In the base case of $k=0$ we simply need to give a value from the output type, and therefore $\atoms^0 \Rightarrow Y$ is the same as $Y$. Consider now the induction step of $k > 0$.  
    We observe that a single-use function that inputs $\atoms^k$ must begin with some equality test, and then continue with one of two single-use functions that have fewer arguments (one for the case when the equality test returns true, and one for the other case). This observation leads to the following definition of the type $\atoms^k \Rightarrow Y$:
\begin{align*}
\myunderbrace{ \coprod_{i \in \set{1,\ldots,k}} 
    \atoms \times (\atoms^{k-1} \Rightarrow Y) \times (\atoms^{k-1} \Rightarrow Y)
 }{starts by comparing $i$-th  \\
 coordinate to some constant}  \quad + \quad 
\myunderbrace{\coprod_{i, j \in \set{1,\ldots,k}} (\atoms^{k-2} \to Y) \times (\atoms^{k-2} \to Y)}{
    starts by comparing the \\ 
    $i$-th and $j$-th coordinates}.
\end{align*}
Note that the above representation of the function space is not necessarily unique, i.e.~the same function can be represented in several different ways. For example, the order in which equality tests are performed will matter for the representation, but might not matter for the function. This is not something that we worry about, and we will use function spaces with non-unique representations in the paper, see also \Cref{sec:quotient-category} for how we deal with non-uniqueness. \exampleend
\end{example}


\subparagraph{Problem with currying.}
Unfortunately, the proposal illustrated in Example~\ref{ex:decision-tree-types} and described in more detail in~\cite{stefanski-phd}  does not give a satisfactory solution to the problem of function spaces. The problem is that the set of representations $X \Rightarrow Y$ should  also support operations on functions. More specifically, we should be able to indicate single-use operations which do the following:
\begin{description}
    \item[evaluation:] a single-use function from $(X \Rightarrow Y) \times X$ to $Y$ which inputs a representation of a function and applies it to an argument;
    \item[composition:] a function from $(X \Rightarrow Y) \times (Y \Rightarrow Z)$ to $(X \Rightarrow Z)$ which inputs the representations of two functions and returns a representation of their composition.
    \item[currying:] for each single-use function from  $ X \times Y $ to $Z$, there should be a single-use function from $X$ to $Y \Rightarrow Z$ which inputs a first argument and returns a representation of the  partially applied function;
\end{description}
Only in the presence of all of these operations can we speak of a function space, and the corresponding category can be called closed. (Composition can be obtained through evaluation and currying, so the essential operations are evaluation and currying.) The following example shows that the currying operation is not single-use, and therefore the space of single-use functions as defined in this section is not closed.


\begin{example}\label{ex:currying-not-single-use}
    Consider  the single-use function
    \begin{align*}
    f : \atoms \times \atoms \to 1 + 1 \qquad (a,b)  
    \mapsto \begin{cases}
        \text{result of test $a = $ Mark} & \text{if } b = \text{Eve} \\
        \text{result of test $a = $ John} & \text{otherwise}.
        \end{cases}
    \end{align*}
The currying of this function, is a new function which  maps a first argument $a \in \atoms$ to the partially applied function $f(a,\_)$. This currying is
\begin{align*}
    a \mapsto \begin{cases}
        b \mapsto b = \text{Eve} & \text{if } a = \text{Mark} \\
        b \mapsto b \neq \text{Eve} & \text{if } a = \text{John} \\
        b \mapsto \text{false} & \text{otherwise}
        \end{cases}
\end{align*}
Recall that in Example~\ref{ex:first-single-use-function-space} we showed that the space of single-use functions of type $\atoms \to 1 + 1$ can be represented as $1 + 1 + \atoms + \atoms$. If we use this representation,  then the currying of the function $f$  is not single-use, because we need to compare the input atom $a$ to two constants, Mark and John. If we use the representation from Example~\ref{ex:decision-tree-types}, then the corresponding type will be $\atoms \otimes (1 + 1)^2$, but the problems with currying will persist. \exampleend
\end{example}

For similar reasons, 
the function space, we proposed above, will also not support function composition, which means that it cannot be used to convert automata into single-use monoids, as we would like to do, since the resulting monoid would need to use function composition as its monoid operation.\footnote{This problem is solved in~\cite{bojanczykstefanski2020} and~\cite{stefanski-phd} in a different way, namely by showing that every orbit-finite monoid necessarily divides a single-use monoid, using a  Krohn-Rhodes construction. However, this construction is difficult and delicate, in particular it does not work for atoms that have more structure than equality alone. In contrast, the proposal that we give in this paper works for other kinds of atoms, as discussed in Section~\ref{sec:beyond-equality}. }

To solve the problems above, we will introduce a more refined type system, which is based on linear types.  The main idea is to pay more attention to type  in Example~\ref{ex:decision-tree-types}. In that definition, we describe a single-use function by specifying the first equality test that it makes, and then giving two descriptions of the functions that will be used in each of the two possible outcomes of the equality test. The main observation is that these two outcomes are mutually exclusive, and therefore we intend to use only one of the two descriptions. For this reason, we will use a  type constructor $\&$ that comes from linear logic. The intended meaning is  that an object of type $X \& Y$ consists of two objects, but with the ability to use only one of them. Since linear logic uses $\otimes$  for the  product that we have so far denoted by $\times$, we will also follow that convention. Using these two kinds of products, the appropriate type for Example~\ref{ex:decision-tree-types} will now become:
\begin{align*}
\coprod_{i \in \set{1,\ldots,k}} 
        \atoms \otimes ((\atoms^{k-1} \Rightarrow Y) \& (\atoms^{k-1} \Rightarrow Y))
          \quad + \quad 
    \coprod_{i, j \in \set{1,\ldots,k}} (\atoms^{k-2} \to Y) \& (\atoms^{k-2} \to Y).
    \end{align*}
Under this definition, the problems from Example~\ref{ex:currying-not-single-use} will be solved, at least for the particular type considered in that example. However, by introducing a new type constructor, we will have to redefine the single-use functions, and then we will have to give a representation of functions that allow this new type constructor, without incurring the need to add any other new type constructors. This is the subject of the next section.

\subsection{Linear types and single-use functions on them}
\label{sec:linear-types-and-single-use-functions}
As mentioned above, to solve the problems with single-use function spaces, we will consider a type system with two kinds of products, as in the following definition.
\begin{definition}[Linear types]\label{def:datatypes}
    A \emph{linear type} is any expression constructed from the atomic types $1$ and $\atoms$ using three\footnote{We set up our type system without using the multiplicative disjunction $\rotatebox[origin=c]{180}{\&}$ of linear logic -- morally, we take our inspiration from intuitionistic linear logic, rather than classical linear logic.} binary type constructors $+, \&$ and $\otimes$.
\end{definition}
In our linear types, it is only the products that are differentiated, while  $+$ comes in only one version. 
    Here is the intuitive explanation of the difference between the two kinds of products, following Girard~\cite[p.2]{girard1995advances}. Having a pair $x \otimes y$ is like having the ability of using both components $x$ and $y$. On the other hand, having a pair $x \& y$ is like having the ability to use one of the two components, at our choice, but not both at once. For example, the input type of the equality test will be $\atoms \otimes \atoms$ not $\atoms \& \atoms$, since the test will need to consume both arguments. This intuition can only go so far; for example, it is not entirely clear what ``our choice'' means. We revisit this intuition in  the appendix, where game semantics will be used to indicate who makes which choices. 



We think of each linear type $X$ as representing a set $\sem X$, as defined below:
\begin{align*}
    \sem{1} = 1
\quad 
\sem{\atoms} = \atoms 
\quad 
\sem{X+Y} = \sem X + \sem Y 
\quad 
\sem{X \otimes Y} =
\sem{X \& Y} = \sem X \times \sem Y.
\end{align*}
All sets that arise in this way will be polynomial orbit-finite sets.
Note that the two kinds of product represent the same set, namely the set of pairs in the usual set-theoretic sense. 
However, the two type constructors will be  different, because different functions will be allowed to operate on them. As the expression goes, ``the proof of the pudding is in the eating''; in this case the pudding is the types and the eating is the functions.  

As we did in Section~\ref{sec:single-use-functions-over-polynomial-orbit-finite-sets}, the single-use functions will be defined in terms of prime functions and combinators. The combinators are the same, except that instead of $f_1 \times f_2$ we now have two ways of pairing functions, using $\otimes$ and $\&$. The prime functions are inherited from the previous system, with $\times$ understood as $\otimes$, together with a few new functions for $\&$, as described in Table~\ref{fig:prime-morphisms-with-with}. This is summarized in the following definition.



\begin{definition}[Single-use functions] The class of single-use functions is the least class of functions with the following properties:
    \begin{enumerate}
        \item It contains the functions from Tables~\ref{fig:prime-morphisms-without-with} and~\ref{fig:prime-morphisms-with-with}, with $\times$ in Table~\ref{fig:prime-morphisms-without-with} understood as $\otimes$;
        \item It is closed under composition, as well as under combining functions using  $+$, $\otimes$ and $\&$. 
    \end{enumerate}
\end{definition}




\begin{table}[h!]
    \centering
    \begin{tabular}{lll}
        \textbf{Function} & \textbf{Type} & \textbf{Definition} \\ \\
        diagonal  & $X \to X \& X$ & $x \mapsto x \& x$ \\
        first projection & $X \& Y \to X$ & $x \& y \mapsto x$ \\
        second projection & $X \& Y \to Y$ & $x \& y \mapsto y$ \\
        $\&$ distributes over $\otimes$ & $X \otimes (Y \& Z) \to (X \otimes Y) \& (X \otimes Z)$ & $x \otimes (y \& z) \mapsto (x \otimes y) \& (x \otimes z)$ \\
        $\&$ distributes over $+$ & $X + (Y \& Z) \to (X \& Y) + (X \& Z)$ & $\left\{\begin{tabular}{l}
        $x \& \text{left}(y) \mapsto \text{left}(x \& y)$\\
        $x \& \text{right}(z) \mapsto \text{right}(x \& z)$
        \end{tabular}\right.$ \\ \\ 
    \end{tabular}
    \caption{Prime single-use functions that involve $\&$.}
    \label{fig:prime-morphisms-with-with}
\end{table}


   


Formally speaking, a single-use function consists of an input linear type $X$, an output linear type $Y$, and a function between the sets $\sem X$ and $\sem Y$ that is generated using the prime functions and combinators from the above definition.  As was the case in Section~\ref{sec:single-use-functions-over-polynomial-orbit-finite-sets}, all single-use functions are  finitely supported.  Therefore, one can think of the single-use functions of type $X \to Y$ as being a subset of the set of all finitely supported functions from $\sem X$ to $\sem Y$. This subset is strict: as we will see, the space of single-use functions will be orbit-finite, unlike the space of all finitely supported functions. We will be thinking of the single-use functions as a category.




\begin{definition}[Category of single-use sets]\label{def:single-use-category}
    The category of single-use sets is:
    \begin{enumerate}
        \item The objects are linear types, as per Definition~\ref{def:datatypes}.
        \item Morphisms between types $X$ and $Y$ are partial single-use functions from $\sem X$ to $\sem Y$.
    \end{enumerate}
\end{definition}

In the very definition of the above category, there is a faithful functor to the category of  orbit-finite sets with finitely supported functions. This functor maps objects $X$ to their underlying sets $\sem X$, which are orbit-finite sets, and it maps morphisms to the corresponding functions. The functions seen to be finitely supported, and the functor is faithful  by definition, since the morphisms in Definition~\ref{def:single-use-category} are defined to be single-use functions. 



% We use the name \emph{single-use morphisms} for morphisms of the above category. A single-use morphism is the same as a single-use function, together with the information about the types between which it is a morphism. For example, the idenitity function on $\sem{\atoms \& \atoms}$ is the same as the identity function on $\sem{\atoms \otimes \atoms}$, because the two underlying sets are the same, but the corresponding morphisms are different. 
% From now on, instead of talking about single-use functions of type $\sem X \to \sem Y$, we will speak about single-use morphisms of type $X \to Y$.



 
The main technical result of this paper is that the category of single-use sets has function spaces, as stated in the following theorem.  The appropriate product will be $\otimes$, and not $\&$. Since the Cartesian product in our category is $\&$ and not $\otimes$, this means that the result we are targeting is symmetric monoidal closed with respect to $\otimes$, and not Cartesian closed. (The same situation will happen  for vector spaces later in this paper.)  Our theorem stops a bit short of saying that the category is monoidal closed, since several different elements of the function space might represent the same function. This non-uniqueness of representation can be overcome  by quotienting the function space  by an equivalence relation that identifies two elements if they represent the same single-use function; this is explained in Appendix~\ref{sec:quotient-category}. For now, we stick with non-uniqueness of representation, and we state the theorem as follows.

\begin{theorem}\label{thm:single-use-closed}
    Let $V$ and $W$ be objects (i.e.~linear types). There exists an object, denoted by  $\funspace V W$, and a morphism (i.e.~a partial single-use function)
    \begin{align*}
    eval : (\funspace V W) \otimes V \to  W
    \end{align*}
    with the following property. For every morphism
    \begin{align*}
    f : {X \otimes V} \to  W
    \end{align*}
    there is a (not necessarily unique) morphism
    \begin{align*}
    h :  X \to (\funspace V W)
    \end{align*}
    such that the following diagram commutes:
    \[
    \begin{tikzcd}
    X \otimes V 
    \arrow[r,"h \otimes id"]
    \arrow[dr,"f"']
    &
    (\funspace V W) \otimes V
    \arrow[d,"eval"] \\
    &
    W
    \end{tikzcd}
    \]
\end{theorem}

The above theorem is the main technical contribution of this paper, and its proof is presented in Appendix~\ref{sec:game-semantics}. The difficulty in the proof is finding a representation of the single-use  functions that is rich enough to capture all functions, but simple enough to be described by a linear type (in particular, the corresponding set will be orbit-finite). In  Section~\ref{sec:single-use-functions-over-polynomial-orbit-finite-sets}, when the types did not have $\&$, we could pull off a relatively simple construction, which was possible mainly due to the strong distributivity rules that allowed converting each type into a normal like $\atoms^{n_1} + \cdots + \atoms^{n_\ell}$. In the presence of $\&$, the distributivity rules are not as strong, and the way in which a single-use program can interact with its input is rather subtle. 
Our solution, and the technical core of this paper, is to use game semantics, appropriately extended to describe the type $\atoms$ and the operations on it that are allowed. This solution is presented entirely in the appendix.










% \input{programs}
\section{Traced categories and two-way automata}
\label{sec:two-way-automata}

In this section, we show that the single-use category is traced with respect to co-product, and we use this to fact to model deterministic two-way automata.
We begin by describing the usual trace construction in the category of sets and partial functions. 
Consider some partial function $
f : A + X \to B + X.$
For a number $k \in \set{0,1,\ldots}$ we can define a partial function of type $A \to B$, which is called the \emph{$k$-th iteration}, as  follows by induction on $k$. The $0$-th iteration is  completely undefined. For $k > 0$, the $k$-th iteration is defined as follows. First we apply $f$ to the input. If the output is undefined or from  $B$, then that is the final output of the $k$-th iteration. Otherwise, if the output is from $X$, then we apply the $(k-1)$-st iteration and return that output (which may be undefined). It is easy to see that the iterations are ordered by inclusion, i.e.~when viewed as binary relations they form an increasing chain. The limit of this chain, which is a partial function from $A$ to $B$ is defined to be the \emph{trace of $f$}. 

We will now show that the traces exist also in the single-use category, even if we take the general variant from Section~\ref{sec:beyond-equality} that arises from some relational structure. However, we will need to make a certain assumption on this structure. Call a structure $\atoms$  \emph{oligomorphic} if for every $k \in \set{0,1,\ldots}$, there  are finitely many orbits in $\atoms^k$ under the action of the automorphism group of $\atoms$.
Oligomorphism is the standard assumption in the theory of sets with atoms~\cite{bojanczyk_slightly2018}. In particular, ensures that the notion of orbit-finite set is meaningful. Examples of oligomorphic structures include: the atoms with equality only, the rational numbers with their linear order, and the Rado graph. Nonexamples include: the integers with their linear order, Presburger arithmetic, and the real field. 



The main observation is that for single-use functions over an oligomorphic structure, the trace is achieved in finitely many steps.
\begin{lemma}Let $\atoms$ be an oligomorphic structure, let $A,B,X$ be linear types, and let 
    \begin{align*}
        f : \sem{A + X} \to \sem{B + X}
        \end{align*}
        be a single-use function over $\atoms$. There is some $k$ such that the trace of $f$ is equal to $f^{(k)}$.
\end{lemma}
\begin{proof}
    We only need the weaker assumption, which is that the function $f$ is finitely supported. Consider some set of atoms that supports $f$. This same set of atoms will also support all functions $f^{(k)}$, and it will also support their domains. In particular, the domains of these functions will form a decreasing sequence of subsets of $\sem{A + X}$, with all subsets having the same support. Since an orbit-finite set can have finitely many subsets with a given support, this sequence must terminate in finitely many steps.
\end{proof}

Since the trace is achieved in finitely many steps, and it is easily seen to be constructed using operations on functions that preserve the single-use condition, the trace is also single-use. (In this section, we use single-use partial functions, which are defined to be single-use functions of type $X \to Y +1$, where $1$ represents the undefined value.) Therefore, the single-use partial functions over an oligomorphic structure form what is called a \emph{traced category}, with respect to $+$. 



We now use the trace construction to model deterministic two-way automata. (The idea that traced categories are a natural setting for two-way automata was already noted in~\cite{hines2003categorical}.) We define a deterministic two-way automaton in the same way as a deterministic one-way automaton, except that the transition function now is a partial function from $\Sigma \otimes (Q + Q)$ to $Q + Q + 1$. 
The two copies of $Q$ represent entering the letter from the left or right, and the $1$ in the output represents acceptance.  The function is partial, and thus if we want to model it as a complete function, then there will be another $1$ in its output, representing rejection.

\begin{theorem}\label{thm:two-way-automata}
    Let $\atoms$ be an oligomorphic structure with a decidable first-order theory. Then the emptiness problem is decidable for single-use deterministic  two-way automata over $\atoms$.
\end{theorem}



\section{Beyond equality}
\label{sec:beyond-equality}

So far, we have only studied the case when the atoms are equipped with equality only. Consider now a more general case: let $\atoms$ be any relational structure, i.e.~any set equipped with relations (but not functions). For example, we could use the rational numbers with their linear order. Another example would be Presburger arithmetic, i.e.~the intergers $\Nat$ together with addition $+$. Since we want to have relations only, we think of addition as a ternary relation $x + y =z$. 

The construction of the single-use category from Section~\ref{sec:single-use-sets} is generic enough so that it be generalized to any relational structure, and not just atoms with equality. (In fact, there is also a suitable generalization for function symbols, but we do not prove anything about it, so we do not talk about it.)

\begin{definition}[Single-use functions over a relational structure]\label{def:single-use-category-relational-structures}
    Let $\atoms$ be a relational structure. The \emph{single-use category over $\atoms$} is defined in the same way as in Definition~\ref{def:single-use-category}, except that the list of prime functions is extended with one prime function 
    \begin{align*}
        \myunderbrace{\atoms \otimes \cdots \otimes \atoms}{$n$ times} \to 1+1
        \end{align*}
        for every $n$-ary relation in the vocabulary of $\atoms$. 
\end{definition}


Not only is the definition of the category generic, but the same is true for  the proof that function spaces exist. The generalized version of this theorem, stated below, makes only a relatively tame assumption, namely that the vocabulary has finitely many relations of each arity. For example, such vocabularies will arise if we start with some finite vocabulary, and then enrich it by creating a new relation symbol for every quantifier-free formula (modulo equivalence of quantifier-free formulas). In the following theorem, when speaking of a symmetric monoidal closed category, we mean that properties stated in the conclusion of Theorem~\ref{thm:single-use-closed}, in particular we do not require unique currying.



\begin{theorem}\label{thm:single-use-closed-relational-structures}
    Consider a relational structure $\atoms$, in which for every $k \in \set{0,1,\ldots}$ there are finitely many relations of arity $k$. Then the single-use category over $\atoms$ is symmetric monoidal closed, with respect to the tensor product $\otimes$.
\end{theorem}

The theorem is proved in the same way as Theorem~\ref{thm:single-use-closed}.  The assumption on the vocabulary is used to ensure that in the corresponding game semantics, there are finitely available moves in any given moment, because the vocabulary can only be queried up to the maximal number of atoms in the input, due to the single-use restriction.

Note that this  theorem can be applied to any relational structure, including undecidable ones. Clearly there must be some benefit from assuming that the structure has a decidable first-order theory, which means that there is an algorithm which checks if a first-order sentence is true in the structure. 
The benefit is that we can check if two morphisms are equal, as expressed in the following theorem. 

\begin{theorem}
    Consider a relational structure $\atoms$, in which for every $k \in \set{0,1,\ldots}$ there are finitely many relations of arity $k$. If this structure has a decidable first-order theory, then there is an algorithm for the following problem:
    \begin{itemize}
        \item {\bf Input.} Two morphisms, described by expressions using prime functions and combinators.
        \item {\bf Question.} Are they the same morphism?
    \end{itemize}
\end{theorem}




The above theorem gives us a reasonable category of single-use functions over a relational structure with a decidable first-order theory. This applies to structures such as Presburger arithmetic, or the real field $(\mathbb R, +, \cdot, \leq)$ with the field operations viewed as ternary relations. 
However, a decidable first-order theory is not the only property needed to ensure that the category is appropriate to automata. If we want to model automata and their decision procedures, we will also need to execute certain fix-point algorithms, as explained in~\cite{bojanczyk_slightly2018}. This will be illustrated in the next section, where we prove that emptiness for automata, including two-way automata, is decidable under an additional assumption on the structure called \emph{oligomorphism}. This assumption is the standard assumption used to ensure that orbit-finiteness is meaningful. 

% Preseburger arithmetic and the real field do not satisfy this assumption, and also emptines for automata is undecidable for them.


\begin{theorem}\label{thm:single-use-automata-relational-structures}
    Consider a relational structure $\atoms$ that is oligomorphic and has a decidable first-order theory. Then the emptiness problem is decidable for all single-use automata models: monoids, and deterministic automata in both the one-way and two-way variants.
\end{theorem}

In the case of two-way automata, the single-use restriction is crucial, since without it the emptiness problem becomes undecidable for all choices of $\atoms$ where the universe is infinite. The above theorem is the first known example of two-way model that has decidable emptiness and uses atoms that are not the equality atoms. 


\section{Orbit-finite vector spaces}
Our first solution to the problem of orbit-finite function spaces is to add more sets and functions, so that function spaces are recovered. (The second solution will go the other way, and restrict the sets and functions.) 
This generalization is the orbit-finitely spanned vector spaces that were introduced in~\cite{bojanczykKM21OrbitFiniteVector}. This section contains no new technical results beyond those from~\cite{bojanczykKM21OrbitFiniteVector}. If any, the only added value is a different perspective.

To explain the intuitive reason behind the usefulness of vector spaces, consider finite (and not orbit-finite) sets. The number of functions from set with $n$ elements to a set with $m$ elements is $m^n$. However, if we consider linear maps instead of functions without any structure, then we get an exponential improvement, namely instead of $m^n$ we will have $m \cdot n$. 
This is because a linear map from a vector space of dimension $n$ to a vector space of dimension $m$ can be specified by giving an $m \times n$ matrix. As we will show in this section,  in the orbit-finite world, the improvement is more pronounced: instead of infinite, the space will be finite.

For the sake of concreteness, all  vector spaces will be over the field of rational numbers. However, the choice of field will not affect the results. The field has no atoms in it, and therefore it has a trivial action of atom automorphisms. In particular, when below we say that scalar multiplication is equivariant, we mean that for every scalar $\lambda$, the function $x \mapsto \lambda x$ is equivariant.
\begin{definition}[Vector space with atoms]
    A \emph{vector space with atoms} is a set with atoms, equipped with a vector space structure, such that the vector space structure is equivariant, i.e.~vector addition and scalar multiplication are equivariant operations.
\end{definition}


\begin{example} \label{ex:lina} For a set $X$, let us write $\Lin X$ for the vector space that consists of finite linear combinations of elements from $X$. When $X$ is the set of atoms, the resulting vector space $\Lin \atoms$ is a vector space with atoms, because it comes equipped with a natural action of atom automorphisms.  An example vector in this space is $
3 \cdot \text{John} + 2 \cdot \text{Eve} - 5 \cdot \text{Mark}.
$
\end{example}

When talking about vector spaces with atoms, one has to be careful about using bases. This is because finding a basis might require choice, and choice is not available in the presence of atoms. 
For example, consider space $\Lin \atoms$ from Example~\ref{ex:lina}, and its subspace $V$ which contains those vectors where all coefficients sum up to zero. This space is spanned (i.e.~generated using linear combinations) by the orbit-finite set which consists of all pairs $a - b$, where $a$ and $b$ are distinct atoms.
However, this spanning set is not a basis, since it contains linearly dependent vectors, for example John - Eve and Eve - John. Keeping only one of these vectors would require choice, which cannot be done using equivariant functions, and in fact this space has no basis that is equivariant~\cite[Example 6]{bojanczykKM21OrbitFiniteVector}. For this reason, the appropriate notion of finiteness is having an orbit-finite spanning set, which is not necessarily linearly independent. This leads to the following category.

\begin{definition}\label{def:orbit-finite-vector-space-category}
    The category of orbit-finitely spanned vector spaces is:
    \begin{enumerate}
        \item The objects are vector spaces with atoms that have an orbit-finite spanning set.
        \item The morphisms are equivariant linear maps.
    \end{enumerate}
\end{definition}

For the above category, we will use tensor product $\otimes$; this is the usual notion of tensor product for vector spaces. For example, if we take vector spaces $\Lin X$ and $\Lin Y$, i.e.~finite linear combinations of orbit-finite sets $X$ and $Y$ respectively, then the resulting tensor product is $\Lin (X \times Y)$. More generally, the tensor product of two orbit-finitely spanned vector spaces is also orbit-finitely spanned~\cite[Theorem VI.3]{bojanczykKM21OrbitFiniteVector}. 

In the terminology of category theory, the following theorem says that the category of orbit-finite vector spaces with atoms is symmetric monoidal closed, with respect to the tensor product. However, since the readers might not be familiar with category theory, we unfold the definitions in the statement of the theorem. 

\begin{theorem}\label{thm:orbit-finite-vector-space-closed}
    Let $V$ and $W$ be orbit-finitely spanned vector spaces. There exists an orbit-finitely spanned  vector space, denoted by  $\funspace V W$, and an equivariant linear map
    \begin{align*}
    eval : (\funspace V W) \otimes V \to W
    \end{align*}
    with the following property. For every equivariant linear map
    \begin{align*}
    f : X \otimes V \to W
    \end{align*}
    there is a unique equivariant linear map
    \begin{align*}
    h : X \to (\funspace V W)
    \end{align*}
    such that the following diagram commutes:
    \[
    \begin{tikzcd}
    X \otimes V 
    \arrow[r,"h \otimes id"]
    \arrow[dr,"f"']
    &
    (\funspace V W) \otimes V
    \arrow[d,"eval"] \\
    &
    W
    \end{tikzcd}
    \]
\end{theorem}
\begin{proof} We apply the original proof that the category of vector spaces (without atoms and orbit-finiteness) is symmetric monoidal closed. We then observe that: (1) the evaluation morphism is equivariant; (2) the resulting space is orbit-finitely spanned. 
    
    For the first observation (1), we simply note that the construction of the function space and evaluation morphism are constructed using the language of set theory, and therefore they will necessarily be equivariant~\cite[Equivariance Principle]{bojanczyk_slightly2018}.

    Let us now explain the second observation (2), about the function space being orbit-finitely spanned. This uses a non-trivial result from~\cite{bojanczykKM21OrbitFiniteVector} which says that orbit-finitely spanned spaces are closed under duals. 
    The space $\funspace V W$ is defined to be the space of finitely supported linear maps from $V$ to $W$, with the natural vecor space structure. This can be seen as a subspace of a larger space, call it $\funspacenonfs V W$, which contains all linear maps, not necessarily finitely supported. A standard result in linear algebra is that $\funspacenonfs V W$ is linearly isomorphic to $ \funspacenonfs {V \otimes W} 1$, where $1$ is the field. This isomorphism is equivariant, and it preserves the property of being finitely supported. Therefore, this gives us an equivariant linear isomorphism between $\funspace V W$ is isomorphic and $ \funspace {V \otimes W} 1$. The latter space is the dual of the orbit-finitely spanned space $V \otimes W$, and therefore it is orbit-finitely spanned by~\cite[Corollary VI.5]{bojanczykKM21OrbitFiniteVector}.
\end{proof}

A corollary of the above theorem is that, in the category of orbit-finitely spanned vector spaces, deterministic automata recognize the same languages as monoids. This is because the usual translation can be done, with a deterministic automaton of states $Q$ being mapped to a monoid with elements $\funspace Q Q$. This was already observed in~\cite[Theorem VIII.3]{bojanczykKM21OrbitFiniteVector}.

However, there are two limitations of the orbit-finitely spanned vector spaces. 

The first limitation is that the existence of function spaces is dependent on the choice of atoms. Theorem~\ref{thm:orbit-finite-vector-space-closed}  works when the atoms have equality only, and it also works when the atoms are equipped with a total order. This is because the dual spaces are orbit-finitely spanned in these cases, as proved in~\cite[Corollary VI.5]{bojanczykKM21OrbitFiniteVector}. However, the dual spaces are no longer orbit-finitely spanned for other choices of  atoms, such as atoms with a graph structure, see~\cite[Example 9]{bojanczykKM21OrbitFiniteVector}.

A second limitation is the inability to simulate two-way automata with one-way automata. This limitation, and the corresponding properties of the underlying category, will be discussed in Section~\ref{sec:two-way-automata}. In the following section, we discuss an alternative category, called the category of single-use sets, which is also based on the category of orbit-finite sets, and which does not have these limitations. However, the alternative category is smaller, in the following sense: there are faithful functors 
\[
\begin{tikzcd}
\text{single-use sets} \ar[r, ] & 
\text{orbit-finite sets} \ar[r] &
\text{orbit-finite vector spaces}
\end{tikzcd}
\]

\bibliography{bib}

\appendix

\newcommand{\invar}[1]{#1_{\mathrm{in}}}
\newcommand{\outvar}[1]{#1_{\mathrm{out}}}

\section{Game semantics}
\label{sec:game-semantics}

This section is devoted to the proof of Theorem~\ref{thm:single-use-closed}. To construct the function space $X \Rightarrow Y$, we use game semantics to identify a certain normal form of programs that compute single-use functions. The presentation in this section is self-contained, and does not assume any knowledge of game semantics. We base our notation on~\cite{abramsky2013semantics}.

Let us begin with a brief motivation for why game semantics will be useful.

While it is intuitively clear which functions should be allowed as single-use for simple types such as $\atoms \to 1+1$ or $\atoms \otimes \atoms \to \atoms + \atoms$, these intuitions start to falter when considering more complex types. How can one show that a function is \emph{not} single-use? If one were to use the definition of single-use functions alone, then one would need to rule out any possibility of constructing the function from the primes using the combinators, including constructions that use composition many times, and with unknown intermediate types. 

This is the reason why we consider game semantics. It will allow us to give  a more principled description of the intuition that pairs of type $X \otimes Y$ can be used on both coordinates, while pairs of type $X + Y$ can be used on a chosen coordinate only. The idea behind game semantics is to give the description in terms of an interaction between two players.  The two players are:
\begin{enumerate}
    \item System, who represents the function (we will identify with this player); and
    \item Environment, who supplies inputs and requests outputs of the function.
\end{enumerate}
One of the intuitions behind the setup is that if a type $X \& Y$ appears in the input of the function, then it is the System who can choose to use $X$ or $Y$, while if the type appears in the output, then it is the Environment who makes the choice. (In this paper, we consider functions of first-order types of the form $X \to Y$, where $X$ and $Y$ are linear types that do not use $\otimes$, and therefore there will be a clear distinction between input and output values.)
Before giving a formal definition of game semantics, we give simple example of the interaction.

\begin{example}\label{ex:amp-otimes-distr}
    Consider the two types 
    \begin{align*}
        X \otimes (Y \& Z) 
        \quad \text{and} \quad
        (X \otimes Y) \& (X \otimes Z).
    \end{align*}
    Among the prime functions in Table~\ref{fig:prime-morphisms-with-with}, we find distributivity in the direction $\rightarrow$, but not in the direction $\leftarrow$. We explain this asymmetry using the interaction between two players System and Enviroment.

    Let us first consider the interaction in the  direction $\rightarrow$. Player Environment begins be requesting an output. Since this output is of type $(X \otimes Y) \& (X \otimes Z)$, this means that Environment can choose to request either of the two types  $X \otimes Y$ and $X \otimes Z$. Suppose that Environment requests $X \otimes Y$. Now player System needs to react, and produce two elements: one of type $X$ and one of type $Y$. Both can be obtained from the input; for the second one player System can choose how to resolve the input $Y \& Z$ to get the appropriate value. 

    Consider now the interaction in the opposite direction $\leftarrow$. As we will see, player System will be unable to react to the behavior of player Environment, which will demonstrate that there is no distributivity in this direction. The problem is that player Environment can begin by requesting an element of type $X$, since the output type is $X \otimes (Y \& Z)$, while still reserving the possibility to request $Y \& Z$ in the future (because the tensor product $\otimes$ means that both output values need to be produced). To produce this element, player System will need  choose one of the two coordinates in the input type $(X \otimes Y) \& (X \otimes Z)$, and any of these two  choices will be premature, since player Environment can then request the opposite choice in the output type.  \exampleend
\end{example}

As illustrated in the above example, we will use a game to describe the possible behaviours of a function, as modelled by behaviours of player System. The game will be played in an arena, which will arise from the type of the function, and will tell us what are the possibilities for the moves of both players.  Here is the  outline the plan for the rest of this section. 
\begin{enumerate}
    \item For every two linear types $X$ and $Y$, we  define an arena, which is a data structure that describes all possible interactions between players Environment and System that can arise when running a  single-use  functions of type $X \to Y$;
    \item In each arena, we will be interested in the strategies of player System, i.e.~the ways in which System reacts to moves of Environment. We will show how such strategies can be interpreted as single-use functions: to a strategy we will assign a single-use function of type $X \to Y$ that is represented by this strategy. This mapping will be partial, i.e.~some ill-behaved strategies will not represent any functions. 
    \item We will show that the  set of strategies in an arena is  large enough to represent all single-use functions, but small enough to be orbit-finite. 
    \item We will then strengthen this: not only is the set of strategies orbit-finite, but it can be equipped with the structure of a linear type, such that both evaluation and currying will be single-use functions.
\end{enumerate}

The arenas from the first step of the plan will  be defined in two sub-steps. We begin by defining arenas and strategies for functions that do not use the structure of the atoms, i.e. constants and equality tests. This will be a fairly generic definition, almost identical to the classical game semantics for linear logic. Then we will extend the definition to cover the extra structure. 

\subsection{Arenas and strategies without constants and equality tests}
\label{sec:arenas-without-constants-and-equality-tests}

We begin with a simpler version of the game semantics, in which the arenas and strategies will describe functions that are not allowed to perform equality tests, and are not allowed to use constants. These strategies will model functions such as the identity function $\atoms \to \atoms$, which directly passes its input to its output, but they  will not model the equality test $\atoms \otimes \atoms \to 1 + 1$, or the constant functions of type $1 \to \atoms$. The general idea is to use standard game semantics for linear logic, with an extra feature that we call \emph{register operations}. The register operations will be used to model the way in which atoms are passed from the input to output. For example, in the identity function,  Environment will  write the input atom into the register, and then player System will read the output atom from that register. The following definition of an arena is based on the definition from \cite[p.4]{abramsky2013semantics}, slightly
adapted for the context of this paper:
\begin{definition}[Arena] \label{def:arena}
    An \emph{arena} consists of:
    \begin{enumerate}
        \item A set of \emph{moves}, with each move having an assigned owner, who is either ``System'' or ``Environment'', and one of three\footnote{\label{footnote:read-write} In all arenas that we consider, the ``read'' moves will be owned by System and the ``write'' moves will be owned by Environment. Therefore, we could simplify the register operations and have just one, called ``read/write'', whose status is determined by its owner. 
        } register operations, which are ``none'', ``read'', or ``write''.
                \item A set of plays, which a set of finite sequences of moves that is closed under prefixes, and such that in every play, the owner of the first move is Environment, and then the owners alternate between the two players.
    \end{enumerate}
\end{definition}




An arena will correspond to a type. The inhabitants of that type, which will be  functions if the type is a functional type $X \to Y$, will be described by strategies in the arena. Such a strategy tells us how player System should react to the moves of player Environment. Intuitively speaking, in the case of a functional type, a strategy will say how the function reacts to requests in the output type and values in the input type. We will only be talking about strategies for player System, so from now on, all strategies will be for player System. The following definition corresponds to the definition from \cite[p.5]{abramsky2013semantics}:
%unless otherwise stated.
% In the above definition, the operations are a non-standard part of the arenas.  The idea is that the operations modify a memory store which has exactly one register, which contain a single atom or be empty.  A move that has an associated operation that is not ``none'' will be called an \emph{atom move}.


\begin{definition}[Strategy]
    A strategy  in an arena  is a subset of plays in the arena, which: 
    \begin{enumerate}
        \item is closed under prefixes;
        \item\label{item:sys-ext} if the strategy contains a play $p$ that ends with a move owned by player System, then it also contains all possible plays  in the arena that extend $p$ with one move of player Environment;
        \item\label{item:env-ext} if the strategy contains a play $p$ that ends with a move owned by player Environment, then it contains exactly one play  in the arena that extends $p$ with one move of player System;
         \item there is some $k$ such that all plays in the strategy have length at most $k$;
        \item every ``read'' move is directly preceded by a ``write'' move (in particular a play cannot begin with ``read''), and every ``write'' move is either the last move, or directly succeeded by a ``read'' move.
    \end{enumerate}
\end{definition}

Conditions \ref{item:sys-ext} and \ref{item:env-ext}, which are standard in game semantics,  guarantee that the strategy only ``ends'' when 
Environment has no moves to play.  Let us now comment on the last two conditions, which are not standard.

The fourth condition is motivated by the idea that we study ``finite'' types,  and there will be no need for unbounded computations. 

The last condition will be called the  \emph{immediate read condition}. It ensures that there is matching between ``read'' and ``write'' moves in plays that do not end with write. Since ``write'' will always be owned by Environment, the immediate read condition will ensure a matching between ``write'' and ``read'' moves.
%in the plays that do not end with a move by player Environment, and such plays will be the most important ones. 

We now show how to associate to each linear type a corresponding arena, and also how to associate an arena to a function type $X \to Y$. This definition will be compositional, i.e.~it will arise through operations on arenas that correspond to the type constructors such as $1$ or $\otimes$. The arenas that we will construct so far will not be our final proposal, since the corresponding strategies will not be able to use constants or perform equality tests. This will be fixed in Section~\ref{sec:arenas-with-constants-and-equality-tests}, where a more complex arena will be defined for function types. Before giving formal definition, we begin with a simple example of the arena for the type $\atoms \to \atoms$.
\begin{example}\label{ex:identity-function-without-equality-tests-and-constants}
    We define an arena for the type $\atoms \to \atoms$. This arena will be rather impoverished, since the  only allowed strategy in it will correspond to the identity function. However, this is consistent with the stage that we are at, where we only consider functions that do not use constants or perform equality tests; for such functions the only possibility is the identity function.

     The arena will describe the following interaction between the two players: Environment  requests an output,  then System requests an input, then Environment grants the input, and finally player System grants the output by forwarding the input grant. 
    The arena is shown in the following picture: 
    \mypic{6}
    The methodology of drawing this picture will become clearer later on, as we define operations on arenas such as $\otimes$. For the moment we describe the arena without caring that it arises as a special case of some general construction. The arena has four moves: 
    \begin{center}
        \begin{tabular}{lll}
         move & owner & register operation \\
            \hline
            request output & Environment & none \\
            request input & System & none \\
            grant input & Environment & write \\
            grant output & System & read 
        \end{tabular}    
    \end{center}
    The set of plays is defined as follows. These are all sequences that begin with a move of player Environment, alternate between players, use each move only once, and have the following condition:  ``grant output'' can only be played after ``request output'', and likewise for ``grant input'' and ``request input''.  

    A quick inspection of the above definition reveals that the arena has a unique maximal play, where the moves are played in the order from the table, and all other plays are prefixes of this maximal play. Because of the uniqueness of responses, the set of plays is also a strategy. As mentioned at the beginning of this example, the strategy describes the identity function.  \exampleend
\end{example}

We hope that the above example explains some intuitions about how arenas describe types and strategies describe functions. We now give a formal definition. As mentioned before, this definition is compositional: we define arenas for the basic types $1$ and $\atoms$, and then we define operations on arenas that correspond to the type constructors  $+$, $\&$, and $\otimes$. We begin with the basic types.


\begin{definition}[Arenas for $1$ and $\atoms$]\label{def:arenas-without-atoms-or-functions} \ 
    \begin{enumerate}
        \item     The arena for the type $1$ is empty: there are no moves and the only play is the empty sequence. 
        \item The arena for type $\atoms$ has two moves, which must be played one after the other: first player Environment makes a move called ``request'' that has no register operation, and then player System responds with  a move called ``grant'' that has register operation ``read''.
    \end{enumerate}
\end{definition}

In the above definition, we only described the behaviour of $\atoms$ when viewed as an output type. To get the input type, where the players are swapped and read is swapped with write, we will use duality, which is another operation on arenas. This operation, together with other operations that correspond to the type constructors, are defined below.   

\begin{definition}[Operations on arenas]\label{def:composition-of-arenas}
    Let  $A$ and $B$ be arenas. We define the following arenas (see also Figure~\ref{fig:arena-constructors}):
        \begin{description}
            \item[$A+B$] The moves in this arena are the disjoint union of the moves of $A$ and $B$, with inherited owners and register operations, plus three extra moves: ``ask'' owned by Environment, and ``left'', ``right'' owned by System. 
            The plays are defined as follows. Player Environment begins with  an ask move, then System responds with a left or right move, and the remaining sequence is a play in the arena $A$ or $B$, depending on whether System chose left or right. 
            \item[$A \& B$] The moves in this arena are the disjoint union of the moves of $A$ and $B$, with inherited owners and register operations, plus three extra moves: ``acknowledge'' owned by System, and ``left'', ``right'' owned by Environment. 
            The plays are defined as follows. Player Environment begins by choosing left or right, then player System responds with an acknowledge move, and the remaining sequence is a play  in the arena $A$ or $B$, depending on whether Environment chose left or right.
            (This construction differs slightly from the one from \cite[Excercise~1.10]{abramsky2013semantics} -- this is because we want to keep it analogous to the construction for $A + B$.)
            \item[$A \otimes B$] The moves in this arena are the disjoint union of the moves of $A$ and $B$, with inherited owners and register operations. A play in this arena is any shuffle of plays in the two arenas $A$ and $B$. (A shuffle of two words is any word obtained by interleaving them, e.g.~shuffles of ``abc'' and ``123'' include ``a1b23c'' and ``12ab3c'').
            By Definition~\ref{def:arenas-without-atoms-or-functions}, we require that the owners of the move alternate in the 
            interleaved sequences. (This construction is based on \cite[p.7]{abramsky2013semantics}.) 
            

            \item[$\bar A$] This is called the dual arena of $A$. The moves and plays are the same as in $A$, except the owners are swapped, and the ``read'' and ``write'' register operations are swapped.
        \end{description} 
\end{definition}

\begin{figure}
The arena $A+ B$:
            \mypic{7}
The arena $A \& B$: 
            \mypic{8}
The arena $A \otimes B$:
            \mypic{9}
    \caption{\label{fig:arena-constructors} Pictures of the operations on arenas. The picture for $\otimes$  is less useful than previous two, since the root node of the tree is not a player, but a node labelled by $\otimes$. The intuition is that the game is played in parallel on both arenas, and therefore a position in it can be visualized as a pair of positions in the two arenas.}
\end{figure}


        
Equipped with the above definitions, we  present our first attempt at assigning arenas to types. In the second item of the following definition, we use the name \emph{library-less}, because our final definition of the arena for a function type, as presented in the next section, will be equipped with an extra feature that will be called a library. 

\begin{definition} Let $X$ and $Y$ be linear types.
    \begin{itemize}
        \item The \emph{arena for  $X$} is defined by inductively applying the constructions from Definition~\ref{def:arenas-without-atoms-or-functions} and Definition~\ref{def:composition-of-arenas} according to the structure of the type.
        \item The \emph{library-less arena for $X \to Y$} is defined to be (dual of arena of $X$) $\otimes$ (arena of $Y$).
    \end{itemize}
\end{definition}

As discussed previously, our notion of arenas does not yet take into account the structure of the atoms, i.e.~the constants and equality tests. This will be fixed in the next section, by modifying the second item in the above definition. On the other hand, the arenas from that first item in the above definition, for linear types without function types, are already in their final form. 

In principle the construction from the second item in the above definition can be nested, and thus used to assign arenas to higher order types that can nest $\to$ with the other type constructors. This is how it is usually done in linear logic. However,  the construction that we will describe in the next section will  be less amenable to nesting, and will use it only  to describe functions between types that do not use $\to$.


 

\begin{example}
    The arena for the identity type $\atoms \to \atoms$ is the same as the arena from Example~\ref{ex:identity-function-without-equality-tests-and-constants}.
\end{example}





\subsection{Arenas and strategies with constants and equality tests}
\label{sec:arenas-with-constants-and-equality-tests}
In the Section~\ref{sec:arenas-without-constants-and-equality-tests}, we described arenas for functions that did not use the structure of the atoms, i.e.~constants and equality tests. We now show how these arenas can be extended to cover this structure. The general idea is to equip the arenas with an extra part, which we call the \emph{library},  that describes the allowed operations on the atoms. (The library as we present it here only contains functions for equality and constants, but in the proof of Theorem~\ref{thm:single-use-automata-relational-structures}, we can use a library that has other relations beyond equality. )



\begin{definition}[The library arena] \label{def:library-arena} The library arena and its parts are defined as follows, see Figure~\ref{fig:library-arenas} for pictures.
    \begin{enumerate}
        \item The \emph{constant choice arena} is the following arena $\atoms+1$ moves:
        first player System chooses an atom, then player Environment plays move with register operation ``write''. 
        \item The \emph{equality test arena} is an arena which the plays are as follows:
    \begin{enumerate}
        \item first player System plays a move with register operation ``read'';
        \item then player Environment plays an move with no register operation;
        \item then player System plays a move with register operation ``read'';
        \item then player Environment plays one of two moves, called $=$ and $\neq$, with no register operation.
    \end{enumerate}
    \item The \emph{library arena} is defined to be an arena that is obtained by applying $\otimes$ to infinitely many copies of the constant choice arena and infinitely many copies of the  equality test arena.
    \end{enumerate}
\end{definition}

\begin{figure}
The constant choice arena:
        \mypic{10}
The equality test arena:
    \mypic{11}
\caption{\label{fig:library-arenas} Pictures of the library arenas.  We use the convention that the register operations are in red, and the names of the moves, which have no other role than to distinguish them, are in black. 
Note that the first move in this arena is owned by System, and we assume in Definition~\ref{def:arena} that the first move is owned by Environment. This is because this arena, like all arenas in Definition~\ref{def:library-arena}, is not intended to be a stand-alone arena, but only as part of the bigger arena from Definition~\ref{def:arena-for-function-type} where the first move is indeed owned by Environment. } 
\end{figure}
       
The library arena is infinite. Taking the tensor product of infinitely many copies of the two arenas ensures that the library arena satisfies the following property, which corresponds to the $!$ operation from linear logic: 
\begin{align}\label{eq:bang-library-arena}
\text{library arena} 
\quad \equiv \quad 
\text{(constant choice arena)} \otimes 
\text{(equality test arena)} \otimes
 \text{(library arena)}.
\end{align}
In the above, $\equiv$ refers to isomorphism of arenas, which is defined in the natural way: this is a bijection between the moves, which is consistent with the owners, register operations and plays in the expected way.  Another property is that the library arena is isomorphic to a tensor product of itself: 
\begin{align}\label{eq:library-arena-isomorphism}
\text{library arena}
\quad \equiv \quad
\text{library arena} \otimes \text{library arena}.
\end{align}


We are now ready to give the final definition of arenas for functions between linear types, which takes into account the structure of the atoms.

\begin{definition}[Arena for a function type]\label{def:arena-for-function-type} For linear types $X$ and $Y$, the arena of $X \to Y$ is 
    \begin{align*}
    \text{(library arena)} \otimes \text{(dual arena of $X$)} \otimes \text{(arena of $Y$)}.
    \end{align*}
\end{definition}

This completes the game semantics of linear types and functions between them. We do not intend to give game semantics for higher order types, such as functions on functions etc. As a result, we will only be using the dual once, namely for the arena of the input type. Also, note that the read/write operations will be used in a restricted way, as announced in Footnote~\ref{footnote:read-write}, namely that the ``read'' moves will be owned by System and the ``write'' moves will be owned by Environment.  This is because the library arena has this property, the arena for $\atoms$ has this property, and all operations on arenas that we have defined preserve this property.

\subsubsection{Composition of strategies}
\label{sec:composition-of-strategies}
One of the main points about strategies in game semantics is that they can be composed.
The usual construction for the function type $X \to Y$ is to take the tensor product of the dual arena for $X$ and the arena for $Y$. 
Our construction is a bit more involved, since the arena that we use has a copy of the library arena. We now explain how to compose strategies in a way that accounts for the library arena. 

We begin by describing the usual construction for composing strategies, which we call \emph{shuffling and hiding}, see \cite[p.12]{abramsky2013semantics}, with a minor adaptation to our setting that has register operations. 

\begin{definition}[Shuffling and hiding]\label{def:shuffling-and-hiding}
    Let $A, B, C$ be arenas, and consider two strategies
    \begin{align*}
    \sigma_1 \text{ in the arena $A \otimes \bar B$}
    \qquad
    \sigma_2 \text{ in the arena $B \otimes C$}.
    \end{align*}
    The shuffling and hiding strategy for  $\sigma_1$ and $\sigma_2$, which is a strategy  in the  arena $A \otimes C$, is defined to be the set of plays    $p$  such that there exist plays $p_1 \in \sigma_1$ and  $p_2 \in \sigma_2$  with the following properties: 
\begin{itemize}
    \item the play $p$ satisfies the immediate read condition;
    \item the following two sequences of moves are equal: 
    \begin{enumerate}
        \item the subsequence of moves in $p_1$ that are from the arena $\bar B$;
        \item the subsequence of moves in $p_2$ that are from the arena $B$.
    \end{enumerate}
\end{itemize}
\end{definition}

Note that the set of moves in the arenas $\bar B$ is the same as the set of moves in the arena $B$ (the owners and register operations are changed),  and therefore the subsequences in items 1 and 2 above can be meaningfully compared.  The following lemma, whose straightforward proof is left to the reader,  shows that the above definition is well-formed.


\begin{lemma}
    The strategy described in Definition~\ref{def:shuffling-and-hiding} is a valid strategy.
\end{lemma}

% To show that this construction 
% preserves the immediate read condition, let us consider two consecutive moves 
% $m_e, m_s$ in a sequence from $\sigma;\tau$,
% such that $m_e$ is a write move by the environment, 
% and $m_s$ is the system's response, and let us show 
% that $m_s$ is a read move. Since moves $m_e$ and $m_s$ are consecuive in 
% $\sigma ; \tau$, it follows that in some sequence in $\sigma || \tau$, 
% they are separated by some moves from $B$:
% \[ \ldots m_e b_1 b_2 \ldots b_n m_s \ldots \quad \in \quad \sigma || \tau \]
% Assume (w.l.o.g.) that $m_e$ belongs to $\sigma$.
% Then, from definition of $\sigma || \tau$,
% we know that $m_e$, $b_1$ is a consecutive pair of moves in (some sequence from) $\sigma$. 
% It follows that $b_1$ is a read move in $\bar{A} \otimes B$,
% which means that $b_1$ is a write move in $\bar{B} \otimes C$. 
% By definition of $\sigma || \tau$, we know that $b_1, b_2$
% are consecutive moves in $\tau$, which means that $b_2$ is 
% a read-move in $\bar{B} \otimes C$, which in turn means that 
% it is a write-move in $\bar{A} \otimes B$. By repeating this reasoning, 
% we obtain that $b_n$ is a wite move in either $\bar{A} \otimes B$ or
% $\bar{A} \otimes B$, which means that $m_s$ is a write move from that arena.
% We can use the same reasoning to show that if $m_s$ is a read move, 
% then $m_e$ has to be a write move.  


We can now use shuffling and hiding to compose strategies in the arena for a function type that were defined in Definition~\ref{def:arena-for-function-type}.
Consider three linear types $X, Y, Z$, and two strategies, in the arenas for $X \to Y$ and $Y \to Z$, respectively.
 By unfolding the definition of arenas from Definition~\ref{def:arena-for-function-type}, these are strategies in the arenas 
\begin{align*}
\text{(library arena)} \otimes \overline{\text{arena for $X$}} \otimes \text{arena for $Y$} & \qquad \text{the arena for $X \to Y$, and}
\\
\text{(library arena)} \otimes \overline{\text{arena for $Y$}} \otimes \text{arena for $Z$} & \qquad \text{the arena for $Y \to Z$}.
\end{align*}
Using the construction from Definition~\ref{def:shuffling-and-hiding}, we can combine them into a single  strategy in 
\begin{align*}
\text{(library arena)} \otimes  \text{(library arena)}\otimes \overline{\text{arena for $X$}} \otimes \text{arena for $Z$}.
\end{align*}
Since the library arena is isomorphic to a tensor product of two copies of itself, the above strategy gives us a strategy in the arena 
\begin{align*}
    \text{(library arena)} \otimes   \overline{\text{arena for $X$}} \otimes \text{arena for $Z$} & \qquad \text{the arena for $X \to Z$.}
    \end{align*}

This strategy is defined to be the \emph{composition} of the original two  strategies. We write $\sigma; \tau$ for this composition. Thanks to \cite[Proposition~1.2]{abramsky2013semantics}, we know that the the usual
composition of library-free strategies is associative. It follows that our composition of library strategies is associative 
up to isomoprhism. 


\subsection{Strategies as single-use functions}
In this section, we explain how a strategy in a function type $X \to Y$ can be interpreted as a single-use function from $\sem X$ to $\sem Y$.
% This is done by interpreting values in the sets $\sem X$ and $\sem Y$ as strategies in the arenas $1 \to X$ and $1 \to Y$, respectively, and then lifting the interpretation to the function type using strategy composition. 




% \subsubsection{Elements of $\sem X$ as strategies in $1 \to X$}
% \label{sec:elements-of-sem-x-as-strategies-in-1-to-x}
% % We begin by interpreting the strategies in a linear type $1 \to X$ as elements of the underlying set $\sem X$.   
% % This will be done in two directions, as depicted in the following diagram.
% % \[
% % \begin{tikzcd}
% %     [column sep=6cm]
% % \text{strategies in the arena for $1 \to X$} 
% % \arrow[r, "{\text{to each strategy $\sigma$ we can associate an element $\sem \sigma \in \sem X$}}", from=1-1, to=1-2, start anchor=east, end anchor=west, bend left=10, shift right=-1ex, twoheadrightarrow]
% % &
% % \sem X
% % \arrow[l, "{\text{to each element $x \in \sem X$ we can associate a strategy $\sigma_x$}}", from=1-2, to=1-1, start anchor=west, end anchor=east, bend left=10, shift right=-1ex, rightarrowtail]
% % \end{tikzcd}
% % \]
% We begin by showing how an element of the underlying set in a type $X$ can be represented by some strategy in the arena for $1 \to X$. This is done by a straightforward induction on the structure of $X$, as described below. 

%     \begin{enumerate}
%         \item The strategy corresponding to the unique element of $1$ is the empty strategy.
%         \item The strategy corresponding to an atom $a \in \atoms$ is defined as follows: player Environment requests and output, then System requests the constant $a$ in the constant choice arena, Environment grants the constant, and System grants the output;
%         \item The strategy corresponding to $x_1 \otimes x_2$ is defined by composing the strategies corresponding to $x_1$ and $x_2$ in the natural way. The only point that we need to care about is the immediate read condition, so System needs to respond immediately in the same of the two arenas $X_1$ and $X_2$ when a write operation is played. 
%         \item The strategy corresponding to $x_1 \& x_2$ is defined as follows. In the first move, Environment chooses left or right, then System acknowledges this choice, and in the remaining game System plays according to the strategy for $x_1$ or $x_2$, depending on the first choice of player System.
%         \item Similarly, we define the strategy corresponding to $x_1 + x_2$.
%     \end{enumerate}

\subsubsection{Strategies in $1 \to X$ as values in $\sem X$}
\label{sec:strategies-in-1-to-x-as-elements-of-sem-x}
We start by showing how to interpret strategies in the arena for  $1 \to X$ as values in $\sem{X}$. Later, we will lift  this interpretation to functions  using strategy composition. This interpretation is a partial function, i.e.~some strategies in the arena for $1 \to X$ will be considered invalid, and will not represent any value in $\sem X$. 

We now describe the converse of the above construction, i.e.~we show how every strategy in the arena for $1 \to X$ can be interpreted as representing some value in $\sem X$. (For now we do not require that the transformation from strategies to values be single-use.)
Observe that we are not using the arena for a linear type $X$ itself, but for the function type $1 \to X$. The reason is that the second arena contains the library, which will be used to produce individual atoms. 
% This interpretation is more interesting than the previous one, since we need to account for suboptimal strategies of player System, in which the library arena is used to execute meaningless operations. For example, System might ask if atoms Mark and John are equal, and to make things worse, Environment might respond dishonestly that they are equal.  
% \begin{definition}
%     A strategy in an arena of type $X \to Y$ is called \emph{$k$-bounded} if all plays have length at most $k$. It is called \emph{bounded} if it is $k$-bounded for some $k$. 
% \end{definition}
% Formally, for every type $X$, we would like to define a function: 
% \[ \mathtt{val} : \textrm{(strategies in $1 \to X$)} \to X \]
% For this we are going to define inductively the following function: 
% \[ \mathtt{val'} :  \textrm{(strategies in $1 \to X_1 \otimes \ldots X_n$)} \times (\atoms + \bot)  \]

The general idea for the construction is that, given a strategy of system, we play an the environment to extract the value. 
Here are the details:
\begin{enumerate}
    \item For the type $1$ there is no difficulty, since the corresponding set has a unique element, which will be assigned to all strategies. Also, there is a unique play in the corresponding arena, which is $1 \to 1$. This unique play is  the empty play,  because player Environment cannot make a first move. Therefore, the assignment of values to strategies poses no difficulty, since there is one strategy and one value.  However, already at this stage there is a slightly subtle point, since the arena for $1 \to 1$ is not completely empty, because it  contains a copy of the library arena, however all moves in this arena are blocked, since they must begin with a move by player System, and player System cannot  move before Environment; this is similar to the ordre de préséance at Versailles.
    \item Consider now the type $\atoms$.  We will assign to each strategy in the arena for
    $1 \to \atoms$ an element of the underlying set $\atoms$.
    As we play as the environment, the first move belongs to us. 
    We use it to play ``request'' in $\atoms$ (this is the only move 
    available to the enviroment). Then we look at the response of the system according to the strategy.
    It has to be an atom move in the constant choice arena (the only other two moves 
    avaliable to the system are ``grant'' in in $\atoms$ and ``read'' in equality test, 
    however both of those are read moves, and they can only be played immediately after a 
    write move). We (i.e. the environment) respond with the write move in the constant choice arena. 
    At this point the system might decide to spend some time in the library requesting 
    for constants and comparing them to each other (this is a non-standard 
    behaviour, because the system should already know the results of constants comparisons). 
    While the system does this, we (i.e. the environment) simulate the library, granting the contants, and replying 
    to equality queries according to the equality of the requested constants.    
    Since the strategy of the system is bounded, it has to finish the game in $k$ moves, 
    so it will eventually have to play ``grant'' in $\atoms$ (beacuse only then enviroment will be left with no moves). 
    At this point we assign to the strategy, the most recent atomic constant requested by the system.

    % To define this atom, we will simply follow the strategy according to an honest behavior of player Environment, i.e.~where Environment gives correct answers to all equality tests. 
    
    % \begin{definition}
    %     A play in an arena of type $X \to Y$ is called \emph{honest} if all equality tests are answered correctly by player Environment. More formally, (...)
    % \end{definition}

    % One can see that for every strategy in the arena for $1 \to \atoms$, there is a unique maximal honest play that is consistent with this strategy. In this play, eventually System must execute the read operation on the atom, and the preceding write move necessarily had to be an atom constant (since these are the only write moves in the arena). This constant is defined to be the value of the strategy. 

    \item Now we show how to assign a value $\sem{X + Y} = \sem{X} + \sem{Y}$ to a strategy of type $1 \to X+Y$. We start by playing 
    ``ask''. The system can now, again, spend some time in the library requesting for constants and comparing them to 
    each other -- we simulate this by replying to equality tests according to the equality of constants. 
    Since the system is bounded it will eventually have to play either ``left'' or ``right''.
    If it plays ``left'', we obtain a strategy for $1 \to X$. We use induction to
    compute its value $v \in \sem{X}$ and we return $\text{left}(v)$. 
    If the system plays ``right'', the construciton is analogous. 

    \item Now we show how to compute $\sem{X \& Y} = \sem{X} \times \sem{Y}$ from the strategy $1 \to X \& Y$.
          This means that we need to construct two values: $x \in \sem{X}$ and $y \in \sem{Y}$.
          To compute $x \in X$, we start by playing ``left''. Then system might spend some time in the library -- 
          we deal with this as in the previous cases, but eventually it will have to play ``acknowledge''. 
          At this point we obtain a system's strategy for $1 \to X$, and again we use induction to comptue $x \in \sem{X}$.
          In order to compute $y \in Y$, we take the original strategy and play ``right'' instead. 


    \item Finally, we show how to compupute a value $\sem{X \otimes Y} = \sem{X} \times \sem{Y}$ from 
          the strategy $\sigma : 1 \to X \times Y$. We start by treating $\sigma$ as in strategy 
          in $1 \to X$ and use the inductive assumption to compute the value $x$.
          This is possible because if the environment never plays in $Y$, then 
          the system can never reply in $Y$ (due to the Versailles-like customs of the arenas).
          Then, we are left with a strategy that is effectively equivalent to a strategy
          in $1 \to Y$ (beacuse the $X$-part of $\sigma$ is already entirely played out), 
          so we can use the induction assumption to transform it into the valye $y \in Y$, 
          and return $(x, y) \in \sem{X} \times \sem{Y}$. 
\end{enumerate}



\subsubsection{Strategies in $X \to Y$ as single-use functions}
We now show how to interpret a strategy in the arena for $X \to Y$ as a single-use function of type $X \to Y$:
%This is done by putting together the operations that have been defined previously. 

\begin{definition}\label{def:strategy-as-single-use-function}
    Let $X$ and $Y$ be linear types. For a strategy $\sigma$ in the arena $X \to Y$,
    and a function $f : \sem{X} \to \sem{Y}$, we say that $\sem{\sigma} = f$ if for every strategy 
    $\nu_x : 1 \to X$, it holds that: 
    \[ \mathtt{val}_Y(\nu_x ; \sigma) = f(\mathtt{val}_X(\nu_x)) \textrm{,} \]
    where $\mathtt{val}_X$ is the mapping from strategies of type $1 \to X$ to elements of $\sem{X}$ 
    defined in Section~\ref{sec:strategies-in-1-to-x-as-elements-of-sem-x}.
    % the represented function  $\sem \sigma$ is defined to be the composition of the following three transformations:
    % \[
    % \begin{tikzcd}
    % \sem X  
    % \ar[d,"{\text{transformation defined in Section~\ref{sec:elements-of-sem-x-as-strategies-in-1-to-x}}}"] 
    %  \\ 
    % \text{strategies in arena for $1 \to X$}
    % \ar[d,"{\tau \mapsto \text{$\tau;\sigma$ as defined in Section~\ref{sec:composition-of-strategies}}}"] 
    % \\
    % \text{strategies in arena for $1 \to Y$}
    % \ar[d,"{\text{transformation defined in Section~\ref{sec:strategies-in-1-to-x-as-elements-of-sem-x}}}"] 
    % \\ 
    % \sem Y
    % \end{tikzcd}
    % \]
\end{definition}

We now show that game semantics is a complete representation for single-use functions:
\begin{lemma}
    Let $X$ and $Y$ be linear types, then every single-use function of type $X \to Y$ is represented by at least one strategy. 
    % \begin{itemize}
    %     \item \emph{Soundness.} Every function represented by a strategy in the arena for $X \to Y$ is single-use.
    %     \item \emph{Completeness.} Every single-use function of type $X \to Y$ is represented by at least one strategy.
    % \end{itemize}
\end{lemma}
\begin{proof}
    It suffices to show that all single-use prime functions can be represented as strategies,
    and show how to raise combinators $\circ$, $\otimes$, $\times$, $\&$ from functions to 
    strategies in a way that preserves the semantics. Let us start with the combinators:
    \begin{description}
        \item[$f \circ g$] Here we use the construction for strategy composition from Section~\ref{sec:composition-of-strategies}.
        This means that we need to show that for all strategies $\sigma : X \to Y$ and $\tau: X \to Z$, 
        and for all functions $f : \sem{X} \to \sem{Y}$ and $g: \sem{Y} \to \sem{Z}$,
        if $\sem{\sigma} = f$ and $\sem{\tau} =g$, then $\sem{\sigma; \tau} = g \circ f$. By \ref{def:strategy-as-single-use-function}, 
        this means that we need to show that for all strategies $\nu : 1 \to X$, it holds that: 
        \[ \mathtt{val}(\nu; (\sigma; \tau)) = g(f(\mathtt{val}(\nu))) \]
        First, let us notice that (as explained in Section~\ref{sec:composition-of-strategies}) 
        the strategy $\nu; (\sigma; \tau)$ is equal to $(\nu; \sigma); \tau$ up to arena isomorphism. 
        Since area isomoprhisms preserve $\mathtt{val}$ (as they only rename library funcions), 
        we obtain that:
        \[ \mathtt{val}(\nu; (\sigma; \tau)) = \mathtt{val}((\nu; \sigma;) \tau) = g(\mathtt{val}(\nu; \sigma)) = g(f(\mathtt{val}(\nu))) \]

        \item[$f + g$] First, let us show how to construct a strategy $\sigma_1 + \sigma_2 : (X_1 + X_2) \to (Y_1 + Y_2)$ 
        from strategies $\sigma_1 : X_1 \to Y_1$ and $\sigma_2 : X_2 \to Y_2$. The strategies work as follows (remember that 
        now we play as system): The first move of environment has to be ``ask' (in $Y_1 + Y_2$), 
        we reply with ``ask'' in $X_1 + X_2$, then environment can only reply with either ``left'' or ``right''
        in $X_1 + X_2$ -- if it replies ``left'', we reply with ``left'' in $Y_1 + Y_2$ and contiune playing according to $\sigma_1$, 
        else we reply ``right'' and continue playing according to $\sigma_2$.
        For $\sigma_1 + \sigma_2$ defined in this way, it is not hard to see that if $\sem{\sigma_1} = f_1$ and $\sem{\sigma_2} = f_2$, 
        then $\sem{\sigma_1 + \sigma_2} = f + g$. This because by definition of strategy compsition, $\mathtt{val}$ and $+$, 
        we know that $\nu; (\sigma_1 + \sigma_2)$ is either equal to (a) $\nu'; \sigma_1$ for some $\nu'$ 
        such that $\mathtt{val}(\nu) = \mathtt{left}(\mathtt{val(\nu_1)})$, or
        $\nu'; \sigma_2$ for some $\nu'$  such that $\mathtt{val}(\nu) = \mathtt{right}(\mathtt{val(\nu')})$. 

        \item[$f \& g$] Here the construction is similar to the one for $+$: We strat by showing 
        how to construct a strategy $\sigma_1 + \sigma_2 : (X_1 \& X_2) \to (Y_1 \& Y_2)$ 
        $\sigma_1 : X_1 \to Y_1$ and $\sigma_2 : X_2 \to Y_2$: The first move of environment has to be either 
        ``left'' or ``right'' in $Y_1 \& Y_2$. We respond with the same in $X_1 \& X_2$. The only move of 
        the environment is now to ``acknowledge'' in $Y_1 \& Y_2$, after which we ``acknowledge'' in $X_1 \& X_2$. 
        Then we play either according to $\sigma_1$ or $\sigma_2$ (depending on whether the first move of the environment
        was ``left'' or ``right''). To see that if $\sem{\sigma_1} = f_1$ and $\sem{\sigma_2} = f_2$, then 
        $\sem{\sigma_1 \& \sigma_2} = f_1 \& f_2$, we notice that by definition of composition and $\&$, 
        we know that after environment plays ``left'' in $\nu; (\sigma_1 \& \sigma_2)$ the game proceeds 
        to a state equivalent to $\nu_1 ; \sigma_1$, where $\pi_1(\mathtt{val}(\nu)) = \mathtt{val}(\nu_1)$. 
        (and analogously for ``right'').

        \item[$f \otimes g$] Again, let us start by showing how to constructed
        $\sigma_1 \times \sigma_2 : (X_1 + X_2) \to (Y_1 + Y_2)$ from $\sigma_1 : X_1 \to Y_1$ and $\sigma_2 : X_2 \to Y_2$.
        The first move is by environment. It can either play in $Y_1$ or in $Y_2$. Let us assume that
        it plays in $Y_1$ (as the other case is summetrical). Then, we play according to 
        $\sigma_1$. As $\sigma_1$ is bounded, this means that after some interaction with $X_1$ and the library\footnote{
            It might be worth clarifying that during this time, if we play in $X_1$ the environment needs to respond
            immediately in $X_1$ and if we play in the library, the environment needs to respond immediately in the library, 
            as it has no other available moves. 
        }
        we will respond in $Y_1$. Then the environment can choose again if it wants to play in $Y_1$ or $Y_2$
        if it plays in $Y_1$ we play according to $\sigma_1$, and if it plays in $Y_2$ we play according to $\sigma_2$. 
        We continue in this fashion until the environment has no available moves in $Y_1$ or $Y_2$.
        To see that if $\sem{\sigma_1} = f_1$ and $\sem{\sigma_2} = f_2$, then
        $\sem{\sigma_1 \otimes \sigma_2} = f_1 \otimes f_2$. Let us observe that during the 
        $Y_1$-part of computing $\mathtt{val}(\nu; (f_1 \otimes f_2))$ the game looks exactly the 
        same as the game for computing $\mathtt{val}(\nu_{Y_1}; f_1)$, 
        where $\nu_{Y_1}$ is define as the plays subset of plays from $\nu$, where 
        environment never plays in $Y_2$. This is a strategy in $1 \to Y_1$, because if 
        the environment never plays in $Y_1$ then system cannot respond in $Y_1$. 
        It is not hard to see that $\mathtt{val}(\nu_{Y_1}) = \pi_1(\mathtt{val}(\nu))$. 
        It follows that during the $Y_1$-part of the play for computing $\mathtt{val}(\nu; (\sigma_1 \otimes \sigma_2))$, 
        we obtain $f_1(\pi_1(\mathtt{val}(\nu)))$. Similarly, one can show that during the $Y_2$ part of computing
        $\mathtt{val}(\nu; (\sigma_1 \otimes \sigma_2))$, we obtain $f_2(\pi_2(\mathtt{val}(\nu)))$. It follows that: 
        \[ \mathtt{val}(\nu; (\sigma_1 \otimes \sigma_2)) \quad = \quad \bigl( f_1(\pi_1(\mathtt{val}(\nu))),\ f_2(\pi_2(\mathtt{val}(\nu))) \bigr) \quad = \quad (f_1 \otimes f_2) (\mathtt{val}(\nu))\]
    \end{description}
    Let us now deal with prime functions. For the sake of brevity we only show how to implement them 
    for the following representative set of examples (all of the prime functions not listed below, 
    can be implemneted without ever playing the library, and are well known tautoligies of the affine logic):
    \begin{description}
        \item[$\textrm{eq} : \atoms \otimes \atoms \to 1 + 1$]:
        For this funcion the system wants to call the library function for comparing atoms 
        and return the result. Here is the exact strategy:    
        First, the environment has to play ``ask'' in $1+1$. Then we play ``request'' in the left $\atoms$.
        Environment replies with ``grant'' which is a write move. Then we play the read 
        move in atoms equality library function. Environment acknowledges. We play ``request'' in the 
        right $\atoms$. Environment plays ``grant''. We play the read move in the atoms equality function. 
        The environment replies with either $=$ or $\neq$, to which we respond respectively with ``left''
        or ``right'' in $1+1$.

        \item[$\textrm{const}_a : 1 \to \atoms$] For this function we want to call the constant
        functionality of the library. Here is the exact strategy: First, the environment has to play 
        ``request'' in $\atoms$. We respond in the $a$-move in the constant library component. 
        The environment has to respond with the write move, we respond with ``grant''. 

        \item[$\textrm{proj}_1 : X \otimes Y \to X$] Here we play a restricted version of the copycat strategy:
        whenever the environment plays in the right copy of $X$, we play the same move in he left copy of $X$, 
        and whenever the environment responds in the left copy of $X$, we respond with the same move in the right 
        copy of $X$. 

        \item[$\textrm{distr}_{\&, \otimes} : X \otimes (Y \& Z) \to (X \otimes Y) \& (X \otimes Z)$]
        Here we play according to the strategy from \ref{ex:amp-otimes-distr}: The environment 
        starts by either playin ``left'' or ``right'' in $(X \otimes Y) \& (X \otimes Z)$, 
        we respond with the same move in $(Y \& Z)$. The environment has to play ``acknowledge'' 
        in $(Y \& Z)$, after which we play acknowledge. This leaves in a state that is either 
        equivalent to $X \otimes Y \to X \times Y$ or  $X \otimes Z \to X \times Z$. 
        In both cases we play according to the copycat strategy.  \rafal{Define copycat?}
    \end{description}
\end{proof}
One can also show that the system is sound (i.e. that every function repersented by a strategy is single-use). \rafal{reference 
to a claim, which follows from SMCC-ness. We don't defer the proof of completeness because the proof of SMCC-ness depends on it.}

\subsection{The set of strategies as a linear type}
The purpose of this section is to show that the set of strategies in an arena for a function type $X \to Y$ can be described using some linear type, and furthermore the relevant operations on strategies, such as application and currying, can be performed in a single-use way.

Before we do this we introudce the idea of a \emph{partially applied arena}.
Let $A$ be an arena, and $m$ a move. We define the partially applied arena ${m^{-1}}A$ 
to be set of sequences that follow $m$ in $A$, i.e:
\[m^{-1}A = \{s \ | \ ms \in A \}\]
The partially applied arena also remembers two bits of information about its context: (a) wheather the current move belongs to the Environment or to the System
(b) wheter the previous move was a write move (so that we know if the next move has to be a write move). We finish our discussion with the observation 
that if moves $a, b$ belong to $\atoms$ (i.e. they are constant requests from the library), then $a^{-1}A = b^{-1}A$. 
For this reason we define the notation $\atoms^{-1}A$ for partrially applying any of the atomic moves. 


We are now ready the type for storing $k$-bounded strategies. 
\begin{definition}
    For a (partially applied) arena $A$, we define the type $\Strat(A, k)$ of $k$-bounded strategies on $A$ using the following induction:
\begin{itemize}
    \item $k=0$. This type is $1$ if the Environment owns $A$ and has no moves to play. Otherwise, this type is $\emptyset$.
          (Observe that $\emptyset$ is not a valid type in our language, so in the inductive step we are going to handle it explicitly --
          the main idea is that if its used with $+$ then it is going to be ignored,
          and if it is used with $\&$ then the entire type is going to be $\emptyset$.) 
    \item $k + 1$. This type is defined differently, depending on whether Environment or System owns $A$.
    \begin{itemize}
        \item Environment owns $A$. Let $m_1,\ldots,m_n$ be the set of possible first moves. (This set is finite.) The type is defined to be 
        \begin{align*}
            \Strat(m_1^{-1}A, k) \& \ldots \& \Strat(m_n^{-1}A, k).
        \end{align*}
        Let us briefly mention how to handle edge cases: If at least one $\Strat(m_n^{-1}A, k)$ is equal to $\varnothing$, then 
        the entire type is equal to $\varnothing$. If there are no valid moves $m_i$, then the entire type is equal to $1$. 
        \item System owns $A$. Now we also consider two options, depending on whether the previous move by Environment was a write move. 
               If it was, then let $m_1, \ldots, m_n$ be the current set of possible read moves that are not the first moves in any library 
               component. (This set is finite). Then our type is defined as follows:
               \begin{align*}
                \Strat(m_1^{-1}A, k) + \ldots + \Strat(m_n^{-1}A, k) + \underbrace{\Strat({\text{eq}}^{-1}A, k)}_{\substack{\textrm{Start a new call to}\\
                \textrm{atoms comparing}\\
                \textrm{library function}}}.
               \end{align*}
               The other case is when the previous move was not write. Then let $m_1, \ldots, m_n$ be the set of avaliable moves that are not read moves. 
               (This set is finite again.) The the strategy type is defined as: 
               \begin{align*}
                \Strat(m_1^{-1}A, k) + \ldots + \Strat(m_n^{-1}A, k) + \underbrace{\atoms \otimes \Strat(\atoms^{-1}A, k)}_{\substack{\textrm{Start a new call to}\\
                \textrm{atomic constants}\\
                \textrm{library function}}}.
               \end{align*}
               Again, let us mention the edge cases: If some $\Strat(m^{-1}A, k) = \varnothing$ (including the cases where $m$ is the new library move, i.e. $\atoms$ or $\text{eq}$)
               then we simply skip it. If all $\Strat(m_i^{-1}A, k) = \varnothing$, then the entire type is equal to $\varnothing$. 
    \end{itemize}
\end{itemize}
\end{definition}
The intuition behind this defintion is as foollows. If the current move belong to the environment, then the strategy 
for the system needs to be prepared for every possible move by the environment. Hoewever, eventually the environment 
will play only one move, so only one of those paths will materialize. This behaviour corresponds exactly to the connective $\&$. 
(Actually this is the reason why introducing $\&$ makes our single-use categry symetric monoidal closed.) 
When the system is about to move, then it needs to pick one of the possible moves. This corresponds to the behaviour of the connective $+$.

In order to find a linear type that could represent all fuctions in $X \to Y$, we are going to show 
that every $X$ and $Y$ there is a bound $k$, such that every single use function $f : X \to Y$
can be represented as a $k$-bounded strategy, i.e. as $\sigma \in \Strat(X \to Y, k)$. Before we do this, 
let us point out potential problems. 

The main problem is that (as long as $Y \not \approx 1$), there is no universal bound $k$ for strategies in the arena for $X \to Y$
(even though each strategy is bounded by some $k$). The main reason for this is that the system might ask an arbitary number of irrelevant questions, 
i.e. questions that use the equality test on two constants. Interestingly, the irrelevant questions might appear as a result of compositions of strategies:
% The following lemma shows that if irrelevant questions are disallowed, then plays have length bounded by some constant that depends only on the types involved. This will be one of the useful properties of such strategies, since it will allow us to represent them using a finite linear type.
% \begin{lemma}\label{lem:no-irrelevant-questions-are-bounded}
%     For every linear types $X$ and $Y$ there is some $k$ such that if a strategy in the arena for $X \to Y$ does not ask irrelevant questions, then all plays in the strategy have length at most $k$.
% \end{lemma}
\begin{example}[irrelevant questions from composition]\label{ex:irrelevant-questions-from-composition}
    Consider the  following two functions: 
    \begin{enumerate}
        \item the function of type  $1 \to \atoms \otimes \atoms$ that outputs the pair (Mark, John) for its unique input;
        \item the equality test of type $\atoms \otimes \atoms \to 1 + 1$.
    \end{enumerate}
    Like all single-use functions, the above two functions have natural representations that do not ask irrelevant questions.
    However, if we apply the composition construction from Section~\ref{sec:composition-of-strategies} to these two strategies, we will get a strategy that asks the irrelevant question of whether Mark and John are equal. \exampleend
\end{example}

The idea of irrelavent questions captures the intuition of why strategies migh be arbitrarily long, but it is not the only reason for this.
This is beacuse the system might use constants to start an arbitary number of parallel calls to the equality test library function, 
never completing any of them. This is why we need to introduce the following definition:
\begin{definition}
    We say that a strategy is \emph{well-formed} if it never starts a new call to the equality test library function, using a constant. 
\end{definition}

\noindent
Well-formed strategies are bounded, as shown by the following lemma:
\begin{lemma}
    For every linear types $X$ and $Y$ there is a univeral bound $k$, such that all well-formed strategies in $X \to Y$ are bounded by $k$. 
\end{lemma}
\begin{proof}
    We start by defininig the dimention of a type to be the maximal number of $\atoms$ that the type can store at the same time. This is defined inductively, 
    where $\dim(1) = 0$, $\dim(\atoms) = 1$, and the dimention of complex types is computed as follows:
    \[ \begin{tabular}{ccc}
        $\dim(X + Y) = \dim(X \& Y) = \max(\dim(X), \dim(Y))$ & and & $\dim(X \otimes Y) = \dim(X) + \dim(Y)$
    \end{tabular}
    \]
    This is because in case of $X \otimes Y$ both $X$ and $Y$ exist at the same times, and in case of $X + Y$ and $X \& Y$ the values $X$ or $Y$
    only one of them can materialize. 

    The key observation is that in a well-formed strategy, every constant granted by the environment is either compared with an atom from input, 
    or moved to the output. It follows that in a well-formed strategy, the system can ask for at most $\dim(X) + \dim(Y)$ constants
    (because, due to the immediate read rule, each constant has to be used). It is now not hard to see that the lenghts of well-formed strategies 
    in $X \to Y$ are bounded by some $k$ that depends only on the types $X$ and $Y$.
\end{proof}

Since, as illustrated by Example~\ref{ex:irrelevant-questions-from-composition}, well-formedness is not preserved under composition of strategies. 
For this reason we need that every non-well-formed strategy can be made-well formed in a way that preserves its sematics (later we will also 
show that this can be done in a single-use way). Before we do this, we would like to assume a weaker property of the possibly non-well-formed strategy 
of the input:
\begin{definition}
    A strategy is called \emph{read-consistent} if it always uses the same move to consume a value after requesting it from the environment. 
    This request can be made either through a move in the constant library component or a "request" move in an $\atoms$ on the input.


\end{definition}










In light of the above example, we will want to eliminate irrelevant questions. 

\begin{lemma}\label{lem:eliminate-irrelevant-questions}
    For every linear types $X$ and $Y$, every single-use function of type  $X \to Y$ is represented by  some strategy that does not ask irrelevant questions.
\end{lemma}


The following lemma shows that the set of $k$-bounded strategies can be described using a linear type.

\begin{lemma}\label{lem:linear-type-of-k-bounded-strategies}
    Let $X$ and $Y$ be linear types, and let $k \in \set{0,1,\ldots}$. One can find: 
    \begin{enumerate}
        \item \label{it:k-bounded-type} a linear type, call it $X \Rightarrow_k Y$;
        \item \label{it:k-bounded-bijection} a bijection of its underlying set with the set of $k$-bounded strategies in the arena for $X \to Y$;
        \item \label{it:k-bounded-eval} a single-use function of type $X \otimes (X \Rightarrow_k Y) \to Y$;\end{enumerate}
    such that for every $k$-bounded strategy $\sigma$, the following diagram commutes:
    \[
    \begin{tikzcd}[column sep=3.5cm]
    X  
    \ar[rr,"{x \mapsto x \otimes \text{(element corresponding to $\sigma$ via bijection in item~\ref{it:k-bounded-bijection}})}"]
    \ar[dr,"{\text{function represented by $\sigma$}}"']
    & & 
    X \otimes (X \Rightarrow_k Y)
    \ar[dl,"\text{function from item~\ref{it:k-bounded-eval}}"]
    \\
    & Y
    \end{tikzcd}
    \]    
\end{lemma}




\begin{proof}[Proof of Theorem~\ref{thm:single-use-closed}]
    Consider two types $X$ and $Y$.
 Take $k$ to be the number from Lemma~\ref{lem:no-irrelevant-questions-are-bounded}. This ensures that the type $X \Rightarrow_k Y$ is rich enough to represent all strategies that do not ask irrelevant questions. By Lemma~\ref{lem:eliminate-irrelevant-questions} the type is  rich enough to represent all single-use functions. Define the  function space $X \Rightarrow Y$ to be this type.   Define the evaluation function 
    \begin{align*}
    \text{eval} : X \otimes (X \Rightarrow Y) \to Y 
    \end{align*}
    to be the evaluation function in item~\ref{it:k-bounded-eval} in Lemma~\ref{lem:linear-type-of-k-bounded-strategies}. We will now prove that this type admits currying. Consider then some single use function 
    \begin{align*}
    f : Z \otimes X \to Y.
    \end{align*}
    By Lemma~\ref{lem:bounded-k}, there is some $\ell$ such that $f$ is represented by some  $\ell$-bounded strategy $\sigma$ in the game of type $Z \otimes X \to Y$. For strategies, there is a straightforward notion of currying, which is single-use, as given in the following claim.

    \begin{claim}
        Let $X$, $Y$, and $Z$ be linear types, and let $\ell \in \set{0,1,\ldots}$. For every $\ell$-bounded strategy $\sigma$ in the game  of type $Z \otimes X \to Y$  there is a single-use function $\Lambda : Z \to (X \Rightarrow_\ell Y)$, such that the following diagram commutes:
        \[
        \begin{tikzcd}
            [column sep=3.5cm]
        X
        \ar[r, "{x \mapsto z \otimes x}"]
        \ar[dr, "\text{strategy represented by $\Lambda(z)$}"']
        &
        Z \otimes X 
        \ar[d, "{\text{strategy represented by $\sigma$}}"] \\
        &
        Y
        \end{tikzcd}
        \]
        \end{claim}
        The currying operation from the above claim gives us almost the currying that is required in Theorem~\ref{thm:single-use-closed}. The only issue is that the output is in $X \Rightarrow_\ell Y$, where the number $\ell$ depends also on the extra set $Z$, instead of $X \Rightarrow_k Y$. To fix this, we post-compose it with the single-use function from $X \Rightarrow_\ell Y$ to $X \Rightarrow_k Y$  that was described in the remarks after Lemma~\ref{lem:bounded-k}. 
    

\end{proof}

\section{A quotient construction}
\label{sec:quotient-category}
A drawback of Theorem~\ref{thm:single-use-closed} is that the function space $\funspace V W$ can contain different representations of the same function; this will mean that Currying is not unique. To overcome this issue, we use a simple quotient construction. 
Define a \emph{partial equivalence relation} to be a relation that is symmetric and transitive, but not necessarily reflexive.  This is the same as (complete) equivalence relation on some subset. We will use a partial equivalence on the function space $X \Rightarrow Y$ to: (1)
 remove objects that do not represent any morphism; (2) identify two objects if they represent the same morphism. After such a quotient, the function space will have unique representations for functions. 




\begin{definition}
    The \emph{quotiented single-use category} is: 
    \begin{itemize}
    \item Objects are pairs (linear type $X$, equivariant partial equivalence relation on $\sem X$);
    \item Morphisms between objects $(X,\sim_X)$ and $(Y,\sim_Y)$ are single-use functions from $\sem X$ to $\sem Y$ such that the domain of the function is contained in the domain of $\sim_X$, and equivalent inputs are mapped to equivalent outputs.
    \end{itemize}
\end{definition}

The quotiented single-use category is also equipped with a tensor product $\otimes$ on its objects.
\begin{theorem}
    The quotiented single-use category, equipped with the tensor product $\otimes$, is a monoidal closed category, i.e.~it satisfies the conclusions of Theorem~\ref{thm:single-use-closed}, but futhermore the morphism $h$ is unique.
\end{theorem}


\section{Beyond equality}
\label{sec:beyond-equality-appendix}

As mentioned in the main body of the paper, the proof of Theorem~\ref{thm:single-use-closed} extends without any difficulty to the case of relational structures. We only prove the decidability result from Theorem~\ref{thm:first-order-decidable}.


\begin{proof}[Proof of Theorem~\ref{thm:first-order-decidable}]
    When defining the category of single-use functions, we viewed it as a restriction of larger category, in the morphisms were equivariant functions.  We begin by describing the generalization of this larger category to the case where the parameter $\atoms$ is some relational structure. The idea is to use first-order definable functions as the general analogue of equivariant functions. 

\begin{definition}[Category of first-order definable functions]
    Let $\atoms$ be a relational structure. The \emph{category of first-order definable functions over $\atoms$} is defined as follows.
    \begin{itemize}
        \item The objects are polynomial sets over $\atoms$, i.e.~finite disjoint unions of powers of $\atoms$.
        \item The morphisms are the  first-order definable functions. Here, a function 
        \begin{align*}
            f : \atoms^{n_1} + \cdots + \atoms^{n_k} \to \atoms^{m_1} + \cdots + \atoms^{m_\ell}
            \end{align*}
            is called  \emph{first-order definable} if for every $i \in \set{1,\ldots,k}$ and  $j \in \set{1,\ldots,\ell}$ the set 
            \begin{align*}
            \setbuild{ (x,y) \in \atoms^{n_i} \times \atoms^{m_j}}{f(\text{$x$ in the $i$-th component}) = \text{$y$ in the $j$-th component}}
            \end{align*}
            can be defined by a first-order formula over the vocabulary of $\atoms$.
    \end{itemize}
\end{definition}
The above is easily seen to be a category, since first-order definable functions  can be composed. 
If the underlying structure $\atoms$ has a decidable first-order theory, then the category described above is reasonably tame, in particular one can decide if two morphisms are equal.

The single-use category admits a faithful functor to the above category. In particular, this implies that the problem of deciding if two morphisms are equal is decidable in the single-use category. 
\end{proof}







\end{document}
